\documentclass{jarticle}
\title{2025年度 各係の引継ぎ資料}
\date{\today}
\usepackage{input}
\usepackage{color}
\usepackage{ulem}%2017_2_2 \\sout{}で取り消し線に出来る(冨澤)
\usepackage{ascmac}%20150314_追加(浦山)
\usepackage{atbegshi}%20160321 nishio add
\usepackage{listings}%20220404 ekawaguchi add
\ifnum 42146=\euc"A4A2
  \AtBeginShipoutFirst{\special{pdf:tounicode EUC-UCS2}}
\else
  \AtBeginShipoutFirst{\special{pdf:tounicode 90ms-RKSJ-UCS2}}
\fi

% しおり(ブックマーク)の文字化けを回避するおまじない 20260216 織井
\usepackage[
  dvipdfmx,
  pdfencoding=auto,
  colorlinks=false,
  bookmarks=true,
  bookmarksnumbered=false,
  pdfborder={0 0 0},
  setpagesize=false
]{hyperref}
\usepackage{pxjahyper}


\begin{document}
\maketitle

2025年度の各係の引継ぎ内容を示します.各係は以下の引継ぎ内容を確認してください.以下の内容を確認し,よりよい研究室生活を送ってください.

\tableofcontents
 %{\bf inputで読み込んだファイルの中身ここから}\\

\clearpage
\begin{inputs}
\usepackage{input}
\usepackage{color}
\def\red{\color{red}}
\def\black{\color{black}}
  \newcommand{\test}{waya}

\begin{document}
\maketitle
\section{室長}
室長の大まかな仕事内容を以下に示します.
\begin{itemize}
  \item 夜間申請と休日申請を先生に提出(コロナの時は入校申請)
  \item 副室長と院ゼミ・学部ゼミの日程決め
  \item オープンキャンパスの仕切り
  \item 3年生研究室見学会の仕切り
  \item ミーティングの時間決めおよび司会進行
  \item 発表会・報告会の出欠確認
  \item \sout{卒論の製本カバーを買う}
\end{itemize}

\subsection{夜間申請と休日申請を先生に提出}
\red2023年度の途中から,これらの申請は教員が一括で行うことになりました.\black
ここでは,夜間申請と休日申請について説明します.必要な資料は,
\par
/workspace/Work2020/makino/引き継ぎ資料/室長関係/申請書類2020
\par
にあります.
\par
毎週月曜日に夜間申請を関口先生に提出します.2019,2020年度は月曜日でしたが,曜日は年度によって変わる可能性があるため,最初に先生に渡す日程を確認してください.
2020年度からコロナの関係で総研と世田谷で分かれて活動しています.
夜間申請は総研と世田谷でそれぞれ提出が必要です「【総研】研究室・実験室等(夜間)使用願.docx」のWordファイルを使用してください.
これからこのWordファイルの夜間申請の記入箇所について記します.最初に夜間申請を記入する際には以下のことを行ってください.

\begin{itemize}
	\item 4年生の指名と学籍番号を「使用者名簿」に新しく記入してください.
  	\item M1に関しては学籍番号が変更するため,確認して「使用者名簿」に記入してください.
  	\item 「使用学生代表者」に自分の名前,電話番号,学籍番号,学年を記入してください.
\end{itemize}

毎週提出する箇所として以下の2つです.基本的に夜間申請書は同じファイルで使いまわします.

\begin{itemize}
  	\item 1ページ目右上の「届出日」を変更して下さい.今年度は月曜日に先生に渡していて,提出日は渡した翌日とかにしてました.
  	\item 2ページ目の最後の「利用週」を提出する次の週の月曜日から土曜日までの日付を書いてください.
  	\item 世田谷の申請書は,利用時間を記入する欄があります.平日は夜間利用時間に当たる20:00~23:00とし,休日は8:00~20:00としてください.土曜は休日に当たります.
\end{itemize}

過去には(2016年度)関口先生からの提案で”月の初めに,その月の夜間申請をまとめて渡す”形になったそうです.もしかしたらそのような感じになるかもしれないので,夜間申請を週ごとに提出するのか,月ごとにまとめて渡すのか、を先生に確認してみてください.祝日や,入試などで総研に入れない日の場合は,夜間申請書の1pの使用時間におけるチェック欄と2pの使用日の表における使用時間,終了時間を変更してください.
\par
次に休日申請について説明します.総合研究所においては,学年歴に従い休館日になっている場合があります.休館日に利用する場合は,夜間申請とは別に休日使用申請を出す必要があります.休日申請書は,先ほどのファイル内の「夜間・休業日使用願.docx」を使用します.基本的な記入内容は夜間申請と同じでうが,申請する日付けを記入することを忘れないようにして下さい.
総研はキャンパスとは別の機関なので,休日等も学年歴と違うことがあります.長期休暇の前などは事務室に確認することをお勧めします.
\par
ちなみに夜間申請の提出を忘れると,研究室に20時以降入れなくなるので注意してください.

次に2020年度でのコロナの状況での入校申請について記します.
2020年度の主に緊急事態宣言下の状況では,入校すること自体に申請書が必要になっていました.なので,申請書を週ごとに作成して,ワンドライブ上に置き,各自で記入してもらいましょう.室長としてはファイルの準備と,締め切りの勧告.また関口先生への提出を行ってください.
提出するタイミングは先生と話て決めてください.2020年度は前期は研究の進捗として急遽実験等が入ることが少ないので入る週の前の週の月曜日としてました.1月~は研究も佳境に差し掛かっていたので,入る週の前の週の金曜の午前中にしていました.

%---2019まで-----%
%\subsection{院ゼミ・学部ゼミの日程およびページ決め}

%2019年度は院ゼミと学部ゼミが週に1回ありました.行う曜日は月ごとに先生やゼミを行う人と相談して決めてください.院ゼミが何曜日,学部ゼミが何曜日と決めといた方がいいのです.学会の日程次第で2回連続で院ゼミを入れてもいいです.でも翌週は学部ゼミを2回連続で入れる様にバランスを取ってください.去年に引き続き今年度も,ゼミ関係は副室長に全面的に任せていました.

%院ゼミについて記します.

%院ゼミは論文紹介を行います.論文紹介は担当者を決め,その人が読んだ論文を紹介するといった形になります.その担当者を決めてください.学会の発表練習がある週に関しては,その学会に何人参加するかを確認し,先生の都合の付く日に発表練習してもらいます.人数が多い際には2日に分ける事もあります.先輩がメインで決めてくださる場合もあるので,その場合は先輩に日程を聞き,ゼミの日程を調整してください.

%学部ゼミについてです.

%学部ゼミでは解説記事の輪講とテキストの輪講や自分の研究に関する論文や解説記事の紹介を行います.2017年度は,年度前半に自分の研究における制御手法の紹介の輪講,年度後半ではテキストの輪講でした.2018年度は,自分の関係する手法の解説でした.2019年度は,2018年度と同様に自分の研究に関する論文の解説を行いました.今年度の内容に関しては先生と相談するようにしてください.また,何のテキストを使うのかは関口先生に確認してください.
%論文の内容を説明する必要があるため,およそ1人1時間は最低でも必要だと考えてください.

\subsection{勉強会の日程,内容決め}
2019年度までは院ゼミや学部ゼミが開催されていました.2020年度は活動が一部ロボ研と合同になったこともあり,勉強会という形で月に2回程度,1回2名の修士生が研究の要素技術の紹介,解説をするものとなっていました.事前に高機能研の発表の内容,順序を先に決めてその後ロボ研含めて日程調整という形だったようです.
この仕事は副室長に全面的にやってもらっていたので,今後もそれでいいかと思います.何かあったら副室長経験者に聞いてみてください.

\subsection{オープンキャンパスの仕切り}
2020年度はオープンキャンパスはありませんでした.そのためオープンキャンパスについては更新しません.

オープンキャンパスに関係する資料はサーバーのWorkSpaceの
\par
/workspace/Work2019/ktoya/引き継ぎ資料/室長関係/オープンキャンパス
\par
にあります.
\par
オープンキャンパスが6月初旬と8月初旬に,3年生研究室見学会が7月下旬~8月上旬にあります.2019年度に関しては,6月のオープンキャンパスでは去年度の先輩のポスターを使って説明をしました.7月下旬~8月上旬では,自分たちでポスターを作成して説明を行いました.恐らく2020年度も同様の方針となると思います.オープンキャンパスでポスターを使って研究紹介をするに当たり,誰がポスターを作り,紹介するのかを事前に決めます.ポスターを作り終えたら,図書館の印刷機の所に行き,印刷します.図書館に行く理由としてA0もしくはA1の用紙を用いるからです.大きさはM1の先輩にどうだったか確認してみてください.オープンキャンパスでは6月は去年のものを,8月は3年生研究室見学会で作成したものを使用します.また,オープンキャンパスではポスターを使って説明をする人のシフト表を作ると便利です.資料内に2019年度のシフト表のExcelファイルもアップしてあるので,参考にしてください.

次にポスター紹介です.3年生研究室見学会は時間制限がありません.しかし,オープンキャンパスには「ツアー」と「一般」に分かれており,「ツアー」は時間制限があります.そのため,最低でも3つのポスター担当者に時間制限を考慮した台本を作ってもらいましょう.ここは忙しくなると思いますので,ある程度の流れを先輩から確認してください.

\subsection{3年生研究室見学会の仕切り}
日程等は先生と相談して決めてください.
例年だと制御システム設計の最終回の授業後とテスト期間後に行っています.
\subsubsection{例年の対面での場合}
3年生研究室見学会に関係する資料はサーバーのWorkSpaceの
\par
/workspace/Work2019/ktoya/引き継ぎ資料/室長関係/研究室見学会
\par
にあります.
\par
ここでは,2019年度の見学会について記述します.2019年度は,車椅子とドローンは総研で,油圧と車両と脚車は世田谷で行いました.内容は,オープンキャンパスと似ていて,ポスターを使ってそれぞれの研究の説明を行います.だたオープンキャンパスよりも詳細な内容を説明するようにしましょう.研究室でどんな研究をやっているのかを説明して,3年生に自分たちの研究室に興味を持ってもらえるようにできると良いです.また,ポスター以外にも実物の展示やデモンストレーションも行います.2019年度は,車椅子とドローンのデモを行いました.デモで何を見せるのか,そのためのプログラムの準備等も発生するので,早めに準備に取り掛かりましょう.詳細はM1に必ず確認を取りましょう.特に2020年度は世田谷は空っぽですので,総研で全ての見学会を行う可能性がありるため,これについても先生や先輩に確認を取りましょう.当然,オープンキャンパスと同様にシフト表やタイムスケジュールを作る事も忘れずに.
\subsubsection{2020年度のコロナでの場合}
昨年度はコロナの都合上研究室紹介をzoom上で2回行うこととなりました.おおよその回し方は対面の場合と変わりません.
その場合の資料を
/workspace/Work2020/makino/引き継ぎ資料/室長関係/研究室見学会

にあります
大きな変更点としては,例年ポスターでやっているところをショートスライドに変更して紹介を行った点です.
回し方ですがブレイクアウトルームを利用しローテーションを行いました.細かくローテーションを行うのは開催的にも参加者的にも難しいことから,2つのセッションに分けて前後半で入れ替えを行いました.
後半のフリータイムではより少人数のほうが多く質問ができるため,ブレイクアウトセッションを再編成して行いました.
時間の縛りがある行事なので,事前にどんなふうに回すかを話し合い,4年生には細かく共有,理解してもらい,協力してくださる修士の先輩方にも共有しましょう.ここは対面でも変わりませんね.


\subsection{ミーティングの時間決めおよび司会進行}
\subsubsection{~2018年度までのミーティング}
2018年度までのミーティングは以下のように行われていたようです.
\par
ミーティングは週一で基本的に先生方両方の都合のつく日に行われます.ミーティングは研究室全員が連絡事項を把握するために行うものです.手順としてまず野中先生と関口先生が空いている時間を自分のわかる範囲で確認して,日程の候補を3パターンほど考えておきます.

その後,先生方に「この曜日の何時から行いたいと考えているのですが、ご都合の方はどうでしょうか?」などと失礼のないようにい聞きにいきます.(先生方に聞きに行くときは,まず,「ミーティングのことに関してなのですが,お時間はありますでしょうか?」と,今聞いて大丈夫なことを確認してから本題に入りましょう.)それで,2人の都合のよい時間に行います.またゼミや発表会がある週はその終了後に行います.(その場合も先生方への確認は行いましょう.)2017年度では野中先生からの提案で関口先生のみに確認を取っていました.年度によって変わると思うので先生方から提案されない限りはどちらの先生にも確認を取るようにしてください.さらになるべく多くの人数が参加できる時間帯をグーグルカレンダーで把握して,ミーティング時間を決定します.

決定したら,先生方に「○時からミーティングを行います.」と伝え,研究室生徒にはカレンダー,メーリングリストやラインなどで促します.その際にラインを見ない人やガラケーの人には直接言ってください.そして,ミーティング時間になったら5階のレグザ前に集まります.学生全員が集まり次第,先生をお呼びして開始となります.ミーティングはあらかじめ係の人に連絡があるのかを聞いておいてください.そうすると円滑に会が進みます.

まずあらかじめ聞いていた係の人たちを順番に指名し,連絡事項を言ってもらってください.
あらかじめ聞いていた係の人が終わったら,確認で「係で連絡ある方いますか?」などと促し,手を挙げた人を指します.学生における連絡がなくなったら,関口先生,野中先生の順で連絡があるかをお聞きします.野中先生の連絡事項が終り次第「ミーティングを終わりにします.お疲れ様でした.」などといって終了となります.

その後,メーリングリストに内容をまとめたものを記し,全員が周知するようにしましょう.

\subsubsection{2019年度のミーティング}
2019年度のミーティングは昨年までとは異なり,週1回行っていました.発表会後やゼミ後など,研究室の全員が集まっているときに行いたかったからです.2019年度は,だいたい毎週何かしらに発表やゼミがあったため,その時間で行うことができました.特に発表等がない週に関しては,ミーティングはやっていませんでした.自分は先生や学生に事前にミーティングの確認などは行っておらず,いきなり始めていました.
\par
ミーティングの流れとして,まず初めに学生で連絡を持っている人に挙手をさせて,順番に指名.学生が終わったら,関口先生,野中先生の順番で行っていました.ミーティング終了後は,Gmailで本日の連絡事項を纏めて送信していました.Gmailを送信するのが面倒な時にはSlackで流していました.

\subsubsection{2020年度のミーティング}
2020年度はコロナの都合上zoomでミーティングを行っていました.情報共有がしずらい環境だったこともあって1週間に1回のペースで行っていました.
日程の決め方については,すべてスラックで行っていました.先生方と自分のダイレクトメッセージを用いて,行事と先生の予定のカレンダーを共有されると思うのでそこから候補の日程を3,4個挙げてその中から都合のいい日程はどこか確認しましょう.
決まったら,スラックのgeneral等で連絡して研究室のGoogleカレンダーに予定として入れておきましょう.

報告会や勉強会がある週については特に連絡や日程の確認はせずに報告会後や勉強会後にやっていました.

進行の流れとしては,「学生→関口先生→野中先生→連絡事項に対しての質疑応答・連絡し忘れたこと」の順番で行っていました.
また,メモを自分でもいいし副室長などにとってもらって,ミーティングの内容をGmailのメーリングリスト等を利用して周知してください.参加できなかった人,参加し忘れた人がいるので面倒化もしれませんが行ってください.

\subsubsection{2023年度のミーティング}
2023年度はコロナが収まったため対面で週に1回行っていました.
先生方の都合により時間が前後する場合がありますが,基本的にはミーティングを行う曜日と時間は固定していました.
ミーティングを行う日時が決定したらgoogleカレンダーに書き込み,zoomリンクも一緒に設定してください,
何らかの理由でミーティングを欠席する人がzoomで参加する場合があるので,必要に応じてzoomを立ち上げてください.

室長はミーティング開始時刻になったら先生方を呼び,前に立って司会進行を務めます.
連絡事項は学生→関口先生→野中先生の順で伝えてもらっていました.
議事録は副室長などにとってもらい,まとめたファイルをslackに流してください.
また,全体ミーティングについてはslackbotを用いてリマインダーを設定することをお勧めします.

\subsection{発表会・報告会の出欠確認}
発表会と報告会の出席は卒業要件に含まれているため,室長は当日の出欠状況を確認し,Excelなどにまとめてください.
また,出欠状況をまとめたファイルは,発表資料と同じ階層にアップロードしてください.(場合によっては関口先生に直接送るよう指示されるかもしれません.)


%\subsection{ワークスペースのlabDataに必要なファイルを作る}
%院ゼミや報告会PPT・報告書など,必要に応じてファイルを作成します.手順を以下に示します.
%\begin{enumerate}
% \item labdataを開く
% \item 新しいフォルダを作る(名前は他人が見てすぐ分かる様に)
% \item フォルダを右クリックして,プロパティを押す
% \item 項目に「セキュリティ」があるので押す
% \item 「編集」を押す
% \item アクセス許可のフルコントロールの項目に「許可」・「拒否」があるから「許可」をクリック
% \item 「OK」を押す
%\end{enumerate}

\subsection{卒論・修論の製本カバーを買う}
論文の製本作業に関係する資料はサーバーのWorkSpaceの
\par
/workspace/Work2020/makino/引き継ぎ資料/室長関係/製本作業
\par
にあります.
\par
論文が仕上がる前に論文を包む製本カバーを買わなければなりません.購入場所は14号館地下1階です.ちなみに,2019年度からPDF提出のため製本は希望者のみになっています.2020年度は希望者が少なかったため研究室にある在庫のみで対処したため購入はしていないです.

手順として,2月下旬に卒論の仮提出があります.そこから追加されるページ数を各学生は検討しているので,論文の最終的なページ数に応じたカバーの厚さをB4・M2の印刷希望者に聞いてください.製本カバーの厚さは論文のページ数によって決まります.例えば論文10枚で製本カバーが1mmを購入といった形です.しかし論文は両面印刷するので,本来のページ数を半分にして計算してください.背幅サイズは1,2,3,4,5,6,9,12,15,18,21,24,27,(30)mmがあるとのことです.

購入する際,ミスがあったときのために予備分を足して発注してもいいと思います.また,去年以前の余りも総研学生室にあるので,先輩に聞いてみてください.発注なら1週間,もしくは2週間あれば確実に渡せると一昨年の売店の方は言っていました.製本カバーの色は白,オリーブ,ミントグレー,ベージュ,ダークグレー,オーシャンブルー,サックスブルー,パステルブルーの8種類で,2017年度の色はダークグレーでした.全員印刷で提出だった時は色はそろえていたようですが,今は記念用なので厚みを調査するときに同時に色の希望もとると良いと思います.

製本は学科事務室で圧着機を借りるので,カバーだけ買います.現在は主に総研で活動しているため等々力キャンパスでも借りれるかもしれないので,来年度は確認してみてください.
購入時のお金についてですが去年は''研究室に配分された予算(名義は関口先生)''としました.変更がある可能性もあるので事前に購入時のお金が2017年度と同じなのかを関口先生に確認を取ってください.圧着機のレンタル期間ですが事前に先生方にレンタル期間について相談してください.例年だと卒論最終稿の提出における翌日からの5日程度ですが,2017年度はMSCSで印刷物が多くなる関係でレンタル期間が後ろにずれました.2019年度は3月の1週目にレンタルしました.2020年度はコロナの緊急事態宣言が終わる予定だった3/8から4日間としてました.

製本作業における論文の印刷ですが,2018年度以前は情報基盤センターの各自の印刷ポイントを全て使用し,足りなかった分を研究室で印刷していたようです.2019年度は希望者のみの製本作業であったため,研究室のプリンタで印刷を行いました.レンタル後テンプレートに基づいて印刷したタイトルを背表紙に透明テープで貼り付けます.背表紙についてですが,上記のサーバーのフォルダ内のWordファイルにあります.

背表紙のフォントは設定されているMS明朝の12ptで,縦書きです.数字・英字がある場合には縦中横を使って向きを縦にあわせてください.タイトルが長く,1行では収まらない場合は,まず名前の前後のTabスペースを全角1つ分にして収まるかどうか確認してください.それでも収まらない場合はスペースを半角へ,それでもダメだった場合はフォントを1段階ずつ小さくしていってください.名前の間にスペースは入れません.また,行間は1.5に設定しています.

2019年度の予定を示します.2020年度は少し遅れたので,こちらを参考にしてください.
\begin{table}[h]
\centering 
\begin{tabular}{|c|c|}
\hline
\textbf{日付}  & \textbf{内容} \\ \hline
12/31        & 第1稿提出       \\ \hline
1/31         & 第2稿提出       \\ \hline
2月2週目        & 製本カバー余り確認   \\ \hline
2/28         & 論文提出(PDF)   \\ \hline
3/2          & 製本機レンタル     \\ \hline
3/2$\sim$3/6 & 製本作業(希望者)   \\ \hline
\end{tabular}
\end{table}

\newpage%改ページ
\subsection{1年間のスケジュール}
室長が関わる研究室の1年間の予定を表で示しておきます.なお,下記の予定は2020年度のスケジュールです.コロナの影響で全体的に遅れています.
\begin{table}[h]
\centering 
\begin{tabular}{|c|c|}
\hline
\textbf{日付} & \textbf{スケジュール} 							\\ \hline
4/2       	& 研究室ガイダンス        								\\ \hline
6/16      	& オープンキャンパス(中止)       						\\ \hline
8/11      	& 研究室紹介(オンライン)      						\\ \hline
8/14      	& 研究室紹介(オンライン)       						\\ \hline
8/23,24  	& オープンキャンパス(オンラインのため不参加)       \\ \hline
9/24      	& 研究室配属会          								\\ \hline
9/29      	& 初回事例研究          								\\ \hline
2月中    	& 論文製本作業準備        							\\ \hline
3月初週 	& 製本作業            									\\ \hline
\end{tabular}
\end{table}

例年通りのスケージュールとして2019年度のスケジュールも示しておきます.
\begin{table}[h]
\centering 
\begin{tabular}{|c|c|}
\hline
\textbf{日付} & \textbf{スケジュール} 		\\ \hline
4/2         & 研究室ガイダンス        		\\ \hline
6/16        & オープンキャンパス       		\\ \hline
7/19        & 研究室紹介						\\ \hline
7/23        & 研究室紹介          			\\ \hline
8/2,3       & オープンキャンパス       			\\ \hline
9/19        & 研究室配属会          		\\ \hline
9/24        & 初回事例研究          		\\ \hline
2月中         & 論文製本作業準備        	\\ \hline
3月初週        & 製本作業            		\\ \hline
\end{tabular}
\end{table}
\subsection{最後に}
様々な仕事を任されてると思いますが,決して一人で抱え込まずにみんなで協力して取り組みましょう.
自分は室長の仕事をネガティブな気持ちで取り組むことが多く,せっかくの機会だからもっとまじめに取り組めばよかったと後悔しています.
ストレスがかかる場面も多いとは思うけど,ぜひポジティブな気持ちで取り組んでほしいなと思います.
きっといい経験になると思います.頑張ってください.


\end{document}

\end{inputs}

\clearpage
\begin{inputs}
\usepackage{input}
  \newcommand{\test}{waya}
\begin{document}
\maketitle
\section{副室長}
%%%%%%%%%%%%%%%%%%%%%%%%%%%%%%%%%%%%%

本章では2020年度の副室長が行っていた業務を記しています.
リモート体制にになったり,ロボ研と合同になったりしたので,数点例年と異なります.
\begin{description}
 \item[~室長の補佐~]\mbox{}\\
	    副室長のメインとなる業務です.主な内容としては
	\begin{itemize}
		\item 全体ミーティングのとき室長が司会進行なので内容をメモる
		\item 室長の仕事を手伝う
		\item 室長不在の時には室長の仕事を行う(ミーティングの司会,夜間申請の提出など)
	\end{itemize}
です.

室長は仕事が沢山あります.また,イレギュラーな仕事も沢山出てきます.
しっかり連絡取り合って,出来る限りのサポートしましょう.
そのためにも,室長の仕事を把握する必要があるので,この引継ぎ資料の室長の項目も読んでおきましょう.
スラックでは野中先生関口先生室長副室長,関口先生室長副室長のグループを作って業務連絡などしていました.
その方が室長が副室長に仕事の説明をする手間が省けます.

全体ミーティングのメモは,ミーティング後に室長がGmailで研究室全体に全体ミーティングの内容を送るときに使って貰っていました.タイピングなどに慣れるまでは,メモ役はもう1人募ってもいいかもしれません.


\item[~カレンダーの管理~]\mbox{}\\
各自で記入してくれたので,特にこれといったことはしていませんでした.
気づいたときに書いていない行事が無いかの確認くらいでした.
室長副室長は,様々な仕事で先生方とコンタクト取る必要があったので,野中先生関口先生のGoogleカレンダーの予定(tcu関連のお仕事)も特別に共有して頂きました.


\item[~勉強会のスケジュール管理~]\mbox{}\\
例年は,院ゼミでは院生が学部ゼミでは学部生が発表していましたが2020年度はロボ研と合同になったため院生のみ発表でした.名前も院ゼミから勉強会に代わりました.
スケジュールは基本1人で組んでいましたが,月末に開催する場合は報告会の週と被らないように発表係と一緒に開催日を組んでいました.

 \begin{description}
  \item[勉強会は1週間に1回]\mbox{}\\
 その年度の研究室の人数で変わりますが基本的には1週間に1回2人ずつ発表することを意識してスケジュールを立てましょう.また,例年は11月中に終了が目処でしたが,2020年度から勉強会の終了時期は夏休み前がマストになりました.これが決まったのが7月始め辺りだったので,夏休みに食い込み,8月中旬に終わりました.2021年度以降からも同様の体制ならば7月中に終わらせましょう.
 開催時期は4月はm1は研究が固まっていない,m2は就活関連であまり時間がないなどがあり5月初週から始めました.

 また,勉強会より発表会報告会が優先であるため発表会係と連絡を取り,早めに決めていく必要があります.
 そのため,発表会係に早めに決めてもらうように促すようにすることも大事です.
 目安としては,勉強会の発表2,3週間前には発表者に伝える必要があるので3週間前には日程が決まっているとベストです.発表者はスライド作成やプログラム作成などがあるので.

 研究室は多くの行事があるため,先延ばしにすると休日や長期休みに開催なんてこともあり得ます.
 月にゼミを何度行うか決めたら守るように頑張りましょう.


 \item[勉強会の発表順番]\mbox{}\\
 4年の序盤は研究室に関する知識等がないので,勉強会の発表順番は先輩方に決めてもらいました.院生全員と副室長をメンバーにした勉強会用のスラックのチャンネルを作成し,そこで基本先輩方で動いてもらいました.
 ここのグループで副室長が動いていたことは,発表順番以外に関する連絡です.
 2021年度からスラックにロボ研が加わったのでロボ研の方もチャンネルに入れると便利かもしれません.

 発表順番は,高機能とロボ研が交互に発表していました.勉強会は高機能のみのイベントかつ高機能の方が院生の人数が多かったため序盤は高機能が連続して行い,残りの人数がロボ研と同じになったら交互にしていました.ここは年度によって臨機応変にして下さい.

  \item[他行事との調整]\mbox{}\\
   研究室の行事は報告会やPC(Presentation Competition)など色々あります.研究の発表会や学会の発表練習などを行う週には勉強会を入れないようにしましょう.発表会係や先輩と話し合いながら調整して下さい.調整した日で開催可能か最終確認として先生方とロボ研に確認取ってください.佐藤先生にはロボ研の人に確認を取って貰っていました.

 \item[勉強会を行える日時を把握する]\mbox{}\\
   先生は授業や会議などで忙しいです.授業の曜日は決まっているため,自然とゼミを行える曜日は決まっていきます.基本的に行える曜日・時間を把握しておくと日程が組みやすいです.また2020年度は2019年度までと異なりリモートだったのでゼミの日程はとりにくいとは感じませんでした.\\
   ちなみに,2020年度は前期後期共に基本木曜日の3限に行っていました.

 \item[担当されていたゼミ当日に風邪や体調不良で欠席した人が出た場合]\mbox{}\\
   担当していたゼミ当日に風邪や体調不良などの原因によって欠席される人もいます.その場合は代わりにどの日でやるのかということを話し合って,早いうちにスケジュールに反映させましょう.
 \end{description}


研究室内の人とマメにコミュニケーションをとって,「いつ」「誰が」「何を」行うのか把握しておくとスケジュールが組みやすいです.
多くの先輩が学会に参加したり授業があったりするので,日程が決まったら早めに連絡しましょう.

その際に先輩から助言を頂くのも大いにありだと思います.
\end{description}

2020年度の副室長の業務内容の説明は以上です.

色んな人と連絡取りあって情報共有をしましょう.
リモート体制なら週に1回でも4年全体でzoomをしてもいいかもしれないです.
2020年度はこれをしなくて特定の人にだけ仕事が多くなるなんてことがありました.

また,ここに記した以外の仕事もよく入ってきます.
先輩や同期など色んな人と連絡取って上手くこなしてください.特に,室長とはよく連絡取り合ってください.








**以下のものは対面で活動できた2019年度のものです.**

\begin{description}
 \item[~室長の補佐~]\mbox{}\\
	    副室長のメインとなる業務です.主な内容としては
	\begin{itemize}
		\item ミーティングのときは室長と一緒に真ん中に立ちメモを取る
		\item ミーティングの内容を室長と確認
		\item 室長不在の時には室長の仕事を行う(ミーティングの司会,夜間申請の提出など)
	\end{itemize}
です.
室長の仕事は多いので時と場合に応じてサポートや代理を務められるようにしておいて下さい.
そのためには室長の業務を把握している必要があります.
副室長に任命された方は係引き継ぎの室長の欄を確認しておいて下さい.
% また2018年度から室長と副室長は世田谷と総研で別れました.
% 副室長はどちらかの研究室において一部業務が室長と同じ扱いになるので注意しておいてください.\\
% (例)2018年度では室長は総研,副室長は世田谷にいたので室長の仕事で野中先生と日程調整が必要なものは副室長が野中先生と日程調整などを行っていました.また夜間申請に関しては室長が世田谷にいないので副室長が毎回申請を行っていました.
また,2019年度は室長が総研,副室長が世田谷の夜間申請の書類を作成していました.
来年度は副室長が修士にいるので去年の副室長から夜間申請の書類を引き継いでください.
 \item[~カレンダーの管理~]\mbox{}\\
	研究室で共有しているGoogleカレンダーの管理です.
学会や休暇表は各自で記入しているため,書いていない行事が無いかの確認くらいで十分だと思います.
 \item[~院・学部ゼミのスケジュール管理~]\mbox{}\\
院・学部ゼミの内容自体については研究室オリエンテーションで説明されるため,割愛させていただきます.\\
ゼミの資料とかに関しては副室長が用意,または購入してもらうものを周知するために年度初めに関口先生に何が必要か聞いておきましょう.
2019度の院ゼミ・学部ゼミは副室長がスケジュールの管理を行っていました.
研究室内外での行事とかぶらないように調整が必要となるため副室長の業務の中でも最も大変な業務だと思います.
昨年度の副室長がゼミのスケジュールを決める上で気を付けた点をいくつか示しておきます.
	\begin{description}
	 \item[ゼミは1週間に1回]\mbox{}\\
  研究室内の人数で変わりますが基本的には1週間に1回2人ずつ発表するようにすることを意識してスケジュールを立てましょう.
  % この原因として,一つ目に夏休み中に連絡をとり,長期休み明けにゼミを開けなかったこと.2つ目に2018年度は年に一人一回発表だったため楽観視してしまい,前期中にゼミを入れられなかった点があげられます.

  夏休み中に連絡を取りづらい方もいるかもしれないですが,きちんと先生や発表者と連絡を取り休み明けにはすぐにゼミを始められるようにしましょう.また春のオリエンテーションでゼミの回数が説明されると思いますが1,2月は卒論,修論の関係上忙しくなるので,通常で11月中,遅くても12月初旬にゼミを終わらせる計画を立てましょう.また前期中には就活やpcなどで忙しい人がたくさんいますが後期のほうが学会とかの関係上ゼミを開くのが難しくなっていくのでなるべく前期中にゼミの日程は詰め込みましょう.

また,ゼミより発表会報告会が優先であるため発表会係と連絡を取り,早めに決めていく必要があります.
そのため,発表会係に早めに決めてもらうように促すようにすることも大事です.
目安としては,ゼミの発表2週間前には発表者に伝える必要があるので3週間前には日程は確定しているようにしましょう.
研究室はいろいろと行事があるため,「今月は忙しいから来月に回せばいいや」と考えていると「土曜日や夏休みを使って一日の間に6人発表」なんてこともあり得ます.
月にゼミを何度行うか決めたら絶対に守るように頑張りましょう.
	 \item[他行事との調整]\mbox{}\\
		研究室の行事は報告会やpcなど色々あります.研究の報告会や発表会を行う週にはゼミを入れないようにしましょう.発表会係や先輩と話し合いながら調整して下さい.
	\item[ゼミを行える曜日・時間を把握する]\mbox{}\\
		先生は授業や会議などで忙しいです.授業の曜日は決まっているため,自然とゼミを行える曜日は決まっていきます.基本的に行える曜日・時間を把握しておくと日程が組みやすいです.また2019年度は野中先生が学科主任と総研の所長をやっていたためゼミの日程がとりにくい状況でした.なので毎週何曜日の何時限目に行いたいですというのを4月中に野中先生と関口先生に許可を取って決めておいたほうが良いかもしれません.\\
    ちなみに,昨年度は前期後期共に木曜日に行っていました.
  \item[担当されていたゼミ当日に風邪や体調不良で欠席した人が出た場合]\mbox{}\\
    担当していたゼミ当日に風邪や体調不良などの原因によって欠席される人もいます.その場合は代わりにどの日でやるのかということを話し合って,早いうちにスケジュールに反映させましょう.2019年度はこの部分を徹底しなかったせいで1月までゼミが伸びたので,特に気を付けて下さい.
	\end{description}
研究室内の人とマメにコミュニケーションをとって,「いつ」「誰が」「何を」行うのか把握しておくとスケジュールが組みやすいです.
特に多くの先輩が学会に参加しているので,日程が決まったら早めに連絡し先輩同士で話し合ってもらうこともあります.
その際に先輩から助言を頂くのも大いにありだと思います.
\end{description}
長くなりましたが,副室長の業務内容の説明は以上です.
副室長の役職は出来て数年のため,業務内容が他の係と分担・統合されるなど,大きく変更することが考えられます.
与えられた業務をよく確認して,室長のサポートをしつつ,時には室長にサポートしてもらいながら頑張って下さい.
\end{document}

\end{inputs}

\clearpage
\begin{inputs}
\input{works/記録係.tex}
\end{inputs}

\clearpage
\begin{inputs}
\input{works/動画係.tex}
\end{inputs}

\clearpage
\begin{inputs}
\usepackage{input}
\usepackage[dvipdfmx]{hyperref}
\usepackage{wrapfig}
\usepackage[dvipdfmx]{graphicx}
\usepackage[dvips]{graphicx}
\usepackage{listings}
\begin{document}
\maketitle
\section{PC係}
PC係の仕事は大きく分けてパソコンの保守管理と覚えておくべき知識から構成されます.以下にそれぞれ書いていきます.
\subsection{パソコンの保守管理}
パソコンの保守管理は2つに大別できて,1つは研究室に関係するパソコンの管理,もう1つは研究室のサーバ管理です.
\subsubsection{パソコンの管理}
セキュリティ対策のために,この研究室の研究生にWindows Updateはこまめに更新確認をし,
必要であれば適宜更新をするよう勧告してください.
研究室のPCも更新確認をし,必要であれば適宜更新するようにしてください.
ただし,Windows Updateを更新することによりパソコンに不具合が生じることが稀にありますので,
ネットでWindows Update関係のニュースを確認するようにし,問題がなければ更新するようにしましょう.
また,各種ソフトの緊急アップデートなどもあるので,情報の収集に務めてください.
Windows Updateを先延ばしにできても適用しないことは出来ないようになっているので,
しっかり情報収集を行ってください.

研究室で所持している各PCのスペックの管理と誰が使っているか,使っていないPCはどこにあるのを把握することもPC係の仕事となります.
また,PCの他にディスプレイ,LANケーブルに関しても管理しているので注意してください.
基本的にはExcelに一覧として纏まっています.また,PC,ディスプレイを新規購入したり,既存のものと交換した際には新しく管理番号の追加,編集を適宜行うようにしてください.
このExcelファイルは自分のローカルではなくOnedriveのworkspaceに保存しておき,常に最新版が他の人も必要に応じて閲覧できるようにしてください.
研究室で貸し出しているディスプレイに関してもPC同様に管理を怠らないようにしてください.
また,リモート環境になったことで,研究室所有のPCのIPアドレスの固定を進めています.
こちらもExcelで管理しているので適宜更新してください.

研究室で所持しているPCと個人で使用しているPCどちらにおいても,ウイルス対策ソフトの導入を必ず行わせるようにしてください.
個人で使用しているPCに関しては所持している人各々でソフトの導入を行うようにしてください.
ESETが廃止されたため現在はWindowsDefenderのみとなっています.

年に1回程度総研,世田谷それぞれですべてのPCの内部を清掃しています.清掃には総研では本館1Fに置いてあるエアコンプレッサーを使って埃を吹き飛ばしていました.研究室のメンバーに協力してもらい,PC係は清掃の指揮をお願いします.特にリモートデスクトップで使用するため24時間稼働しているPCも多く,埃が溜まりやすくなっているので注意してください.

\subsubsection{サーバ管理}
2022年3月現在,研究室には2台のサーバがあります.
総研学生室に設置してある計算機サーバと総研本館1Fに設置してあるworkspaceサーバです.
workspaceサーバに関してはPC係になったらsudo権限をもらってください.
sudo権限を得ると,workspaceへの新規アカウントの追加,subversionの管理(commit権限の付与,Paper20**の作成)などができるようになります.
commit権限の付与は/lab/PC設定/src/any/subversion-add-user.txtを参照してください.
また,Paper20**の作成は/lab/PC設定/src/any/subversion-commands.txtを参照してください.
また,適宜ログインしてセキュリティアップデートがきていないか確認し,あれば適用するようにしてください.
2台のサーバのOSはUbuntuを使用しています.Ubuntuの場合のソフトウェアアップデートのコマンドは以下になります.Tera Term等でログインし,実行してください.
\begin{lstlisting}
	$ sudo apt update && sudo apt upgrade -y
\end{lstlisting}
workspaceサーバは研究室のHPも兼任しています.
研究室のHPについては,管理は関口先生が行っています.
しかし,研究室のHPにあるPublicationタブ内の各論文のDOIなどの各種情報はPC係がエクセルファイルを追記して対応します.
エクセルファイルは,OneDriveの「LAB/Publication」内の3つのファイルです.
LABフォルダへのアクセス権は関口先生に相談してください.
エクセルファイルを追記したら,その変更を関口先生が取り込むので,適宜先生に報告してください.

計算機サーバについてですが,管理を学生に渡されたのでこちらについてもsudo権限をもらい管理してください.
特にアカウントの作成等をやることはないと思いますが,workspaceサーバと同様にセキュリティアップデートの確認と適用をしてください.
ROS(Robot Operating System)が入っているので,ROSの知識についてもある程度持っておくとよいと思います.
アップデートの際には,ROSを使用している学生と相談をして行ってください.
また,計算機サーバの使い方に関しても覚えておいてください.
tera termとXmingというソフトを使って現在は使用方法を確立しています.
詳しい使い方は/lab/PC設定/計算機サーバの使い方.pdfに纏めてあるのでそちらを参照してください.

\subsection{覚えておく知識}
\subsubsection{プリンタの管理}
プリンターに関しては,以下のURLから状態を確認することができます.\\\\
http://(プリンターのIPアドレス)\\\\
トナーの残量レベルが低くなったら交換時期です.
美化備品係と相談をして購入するようにしてください.
交換する際は,色のトナーを野中先生に申告し,注文してもらってください.
数日で届きますがすぐには交換せず,しばらくは強制印刷で凌いでください.
強制印刷ができなくなったら交換をしましょう.

印刷用紙が少なくなってきたら売店で購入してください.お金は必要ありません.
A4の印刷用紙の箱をレジまで持って行き,研究費で購入したいという旨を述べ,
学科名と研究室名を言えば領収書を発行してくれます.
その領収書に自分の名前と学籍番号を明記すれば会計終了です.
領収書を受け取ったら速やかに関口先生に提出してください.
これらの作業は美化備品係と協力してください.
また,印刷ができない状態が長期間続かないようにすること,
トナーや用紙を購入しすぎてしまうことがないように心掛けてください.

\subsubsection{Subversionの使い方}
新4年生の最初の仕事はSubversionと計算機サーバの講習会です.
詳しい方法については/lab/PC設定/Subversionを参照してください.

\subsubsection{研究室における必要最低限のソフトウェア}
新しいパソコンを研究室で運用する場合,必要最低限として以下のソフトウェアをインストールしてください.
\begin{itemize}
\item TexWorks
\item Visual Studio Code
\item Tera Term
\item subversion
\item MATLAB
\end{itemize}
それぞれのインストール方法,注意点などは別途テキストに纏めてあるのでそちらを参照してください.
さらに,LaTeXのインストールでは必ずインストールしたPCがタイプセットを行う際に描画に関する動作が適切であることを確認することをお勧めします.
インストールの方法は方法はPC設定に資料がすべてまとめてあるのでそちらを参照するようにしてください.
PC係はそれらの資料を適宜更新していくようにしてください.
MATLABは高機能機械制御研究室において,必須といえるソフトウェアであるため,インストールは必ずしましょう.その際にインストールするバージョンは先生や先輩方に相談しましょう.
一昨年まではVisual Studio 2017をインストールしてもらっていましたが,MATLABを使用する学生が大半となったため,Visual Studioに関しては必要な際にインストールする形にしてください.
(lab/PC設定/src/事例研究PC係に確認用ファイル有)

\subsubsection{新しい研究室メンバーが配属された際の注意点}
仮配属等で新しく学生が研究室に入ってくる際に,PC係はアカウント作成とソフトウェアのインストールを担当します.
初めに,配属会後に研究室からの連絡を受け取るGmailアカウントとMicrosoft Educationアカウントを登録させます.
続いて,事例研究の初回授業までにソフトウェアのインストールをしてきてもらいます.

次に,事例研究の初回授業ではworkspaceサーバへのアカウント作成とソフトウェアの確認を行います.
その際,新しく入ってくる学生にパスワードを考えてもらいます.
そのとき考えてもらうパスワードは,セキュリティの観点から他のところで使用していないもの,
また個人を特定できないようなものにすることを必ずアナウンスした上で考えさせてください.
また,そのパスワードをメモするようなことはなるべく避けさせてください.
個人的にメモをしておきたい学生にはメモしたものの管理を徹底させてください.
(/lab/PC設定/src/事例研究PC係/新メンバーworkspace登録手順.txtを参照)
また,ソフトウェアの確認は4年生主体でM1とM2のPC係がサポートとして参加します.例年TeXで問題が発生するため,4年生全体に予め手順を確認するよう説明しておいてください.

\subsubsection{OSのアップグレードの注意点}
OSのアップグレード(例:Windows7からWindows10にする)を行う場合,
研究室における必要最低限のソフトウェアの動作が問題なく行えるかを確認した上で行ってください.
個人のPCに関しては自己責任で行うようアナウンスを行ってください.
特に研究室で管理しているパソコン(Excelファイルに纏めているもの)に関してはしっかり確認を行い,PC係主導で行ってください.
また最近はアップグレードを行う場合,必要なデータをネットワーク経由で落とすことが多いのでアップグレードを実行する際は,
スケジュールをしっかり組んで研究室のネットワークに負担がかからないように行うといいです.

\subsubsection{停電時の対応}
総合研究所,世田谷キャンパスが停電となる際には研究室にあるコンセントはスケジュール全て外しておく必要があります.


\subsection{その他連絡事項}
2019年度から研究室が完全に総合研究所に移動しました.
それに伴い,世田谷と総合研究所間での通信はVPNが必要となります.
さらに総合研究所で利用する研究室PCは年度初めにクリーンインストールを予定しています.
そのため,クリーンインストール後に各種ドライバのインストールも必要になります.
基本的にWindowsUpdateを実行すればインストールできると思いますが,ドライバはPCによって異なるのでインストールできなかったときのことを考えて行動しましょう.
また,研究室のネットワーク,配線などについてもPC係は可能な限り把握しておくと,何かあったときの対応がすぐにできるかと思います.頑張ってください.困ったときはM1とM2のPC係に早めに相談しましょう.
% また,2018年度から高機能機械制御研究室のYouTubeのチャンネルができました.
% YouTubeで「tcu acsl」と調べればチャンネルが検索に引っかかります.
% PC係はこのチャンネルの管理者権限を関口先生から付与されるため,チャンネルへの動画のアップロードを適宜行ってください.
% 基本的に,学生からアップロードしたい旨を受けたら,先生に相談しているか尋ね,相談していないようなら相談するように促してください.
% 先生からアップロードの許可を得ているものに関しては,アップロードしてください.
% 研究室のチャンネルとなるため,載せる内容は注意して選定してください.

%Thank you


\subsection{Prime関連}
\subsubsection{2017年度まで}
Primeの保守点検をおこなう係です.
役職に就く人間はPrimeを使用する人の方がいいと思います.
Motiveのバージョンが新しくリリースされた場合アップデートを行ってください.
また,アップデートをした時にパケットクライアントが新しくなると思うので,それを書き換えるようにお願いします.
新しいパケットクライアントはOptitrackのHPにあります.
また係の仕事として10号館5Fのフィールドの使用者を管理する仕事があります.
具体的にはGoogleカレンダーで使用者が書き込むシステムです.
17年度に作製したアカウントがあるのでそれを載せておきます.
ユーザーIDはtcuprime@gmail.comです.
パスワードは「Only北海道」とググれば出てくると思います.

\subsubsection{2018年度}
今年から総合研究所に移動となり,世田谷キャンパスと等々力キャンパスどちらにもPrimeがあります.
総研にあるPrimeの台数が30台なので
役職に就くのは総研でPrimeを使用する人のほうがいいと思います.
この役職では,昨年に引き続きPrimeの保守点検を行います.
また,Primeからのデータを取得するためのプログラム(通称Packet Client)の作成から保守までを行います.
今年からMATLABが個人のPCにインストールできるようになったため,2017年度まで使っていた
Cプログラムで記述されたPacket Clientではなく,MATLAB用のPacket Clientを作成しました.
雛形を作成してあるので基本的にはそれを更新していくようになると思います.
元のプログラムはOptitrackのホームページのNatNetSDKをダウンロードすればわかると思います.
アップデートが入るたびにデータの形式が変わっていないか確認したほうがいいと思います.
わからないことがある場合はOptitrackのwikiを検索すると載っていると思います.

\end{document}

\end{inputs}

\clearpage
\begin{inputs}
\documentclass[12pt]{jsbook}

\usepackage{input}
\usepackage{msethesis}
\usepackage{amssymb,amsmath}
\usepackage{bm}
\usepackage{url}
\usepackage{here}
\usepackage{math}
\usepackage{subcaption}
\usepackage{comment}
\usepackage{float}
\usepackage{ulem}
\usepackage{color}

\begin{document}

\section{発表会係}

発表会係の大まかな仕事内容を列挙します.

\begin{itemize}
  \item 報告会または発表会の日程調整及び告知
  \item 報告会または発表会のタイムプログラムの作成
  \item 発表資料等の管理
  \item 報告会または発表会の準備・司会進行
  \item 報告会または発表会の画面録画
\end{itemize}

上記の通り,発表会係の仕事の多くは報告会または発表会(以下,発表会)に関するものとなっています.
全てを含めると年間で十数回開催されるため,年間を通して小まめに仕事を続ける係と言えます.

\subsection{報告会と発表会の区別について}
報告会は研究の進展を端的に報告する会であるため,
背景など面倒な前置きを省いて構わないことになっています.
この時,前置きを省く際には,発表の論理的な流れに注意して下さい.
対して,発表会は背景やこれまでの研究内容など,自身の研究内容を一通りさらいながら発表します.
このため,前置きで発表時間の半分以上を使うことがざらです.
また,時間管理が発表会毎に微妙に異なるため,注意して下さい.

\subsection{日程調整及び告知}
発表会の大まかな日程は,新年度の研究室オリエンテーションで配布される資料に記載されているので,そちらを参照下さい.
資料に記載された時期となった場合,slack(発表会日程調整チャンネル)で先生と相談し,詳細な日程を詰めて下さい.
基本的には,4週間前をめどに日程調整をお願いします.
初めは4月中に2ヶ月ぐらい先まで計画しておいた方が良いです.
特に4~6月は就活などで予定が入る人が多いため,早めに日程を決めることをお勧めします.
早めに決めると,教室予約や先生との日程調整が楽になります(特に12月は先生方が忙しくなり,日程調整することが難しいので早めが良いです).
日程調整の際,googleカレンダーに不在や就活などの予定書かれているので,人がいない日は避け,開催日の候補を何個か決めてから,先生に予定を伺うようにしてください.
予定を伺うときは〇日の△時限~△時限と日程だけでなく時間まで決めてください.
また,発表会の時間は夜遅くならないようにしましょう(遅くても19時までのほうが良いです).
発表会当日に全員が参加することが望ましいですが,先生との日程の兼ね合いより,全員の参加が難しいことがあります.
その場合は,発表者が発表時間に必ず予定が空いているように調整してください.
発表会は,野中先生と関口先生がいらっしゃることが前提です.
候補日が決まったら,教学課(月~金は17:00まで)に行き,教室を予約しましょう(複数の候補日で先生とのご都合がついた場合は,良い教室が取れる日を選ぶといいです)
教室はなるべく各机に電源がある教室が好ましいです(絶対ではありません).
各机に電源がない教室の場合は,教学課から延長コードを借りましょう.
2025年度の各教室の設備については,NASのws2025/Lab2025/発表会に入れておきますので参考にしてください.
また,ネット接続が安定してできる教室を選ぶようにしてください. \\
詳細が決まったら,全体ミーティングで発表会について全員に告知しましょう.
基本的には,開催日・開始時間と終了時間・発表資料の提出締め切りについて告知すれば良いです.
開催日が複数ある場合は,その日に発表を行う人の学年も告知しておくといいと思います(基本的には,学部生が1日目となります).
また,googleカレンダーにも発表会の予定を入れておきましょう.

\subsection{タイムプログラムの作成}
タイムプログラムの作成はだいたい2週~3週前に行うと良いです.
作成したタイムプログラムは,先生方に確認してもらい,必要に応じて修正を行って下さい.
問題なければ,slack(発表会チャンネル)で共有してください.

\subsubsection{デザイン}
タイムプログラムについては,デザインなどの指定はありません.
2025年度のタイムプログラムをNASのws2025/Lab2025/発表会/タイムプログラムに入れておくので,参考に自作して下さい.
作成する際に,Word・Excel・LaTeXなどのいくつか選択肢がありますが,こちらについても指定はありません.
2025年度のタイムプログラムはExcelで作成したものをWordに貼り付けてpdf化しました.

\subsubsection{研究テーマの記入}
タイムプログラムに作成する際に,発表者の研究テーマを記入します.
このタイトルは2週~3週前までに把握して発表者に確認をしておいてください.
2025年度のタイムプログラムには,一部の発表者の研究テーマが記入されていませんが,基本的には記入しましょう.

\subsubsection{発表順}
タイムプログラムを作成する際に,発表順などは発表会係が決めます.
基本的には,学部生>修士生>博士生の順に行うようにしましょう.
学年内での順番はランダムに決めるといいと思います.
2025年度ではルーレットで順番を決めました.
なお,卒業論文公聴会などの場合は学籍番号順にしましょう.

\subsubsection{時間設定}
発表会の時間設定は,タイムプログラムを作成する際に最も慎重に決定しなければいけない事項です.
発表会での発表者の持ち時間は,発表時間と質疑応答時間の2要素で構成されます.
時間構成は,発表会,報告会ごとに必要とされる時間が異なります.
余裕のある時間設定が望まれるため,少し大目の時間設定を心がけて下さい.
また,休憩時間は1時間から1時間30分毎に10分間取るようにしてください(学年が変わる際には休憩を挟むことが推奨されています).
タイムプログラム通りにいかなかった場合は,発表会係から何時まで休憩かを告知するようにしてください.
昼休憩については,1時間を目処に設定して下さい.

\subsection{準備・司会進行}
発表会当日の大まかな流れと仕事内容を以下に示します.

\begin{enumerate}
  \item 発表会準備
  \item 発表会開始の宣言
  \item 先生方へ開会のあいさつを求める
  \item 発表会の構成についての簡単な説明
\\
\\
-----以下,発表が一通り終わるまでループ-----
  \item 発表者のタイトルおよび氏名の読み上げ
  \item 発表中に,発表時間を知らせるベルを鳴らす
  \item 発表が終わり次第,発表時間を告げる
  \item 質問を受け付け,挙手した人を指名
  \item 学生の質疑を終了し,発表者へ一言挨拶して,拍手 \\
-----以上,発表が一通り終わるまでループ-----
\\
\\
  \item 休憩を挟む際には,休憩時間と再開時間の告知を行う
  \item 先生方へ発表会全体に対する講評を求める
  \item 発表会終了の挨拶
  \item 発表会の片付け
\end{enumerate}

報告会または発表会に応じて必要な作業が増減しますので,臨機応変に対応して下さい.
挨拶の内容などにテンプレートは無いので,事前に自分で考えて準備して下さい.
発表時間には交代等の時間も含まれているのでそれを踏まえて司会進行を行って下さい. \\

\subsubsection{発表資料の提出}
発表会までに,発表者から発表資料を提出してもらう必要があります.
提出方法については,提出場所を事前に作成しておいてください(2025年度はTeamsのLab2025/発表会に書く発表会ごとのフォルダを作成して入れてもらいました).
提出期限は発表会前日(遅くても発表会当日の開始時間の1時間前)までにしておいたほうが良いと思います.
提出された発表資料は,発表会当日に発表用PCにダウンロードしておいてください.

\subsubsection{発表会の準備}
発表会当日,発表会係は1時間前をめどに予約していた教室に行き,発表会の準備をしましょう.
教室のカギは教学課で借りることができます.
発表会の準備としては,プロジェクターとマイクの準備・チャイムの消音・発表資料のダウンロード・ZOOMを用いた画面録画(詳細は「ZOOMを用いた画面録画について」を参照してください)があります.
発表会開始予定時間までには準備を終わらせるようにしましょう. \\

\subsubsection{司会進行}
司会進行の流れと内容について説明します. \\
まずは発表会開始の宣言です.
発表会開始の宣言としては,発表会開始時間になったら,「時間になりましたので,〇月発表会を始めます.」と宣言しましょう.
続いて,「開会のあいさつを〇〇先生,お願いします.」のように,先生方へ開会のあいさつを求めます(職歴順で1人:関口先生).
挨拶をいただいたら,発表会の構成(発表時間・質疑応答時間)について,簡単に説明しましょう.
以上が発表会開始の流れになります. \\
次に発表に移ります.
発表者に前に出てもらい,発表の準備が完了したら,「(題名)と題して,〇〇さん,発表をお願いします.」のように,発表者のタイトルおよび氏名の読み上げを行いましょう.
発表が終了したら,「〇〇さん発表ありがとうございました.ただ今の発表時間は@分@秒でした.それでは,質問のある方は挙手をお願いします.」のように,質疑応答の時間に移る.
質疑応答の時間が終了したら,「時間ですので,質疑応答を終了します.〇〇さん,発表ありがとうございました.」のように,発表者へ一言挨拶して,拍手しましょう.
これを発表者の数だけ繰り返します. \\
最後に,全ての発表が終了したら,「すべての発表が終了しました.それでは,先生方から講評をいただきます.〇〇先生,お願いします.」のように,先生方に発表会全体に対する講評を求めましょう(職歴順で全員:関口先生→野中先生).
講評をいただいたら,「〇〇先生,ありがとうございました.以上をもちまして,〇月発表会を終了します.ありがとうございました.」のように,発表会終了の挨拶をしましょう. \\
以上が,発表会の司会進行の流れと内容になります.

\subsection{発表会の片付け}
発表会が終了したら,速やかに片付けを行いましょう.
片付けの内容としては,発表会で使用した道具の片付け・忘れ物の確認・教室のカギの返却があります.
鍵の返却については,発表会終了時間が17時を過ぎた場合は,教学課ではなく警備室(1号館入ってすぐ左)に教室のカギを返却してください.

\subsection{発表時間}
2025年度の各発表時間の表を示します.
発表時間と質疑応答時間については,発表会の種類や発表者の学年によって異なります.
また,中間発表会については,ポスターセッション形式で発表を行いました.
なお,忘れがちですが最終発表会では発表者は全員スーツ着用なので注意してください.
さらに,修士生以上で学会発表などが近い場合は,学会発表練習として,発表時間と質疑応答時間が他の人とは異なる場合があります.
希望者がいる場合は,相談して時間を決めてください.

\begin{table}[H]
  \centering
  \caption{発表時間一覧}
  \label{table:Method comparison}
  \begin{tabular}{|l|l|c|c|} \hline
    & 学年 & 発表時間 & 質疑応答時間 \\ \hline
    卒論構想発表会 & B4 & 3分 & 2分 \\ \hline
    5月報告会 & B4 & 5分 & 2分 \\ \hline
    5月報告会 & M1~ & 10分 & 5分 \\ \hline
    6月報告会 & B4 & 5分 & 2分 \\ \hline
    6月報告会 & M1~ & 10分 & 5分 \\ \hline
    8月中間発表会 & B4 & 30秒 & 9分30秒 \\ \hline
    8月中間発表会 & M1~ & 8分 & 2分 \\ \hline
    10月報告会 & B4 & 7分 & 3分 \\ \hline
    10月報告会 & M1~ & 10分 & 4分 \\ \hline
    11月中間発表会 & B4 & 30秒 & 4分30秒 \\ \hline
    11月中間発表会 & M1~ & 30秒 & 4分30秒 \\ \hline
    12月報告会 & B4, M1 & 30秒 & 9分30秒 \\ \hline
    12月報告会 & M2~ & 10分 & 4分 \\ \hline
    事例研究中間発表会 & B3 & 3分 & 2分 \\ \hline
    事例研究最終発表会 & B3 & 5分 & 3分 \\ \hline
    卒業論文公聴会 & B4 & 10分 & 4分 \\ \hline
  \end{tabular}
\end{table}

\subsection{合同発表会}
2025年度では,8月中間発表会と11月中間発表会において,ロボティクス・メカトロニクス研究室と合同で発表会を行いました.
発表会が行われる1ヶ月程前にから,ロボティクス・メカトロニクス研究室の代表者とslackで連絡を取り合いながら,発表会の日程調整を進めるようお願いします.
また,提出資料と研究テーマについては,slack(発表会チャンネル)で共有してもらい,タイムプログラムを作成してください.
なお,合同発表会の際には,開会のあいさつは職歴順で1人:藪井先生,講評は職歴順で全員:藪井先生→関口先生→佐藤先生→野中先生となります.


\subsection{道具の管理}
発表用PC・ベル・ポインターの管理が任されます.
引継ぎの際,新しく発表会係となった人が管理をして下さい.
発表用PCについては,充電しながら発表会を行いましょう.
ベルについては,現在使用していません.
ポインターについては,あまり充電しなくても使えますが,充電しておいたほうが安心です.

\subsection{ZOOMを用いた画面録画}
発表会では,ZOOMを用いた画面録画を行います.
ZOOMのURLは先生(基本的には関口先生)に作成してもらうようにしてください.
作成してもらったZOOMのURLは,googleカレンダーに記載されます.
発表用PCでZOOMに入室したら,ホスト権限をもらい,画面共有(画面を拡張して画面2を共有しましょう)と録画(クラウド)を開始してください.
また,発表用PCのミュートを解除しておきましょう.
休憩に入ったら,録画を一時停止してください.
発表会が終了したら,録画を停止してZOOMから退出してください.
発表用PCの他に,確認用として個人のPCでZOOMに入室しておくと安心です.

\subsection{Time Keeperを用いた時間管理}
発表会の時間管理には,Time Keeperというサイトを用いています.
このサイトでは,開始時間・ベルが1回鳴る時間・ベルが2回鳴る時間・ベルが3回鳴る時間を設定できます.
開始時間は0:00に設定し,ベルが1回鳴る時間は発表時間の半分の時間に設定してください.
ベルが2回鳴る時間は発表時間の終了時間に設定してください.
ベルが3回鳴る時間は質疑応答時間の終了時間に設定してください. \\
発表会当日,発表会係は個人用PC(通知は切っておきましょう)でこのサイトを開き,時間設定をしてください.
発表者が発表を開始したら,Time Keeperの開始ボタンを押してください.

\subsection{最後に}
最後に,全体を通した注意点として,発表会について先生と相談する際には,先生に具体的な意見を仰ぐのは極力控えて下さい.
日程調整の相談などを行う際にも,先生に日時を決めていただくのではなく,自身で具体的な日時を提示するようにしてください.
また,発表会を行う教室が縦長の場合は,聴講者たちを前の席に着席させるようにしてください.
発表会で用いた資料などはNASのws2025/Lab2025/発表会に入っているので参考にしてください.
引継ぎ内容は以上となります.

\end{document}

\end{inputs}

\clearpage
\begin{inputs}
\usepackage{input}
  \newcommand{\test}{waya}
\begin{document}
\maketitle
\section{メッセージ係}
\subsection{MESSAGEについて}
\begin{center}
MESSAGE\\
\textcolor{red}{ME}chanical \textcolor{red}{S}ystems \textcolor{red}{S}ymposium by all \textcolor{red}{AGE}s\\
\end{center}

MESSAGEとは,機械システム工学科が主催するシンポジウムの事である.
このシンポジウムは卒業生と在学生の交流を目的に開催されており,機械システム工学科を卒業し,様々な職場で活躍されているOB,OGの方がご来校下さります.
在学生が卒業生の活躍を知ることで,目的意識をもって学習に取り組むことができるようになります.
卒業生による講演だけではなく,開催年度に就職活動を体験し,社会人としての第一歩を踏み出そうとしている在学生による講演も行われます.
本研究室においても,多くの卒業生,先輩方が講演をなさってこられました.

このMESSAGEは各研究室のM1,M0の学生計12名と幹事研究室の教授がMESSAGE実行委員となり行われます.
``MESSAGE係''とは,高機能機械制御研究室を代表しMESSAGE及びCAP\,(Campus Amusement Park)の運営をする係です.
※ただし,研究室展示のことをCAPと呼ぶと一部の先生にとって違和感になるようなので注意.

\subsection{主な仕事内容}
MESSAGE係の研究室での主な仕事内容は,研究室展示の内容の検討,卒業生への周知などがあります.
また講演していただく卒業生は,野中先生と相談して,お願いをします.

MESSAGE実行委員会ではそれぞれの研究室に役職が割り振られ,担当する役職の役割を果たします.
実行委員会は夏季休業前に顔合わせを行い,話し合いを進めながら各研究室で本番へ向け準備を進めていきます.
細かい仕事内容については年度ごとに変化します.
基本的に実行委員長の先生が招集をかけ,話し合いを行いますが,先生によってはこの企画自体に積極的ではないので,自分たちで動かなくてはいけない.

\subsubsection{プログラム係}

プログラム係は,OBの方へ告知をする際に添付する大まかなプログラムと,当日どの研究室がいつどこでどんな仕事をするといった詳細シフトを作製します.詳細シフトは直前になっても大丈夫ですが,OBの方へ送るプログラムは早めにできていないとまずいです.プログラム待ちで連絡できませんといった状況は避けましょう.

\subsection{注意点}
MESSAGE開催までに注意することとして,MESSAGEに来ていただける卒業生は今まさに社会で活躍されている方々です.そのためMESSAGEに関する情報の周知が遅い場合,大変迷惑がかかります.なので,卒業生の方々への周知は早めにする必要があります.また,MESSAGEの具体的な内容の決定が遅くなると各方面に大変迷惑がかかります.なので,早めにMESSAGEで行う内容を決定する必要があります.理想的なスケジュールとしては,7月中に企画内容やテーマを決めてお盆までには出演依頼のメールをOBの方に送れているといいです.あと,その際の注意点では,連絡先を知らなくても野中先生が知っていることが多いので野中先生とよく話をしましょう.

また,MESSAGE当日は世田谷キャンパス学園祭の一日目に開催されるため,研究室の出店などへの参加が難しくなります.
そのため会場の受付など,研究室のメンバーに仕事をお願いする必要があるため,学園祭係との打ち合わせを行い,シフトを作成する必要があります.

\subsection{過去企画}

\subsubsection{基調講演}

ほとんど毎年実施.OBの方に講演をしていただく企画です.話してほしいテーマと時間をお伝えしてスライドを作ってきていただき,講演していただきます.2016年度は社会に出てから10年以上たつベテランの人にお話をしていただくというコンセプトでやりました.中堅1人,若手1人の構成で,僕たちはもちろんOBの方にも世代の異なる人の意見に触れるいい機会にしていただこうという感じでした.

\subsubsection{パネルディスカッション}

2016年度実施.各研究室から1人ずつパネラーのOBの方を出して,壇上でディスカッションしていただく企画.自己紹介用として2ページ程度のスライドテンプレを渡してあらかじめ作ってきていただいて,それをもとに企画が始まる.ちなみに2016年度は司会は幹事研究室の先生.その後はこちらからパネラーの方々に聞きたい質問をアンケートをもとに用意しておいて,それに対して喋っていただく.注意点としては,司会が聴衆からの質問を頻繁に受け付けないといまいち盛り上がらない.あと,ぶっちゃける先輩がいるとすごい盛り上がる.

\subsubsection{ブース}

2015年度に実施.各OBさんが自分の企業のブースを作ってそこにみんながお話をしに行くというもの.講演やディスカッションと違い近い距離で話ができる,自由な質問ができるというのが売り.ただし,企業の人気不人気はどうしても存在するのでブースごとに人の粗密ができてしまう.あとは1教室内で複数ブースがあると音が混じってうるさいらしい.なので,この2点に関しては要工夫.

\subsubsection{懇親会}

毎年実施.メモリアルホールでケータリングを用意して懇親会をする.ただし,学祭では教室内飲食禁止なのでメモリアルホールが取れなかったときは悲惨なことになる.必ず早めに抑えておくこと.予め担当研究室の指揮の下,会場レイアウトとケータリング搬入を済ませておくこと.基本的にMESSAGEのメインイベント.だが,各研究室でも夜に各々で飲み会を企画しているので,夕方ごろ終了になるのが望ましい.結局,それぞれの研究室の同窓会みたいになる上,かなり前のOBは先生くらいしか話せる人がいないのでそのあたりのケアが課題としてある.

\subsection{$2020$年度の活動}
コロナウイルスの影響により世田谷祭が中止となり,MESSAGEも中止となった.そのため,$2020$年度は準備等含め活動していない.MESSAGEについては関口先生から連絡があると前任者から伺いました.

\subsection{$2021$年度の活動}
本年はコロナの感染拡大状況を鑑み,初のオンライン開催となった.以下に当日までの流れを順に示す.

\subsubsection{顔合わせ~事前準備}
OB・OG招待まで \\
本年は熱流体システム研究室の永野先生が主催を務めていた.主催研究室の先生との連絡は基本メール,または先生により開催されたzoomでのミーティング時に行った.コロナ禍であったこともあり顔合わせが例年に比べて大分遅く,初回zoomミーティングは8月の終盤であった.

本年は例年と比べて研究室間の連携が幾分取りづらいものとして,OB・OGに送付する招待状やパネルディスカッション等への参加依頼文はすべて永野先生が用意してくださり,各研究室ごとに連絡代表者や研究室名を変更するという形であった.そのため,学生側の主な仕事はフォーマットが届くまでにパネルディスカッションを依頼するOB・OG(1名)と,グループディスカッションを依頼するOB・OG(2~3名)について,前年度の先輩を交えて候補者を決め,野中先生と関口先生に確認を取ることであった.

文面のフォーマットが届いたのは9月の初週終わりくらいであり,各研究室で文面を整えPDF化してメールに添付する形での送付となった.尚,この際の招待はOB・OGに対して一括で行うため,事前に野中先生乃至は関口先生に送付を依頼しておく必要がある.(招待を送付するのは先生でしたが,以降のプログラム送付やOB・OGの方とのやり取りは学生側で行います.)

最終的な招待は9月中旬ごろに行い,返信の〆切としては10月初週を提示していた.(例年はそもそも夏休み前から準備が始まっているものらしいので,この期間に関しては本年のは参考にしすぎない方がいいかもです.きちんと当代の主催の先生に確認を取ってください.)

その後,講演について了承してくださったOB・OGの方には改めてお礼を述べた上,当日のプログラム送付と合わせてグループディスカッション用のスライド作成を依頼した.(枚数の指定は必要ありませんが,予めどういう話について纏めてほしいかは提示しましょう.ちなみに本年は企業説明や紹介等ではなくOB・OGの皆様の経験談を軸に企業に勤めてからのお話をして頂きました.)

※招待状文面に関しては多分に主催担当の先生の裁量に左右されると思われるため,次年度から例年通り文面も学生側で考案する必要があることも念頭に置いて準備を進めた方がいい. \\

聴講側のOB・OGの招待 \\
10月下旬半ば頃に.永野先生から講演する側ではなく聴講者として参加するOB・OGを招待してほしいと連絡が来た.招待の送付を研究室の教授・准教授がするか,または学生がするかは各研究室内で決めるようにとのことで,本年は関口先生に卒業生の連絡先一覧を頂いて学生側で送付する形を取った.この際も招待文のフォーマットは永野先生が用意してくれていたので,学生側は貰った連絡先をBCCに追加し続けるのと文面に少し手を加えるくらいしか仕事は無かった.(Toに自分の学番メールを設定すると,不具合なく届いたか確認しやすくて便利です.) \\

アンケートの送付 \\
10月最終日,開催一週間前にして永野先生からOB・OGへのアンケートが届く.この際もやはり依頼文は予め用意してくれたので,研究室用にちょこちょこ直して送付という形になった.(事前にこういう追加をしてくる先生がみんな依頼文を用意してくれるとは限らないので,不安な場合は自分から率先して主催の先生に聞いておいた方が良いです.面倒でも連絡はこまめに取りましょう.) \\

オンライン開催に向けた準備 \\
本年は高機能研関口班の朝ミーティングでも使っているSpatialChatにて開催するとのことで,各研究室は上記のアンケート送付の時期と同じくしてチャットルームの背景に使用する研究室写真の提供を求められた.本年は学生室・実験室の写真を撮って永野先生に送付したが,過去の世田谷祭における出店の写真とかでもいいらしい.(ただし,あんまりはっきりと関係ない人の顔が写ってしまっているものは避けるべき.) \\

質疑考案 \\
パネルディスカッションに関しては,当日ももちろん参加する在学生に質問してもらうことになるが,質問が途切れないとは限らないので予めいくつかは担当の学生間で質問を用意しておく必要がある.この際,他の研究室の学生と質問が被るのも避けなければいけないため,事前に割り振られたペアとなる研究室の担当者と質問内容を共有しておいた方が良い.(今年はロボ研とペアでした.)

\subsubsection{当日の流れ}
当日は例年通り世田谷祭の一日目に開催した.開始直前にホストの渡し方が行き届いていなかったり,一度に部屋に入れる人数に制限があったりと若干の手間取りがあったものの,概ねマイク等の機器故障もなく開始.永野先生の司会進行に沿ってパネルディスカッションを終えた.質問は,用意していたものから聞いたものも幾つかはあったが,関口先生曰くオンラインでカメラも無いせいか例年よりみんな積極的だったと好評だった.

その後,午後の部までの昼休みの間にMESSAGE担当学生と永野先生とで簡単なミーティングをし,午後の部開始時の全体連絡と要所要所のタイムキーパーを永野先生に依頼した.

午後からは各研究室用に作られたzoomで言うところのブレイクアウトルームのような小部屋で待機し,永野先生の全体放送が終わり次第各研究室ごとのグループディスカッションとなった.この際の司会進行は学生が務めることになるため,段取りは事前によくよく見直した方が良い.

また,研究室内でのタイムキーパーはMESSAGE担当学生側で担うことになるため,スライドによる発表をお願いした後の質疑応答時には時間配分に気を配る必要がある.

\subsubsection{その他}
主催の先生が新しいツールを持ってきたときには,細部の仕組みまで掴み切れていない場合もある.こういうのを使うと話が出たら,一通り自分でも操作しておいた方が当日不備があった時に補助しやすいのでおすすめ.
\end{document}

\end{inputs}

\clearpage
\begin{inputs}
\input{works/イベント係.tex}
\end{inputs}

\clearpage
\begin{inputs}
\usepackage{input}
  \newcommand{\test}{waya}



\begin{document}
\maketitle
\section{研究成果管理係}
研究成果管理係の仕事は,大きく次の3つです.
\begin{enumerate}
  \item 誰かが学会に参加する度に論文やエビデンスの保管,投稿情報の記載などを行っているかの確認
  \item 昨年度行われた学会発表の内容を英語でまとめたスライドの作成
  \item 卒論$\cdot$修論の保管が行われているかの確認
\end{enumerate}
研究成果管理係は例年M0が担当するので,分からないことがあれば元係だった先輩に聞いてください.
なんなら学会に参加した方であれば大体の流れは把握していると思うので,学会に参加した先輩に聞いてもいいと思います.

\subsection{誰かが学会に参加する度にやること} \label{subsec:conference}
\textcolor{blue}{teams} > Lab20oo > Conference > 学会に関する流れ.xlsx に学会発表に際しての流れが記載されています.
研究成果管理係としては,学会に参加した人が「学会に関する流れ.xlsx」下部の事務作業の部分をきちんと行っているかの確認を行います.
確認事項は以下の通りです.
\begin{enumerate}
  \item 予稿集に掲載された自身の論文を \textcolor{blue}{teams} > Published > 20oo04-20oo03 へ格納しているか
  \item \textcolor{blue}{teams} > Published > domestic.xlsx, international.xlsx, journal.xlsx に投稿情報を記載しているか
  \item エビデンスとして,自身の論文のPDF,tex, 生データ, program,turnitin結果$^{\ast}$を \textcolor{red}{NAS} > ws20oo > Lab20oo > Papers > 学会名 > ローマ字個人名 に保存しているか
  \item 学会の予稿集を \textcolor{red}{NAS} > ws20oo > Lab20oo > Conference に格納しているか
\end{enumerate}
特に,3つ目の\textbf{エビデンスの保存は最重要}です.
年度末になると全体ミーティング等で先生が度々仰ると思いますが,世の中には自身の論文$\cdot$研究の不正を疑い,エビデンスを要求してくる人がいます.
こうした場合に自分自身を守るため,また論文の内容を再現するために必要なものなので,きちんと保存されているか確認してください.
1~4の作業を年度末にまとめてやると大変なので,誰かが学会に参加する度に確認することが大事です.
特に各学会の予稿集$^{\ast\ast}$は,保存期限が設けられているので早めにやらないとダウンロードできなくなるので注意が必要です.

{\footnotesize $\ast$ turnitinとは,論文の剽窃チェックをするためのサービスです.研究室では学会発表の際に提出する論文をturnitinにかけて,その結果をエビデンスとして保存することになっています.}

{\footnotesize $\ast\ast$ 予稿集とは,学会発表の内容が掲載された冊子(データ)のことです.その学会の全ての論文が記載されており,容量が大きいのでNASに保存します.}

\subsection{昨年度対外発表のまとめスライド作成}
昨年度行われた学会発表の内容を\textbf{英語で}まとめたスライドを作成します.
研究室には海外からのお客さんがちょくちょく来られるので,その方々に研究室の研究内容を知ってもらうために,毎年作成しています.
(ただ2025年度時点では,ここ数年スライドを作成したものの実際に見せる場面は無いようです...)
後期が始まるまでには完成させてください.
1人で作成するのは大変なのでB4生に割り振るのをおススメします.
2025年度はM0に割り振って作成してもらいました.
\textcolor{blue}{teams} > 研究室共通 > 研究成果管理係 > 英語スライドテンプレート.pptx にスライドのテンプレートが保存されているので,それを使用して作成してください.
スライドの内容は以下の通りです.
\begin{description}
  \item[Title:] 発表タイトル,発表者名,学会名,記載ページ,開催地(1ページ)
  \item[Introduction:] 導入,研究背景,問題設定など(1~2ページ)
  \item[Method:] 手法(何ページでも)
  \item[Result:] シミュレーション/実験結果(何ページでも)
\end{description}
\textcolor{blue}{teams} > 研究室共通 > 研究成果管理係 > 20oo に過去年度のスライドが保存されているので,それを参考にしてください.


\subsection{卒論$\cdot$修論の保管確認}
B4とM2は年度末に概要集と学位論文を提出することになっています.
概要集とは2コラムで書かれた論文の要約のことで,B4は2ページ,M2は6ページで書くことになっています.
LaTexテンプレートは学科側から配布されますが,例年配布されるのが遅いので先輩からもらったものを使っても良いです.
学位論文とはB4は卒業論文,M2は修士論文のことです.
どちらも数十ページからなる論文で,B4の卒論は研究室での成果及び引継用として残すため,M2の修論は学位授与のために必要なものになります.

\ref{subsec:conference}節の3と同様に,卒論$\cdot$修論のエビデンスも保存する必要があります.
特にB4生は学会発表を経験していない人がほとんどなので,全体ミーティングやSlackで以下の保存を催促しましょう.
\begin{description}
    \item[保存場所:] \textcolor{red}{NAS} > ws20oo > Lab20oo > Papers > 卒論概要集,卒業論文,修論概要集,修論論文 > 個人名
    \item[保存するもの:] 論文PDF,texファイル,生データ,コアとなるプログラム
\end{description}
参考までに卒論$\cdot$修論に関する2025年度の大まかなスケジュールを以下に示します.
\begin{table}[h]
\centering 
  \begin{tabular}{|c|c|c|}
    \hline
    \textbf{日付} & \textbf{スケジュール} & \textbf{提出場所} \\ \hline
    12月末 & 卒論$\cdot$修論の第一稿提出(12月報告書の代わり)& \textcolor{blue}{teams} \\ \hline
    1月頭 & 第一稿に対して先生からのフィードバック & - \\ \hline
    1月末 & B4は概要集提出,M2は概要集と修論の提出 & 大学,\textcolor{red}{NAS} \\ \hline
    2月末 & 公聴会後,卒論$\cdot$修論の最終稿提出 & \textcolor{red}{NAS} \\ \hline
  \end{tabular}
\end{table}

\end{document}

\end{inputs}

\clearpage
\begin{inputs}
\usepackage{input}
\newcommand{\Red}{\color{red}}
  \newcommand{\test}{waya}
\begin{document}
\maketitle
\section{美化係}
\subsection{チェックシート}
研究室を最後に退出する人には,戸締りや電気などをチェックしてもらっています.
そのためのチェックシートがありますのでそれを月初めに印刷して,世田谷キャンパス4階,5階,総合研究所学生室に貼ってください.
また,閉鎖障害のチェックシート,火気関係のチェックシートも同様に貼ってください.こちらは学校に提出するものなので注意してください.
これらのテンプレは美化表テンプレ.xlsxとして置いておきます.
最初の内に,12ヶ月分のチェックシートを作ってしまうと楽です.

\subsection{日々の掃除}
週1回,総合研究所学生室と実験室,本館1階,2階の野中先生が所有する部屋の掃除を行ってください.掃除前までにslackのgeneralに連絡するのも忘れずにしてください.
主な掃除としては,床の掃除機がけ,廊下の箒がけ,机をふく,空気清浄機のフィルター掃除等です.
また,2~3週間に1回は実験室のマットの張り替えを行うようにしてください.
空気清浄機フィルタのゴミを掃除機で吸うのも忘れずに.
2019年度では総研に完全移設をしたため掃除の際に人員が余ってしまいました.そのため,2020年度では2~3班に分割して週ごとに掃除を行う班を変えていくことをお勧めします.
掃除終了後に,ごみは下にあるごみ回収の場所に持っていってください.
※忙しい時期だと,いつもの日程で掃除を行なえない場合があります.そのときは,別日に掃除をするようにしてください.研究室メンバーへの連絡も忘れずに.
※燃えるごみは特に3~4日で満タンとなってしまうため,掃除の日でなくても捨てるようにする.
{\bf{※長期間研究室を空けるときは,必ず空気清浄機の水抜きを忘れずに.分解して乾燥させること.これをしないとカビが生えます.}}

\subsection{大掃除}
長期休暇などの前にある大掃除では,普段の掃除に加え,棚・本棚の整理整頓,エアコンのフィルター掃除,ブラインド・窓の掃除,フィールドの掃除など普段行わない場所を掃除も掃除するようにしてください.美化表テンプレ.xlsxに掃除箇所のチェック表がありますのでそれを参考に掃除を行ってください.
掃除の担当や時間割などを決めておくとスムーズに進みます.(大掃除_~月.xlsx 参照)
日時は発表会後の長期休暇に入る直前に入れると良いです.日時を決定したら早めに告知を行いましょう.
昨年度の大掃除は8月の中間発表の翌日と1月の修士生の中間発表の翌日に行いました.1月の大掃除では当日にM0が授業があった関係上,M0に世田谷の大掃除を任せて,それ以外の学生で総研の大掃除を行いました.

\subsection{ごみの分別}
ごみの分別としては,燃えるごみ,燃えないごみ,ペットボトル,缶・ビン,金属,段ボールに分けて捨ててください.
本やカタログなどを捨てる際にはビニールひもで縛り,世田谷の場合は下にあるごみ回収の場所,または階段の踊り場にあるゴミ捨て場に持って行ってください.総研の場合は,学生室の裏側にある倉庫に持って行ってください.
また,備品を捨てることになった際は,3月中に粗大ごみの回収を学校が行いますので,その時に捨ててください.

2019年度に引っ越しを行い,総研では新しく机やいす等を購入しました.その際梱包で使用されていた段ボールが大量に出ました.総研では段ボールの回収が一ヶ月に一度しかなく,回収される日にちもいつも同じではありません.また,出す場所は学生室他の研究室が出せる分を残したうえでの裏にある倉庫でスペースが決まっている上,他の研究室との共用の場所となっています.そのため,まだ本館2階に段ボールの山が残っています.申し訳ないですが,事務室に回収日を聞いたうえで回収日の1.5週間くらい前までに,他の研究室が出せる分を残して捨ててください.なお,回収日の1週間前に回収されることもあるので動向に注意しましょう.回収される時間は午前8次頃と聞いております.

\subsection{掃除用具の購入}
掃除用具は管理係と相談して購入して下さい.
消耗品としてはごみ袋,食器用洗剤,ハンドソープ,掃除シートがあります.
掃除シートは尾山台のセイジョーで販売,その他のものは学校の購買部で購入できます.

3月中に商品を購入する場合は,3月中に研究室の予算の決済などがありますのでその日程に注意して購入するようにしてください.



\end{document}

\end{inputs}

\clearpage
\begin{inputs}
\usepackage{input}
  \newcommand{\test}{waya}
\begin{document}
\maketitle
\section{安全管理係}
実験室管理係は,実験室使用予定や実験室備品等を管理する係です.ここでの実験室は,世田谷キャンパスの研究室5階フィールド,総研の実験室が該当します.


\subsection{実験室保守}
フィールドや計測機器など,実験室の保守管理を行ってください.世田谷と総研に研究室が分かれている都合上,自力で管理できるのはどちらか一つだと思います.もう一方は誰かに依頼しておくのをお勧めします.

総研実験室に引かれたカーペットの配置には意図があります.掃除などで一度片づける際は,実験で使う人に確認を取りながら並べてください.2018年度は,カーペット張り替えは毎月初回の掃除の日に行っていました.美化係と話し合って決定してください.

総研実験室のPrimeの配置は,カーペット同様実験で使う人と相談しながら決めてください.ただ,定期的な配置変更などは行っていません.必要になった時だけ調節するようにしてください.

総研実験室のサーバーラックの中身に関しては,PC係が主に管理していました.PC係がいなくても中身がどうなっているか分かるよう,把握はしておいてください.

総研実験室にある学生用の机周辺はコードが散らかりやすい空間になっています.モニターに使うHDMIケーブルやLANケーブルは,机の下にある赤い箱の中に入っています.使う時だけ取り出し,使い終わったら戻すよう周知し,徹底してください.


\subsection{実験室使用予定管理}
研究室メンバーの実験室使用予定を管理します.予定管理は,研究室行事等のカレンダーと同様,Googleカレンダーで行います.カレンダー名は『実験室使用予定』です.研究室メンバーにカレンダーを共有し,実験室を使う際は各個人で該当する時間に予定を書き込んでもらうシステムです.2018年度は,以下のようにカレンダーへの記入を指定しました.〇〇は使用者の名前です


~~・世田谷5階フィールド→『5階 〇〇』

~~・総研実験室全体→『実験室全体 〇〇』

~~・総研実験室手前のみ(フォースプレート側)→『実験室手前 〇〇』

~~・総研実験室奥側のみ(カーペット側)→『実験室奥 〇〇』


研究室行事が被らない限り,基本的に実験室使用権利は最初に予定を入れた人が優先です.記入方法,予定に関するルールをメールで共有してください.新B4生のアクセス権限追加も忘れずに行ってください.

引き継いだ時点で,カレンダーのアクセス権限には前年度B4,M2の卒業生も含まれています.設定画面からアクセス権限を開き,卒業生のアクセス権限を解除してください.卒業生のメールアドレスはOB会係が把握しています.2019年度はM1の安部に確認し,カレンダーのアクセス権限の欄から卒業生のメールアドレスを探してください.

カレンダー用アカウントのメールアドレスおよびパスワードはここに記入出来ないため,紙に書いて2020年度M1生の松浦に託しました.受け取ってください.

\color{red}{\bf 総研の実験室は高機能研として使用しているものではなく,『インテリジェントロボティクスセンター』としてロボ研と共同で使用しているものになります.使用予定カレンダーに関する連絡は,ロボ研に所属している総研の実験室を使う学生にも必ず共有してください.特に注意しなければいけないのが,高機能研の行事で実験室を使用する場合です.事例研究や総研でのデモなど,高機能研行事カレンダーに書いてあってもそれをロボ研の学生は見ることが出来ません.総研実験室の使用が出来ない日や時間帯は,必ず『〇〇のため総研実験室使用不可』などの予定を使用予定カレンダーに書いてください.}\color{black}

\subsection{Prime管理}
今年から総合研究所に移動となり,世田谷キャンパスと等々力キャンパスどちらにもPrimeがあります.Primeの保守点検を行ってください.また,Primeからのデータを取得するためのプログラム(通称Packet Client)の作成から保守までを行います.2018年度からMATLABが個人のPCにインストールできるようになったため,2017年度まで使っていたCプログラムで記述されたPacket Clientではなく,MATLAB用のPacket Clientを作成しました.雛形を作成してあるので基本的にはそれを更新していってください.元のプログラムはOptitrackのホームページのNatNetSDKをダウンロードすればわかると思います.アップデートが入るたびにデータの形式が変わっていないか確認してください.

不明な点がある場合はOptitrackのwikiを検索してみてください.


\subsection{その他連絡事項}
(2019/3/1版)

(1)総研実験室において,Primeのマーカーの管理に関する管理方法が曖昧なままでした.車椅子やUAV,障害物やフラフープなど,固定ではなく張り替えたりして使用しているため,使い終わっても付けっぱなしであることがありました.稀に机に放置されていることもあったため,何か管理方法は確立した方がいいと思います.

(2)実験室予定に名前があるにも関わらず,その時間になっても使用者が研究室に来ていないということが何度かありました.その間実験室の利用が出来なくなってしまうため,その点にも何かルール(何分来なかったらその日の予定はカレンダーから削除し優先権は他の希望者に譲渡など)を決めた方がいいかもしれません.実験室を使っていなくても研究室にいるような場合は直接理由を聞けるので問題はないと思います.

\end{document}

\end{inputs}

\clearpage
\begin{inputs}
\usepackage{input}	
  \newcommand{\test}{waya}
\begin{document}
\maketitle
\section{備品管理係}
\subsection{蛍光ランプの交換}
管理係は,研究室内の切れた蛍光ランプの交換を行う.
%%
\subsubsection{蛍光ランプの交換手順}
世田谷での蛍光ランプの交換手順は以下の通りである.
\begin{enumerate}
\item 脚立等を使い,蛍光ランプを取り外す
\item 取り外した蛍光ランプを1号館の施設管理課へ持っていく
\item 切れた蛍光ランプを持っていくと施設管理課から切れた本数分の蛍光ランプがもらえるので,それを受け取る
\item 脚立等を使い,蛍光ランプを取り付ける
\end{enumerate}

総研での蛍光ランプの交換手順は以下の通りである.
\begin{enumerate}
\item 脚立等を使い,蛍光ランプを取り外す
\item 取り外した蛍光ランプを総研本館の事務室へ持っていく
\item 切れた蛍光ランプを持っていくと事務室から持って行った本数分の蛍光ランプがもらえるので,それを受け取る
\item 脚立等を使い,蛍光ランプを取り付ける
\end{enumerate}
総研の脚立を片付けるときは,指を挟まないように注意する.寝かせて片付けるのが良い.
%%
\subsubsection{蛍光ランプの取り外しと取り付け}
研究室の蛍光ランプが取り付けられている照明器具は,電池を外すようにランプを片側に寄せてから下方向に引いて外す.取り付け時は逆の手順で行う.総研実験室の蛍光ランプにはカバーがついている.真ん中にある引っかかりを押しながら外す.

\subsubsection{その他の留意事項}
総研では2019年度に蛍光灯の取り換えを多く行ったが,2020年度以降も取り換えが頻繁に起こる可能性があるため注意すること.世田谷の研究室では,4階の照明はLEDなので交換の必要は無く,交換が必要なのは5階のみである.また研究室の天井は高いため,一人で交換を行うと蛍光ランプを外したり取り付けたりする際にランプを持ったまま脚立を昇降しなければならず,非効率な上にやや危険である.よって,ランプの交換は2人で行う方が良い.その際,一人が脚立に登って交換作業をし,もう一人が脚立の下で外した蛍光ランプの受け取りや新しい蛍光ランプの受け渡しを行うと効率的である.


\subsection{備品の調達}
研究室で使用しているホワイトボードマーカーや養生テープなど,研究室で必要となる備品を購入する.総研関係の備品,世田谷関係の備品共に購入の際は事前に購入希望物品とその理由を野中先生か関口先生に伝える(可能であれば双方に伝える).基本的に月に一回まとめて購入する.調達時の支払方法は仮納品書と立て替え払いの2種類がある.

\subsubsection{経費での購入可能品目}
以下に代表的な研究室運営費で購入できる備品を示す.

\begin{enumerate}
\item ゴミ袋(45L以上)
\item ホワイトボードマーカーの替えインク
\item 印刷用コピー用紙
\item プリンタートナー(純正品)
\item 消毒用アルコール(衛生維持のため)
\item 清掃用スポンジ(激落ちくん等)
\item 雑巾
\item ビニール紐(資材梱包・研究のため)
\item 空気清浄機フィルタ(精密機器保護のため)
\item セロテープ(50mm幅、資材梱包・論文作成のため)
\item ドラフティングテープ(掲示物貼り付けのため)
\item 両面テープ(薄型、プライム用マーカー貼り付けのため)
\item 養生テープ(資材梱包・マーキングのため)
\item 結束バンド
\item テプラカートリッジ
\end{enumerate}

\subsubsection{経費での購入不可品目}
以下に示す備品は,研究室経費では落とせない

\begin{enumerate}
\item ベープ詰め替え
\item ポット洗浄中
\item 害虫駆除関連
\item ティッシュ
\item ウェットティッシュ(アルコールシート)
\item 食器用洗剤
\item ハンドソープ
\item スポンジ
\item 水切りネット
\item 台ふきん
\item 個人的に使用するもの
\end{enumerate}

\subsubsection{仮納品書での購入}
等々力,世田谷キャンパスの文具ストアで購入する際は,仮納品書を店員に書いてもらい,購入する.書いてもらった仮納品書は野中先生か関口先生に渡す.この場合,管理係が立て替え払いを行う必要は無い.

\subsubsection{立て替え払い}
キャンパス内の文具ストア以外の一般の店舗で購入を行う場合は,管理係が立て替え払いをする.この時,領収書を店員に切ってもらうことを忘れないようにする.領収書を切ってもらう際には,以下の「領収書に関する覚書」を参照されたい.また,立て替え金を自分で用意できない場合は,先生に相談すること.
\begin{shadebox}
%%
\begin{center}
領収書に関する覚書
\end{center}
%%
\begin{itemize}
\item 金額は5万円未満,合計で3万円以上になる場合は明細を示す
\item 宛名は「東京都市大学」
\item 品目は具体的に.例:抵抗,アクリル板,アルミ板他
\item レシートは不可.社印が必要.\underline{3万円以上は収入印紙が必要}
\item 可能な限り早く清算する
\item 可能な限り領収書の数を少なくする
\item 文具ストア(ハヤト商事)で買えるのなら,先生がそちらで購入
\item 東急ハンズで購入する場合は割引カードを使用する
\end{itemize}
%%
\end{shadebox}
%%

\subsubsection{購入希望備品情報の収集}
管理係が研究室で必要となっている備品の情報をすべて把握することは困難である.そのため,研究室のメンバーから必要な備品の報告を受け,それを基に備品を購入するのが慣例である.しかし,口答でのみ報告をされると購入希望備品が漏れ落ちることもある.こういったことを防ぐため,購入備品リストを作って管理したり,購入してほしい備品を研究室のメンバーが自由に記入できるようなファイルをworkspaceに置いておき,随時記入してもらうなどの方法を検討すると良いかもしれない.

\subsubsection{その他の留意事項}
冷蔵庫等の大きな備品を購入する際は,購入により捨てることになる古い備品の処分方法や,新しく購入する備品を廃棄する際のことを考えて購入する.
研究室運営費で購入できないものは,会計係で余っているお金を使用させてもらうことが慣例となっているようである.\\
%しかし,2018年度からはキャンパスが分かれたため,管理係が立て替えておき,適宜該当キャンパスの学生に請求することをおすすめする.
\ キャンパス内の文具ストア以外の一般の店舗で購入を行うとき,領収書に商品名が記載されていないことがある.そのようなときには必ずレシートも一緒に先生に渡すようにする.




\subsection{捨てるものリストの作成}
年度末に,大型備品の廃棄を行う.(2016年度の日程は,申請書の提出期限は1月17日,運び出し期間は2月16日頃である.先生がこの日程を把握してない可能性があるので,時期が近付いたら確認すること.)
その際に廃棄する物をリストにまとめ,先生に渡す.捨てるものリストは一度に作成すると大変なので,捨てるものがあればこまめに作成を行うと良い.また,こちらで捨てるものを全て把握することは難しいので,先生や他の研究室のメンバーから報告を受けるようにする.
備品番号は上から3,4桁目が購入年度になっている.購入から10年経たないと捨てることはできない.
%(例:2017年度->M05\textcolor{red}{07}までは捨てることが可能)



\subsection{不要掲示物の撤去}
締め切りの過ぎている学会参加募集や,インターンシップのお知らせなどを適宜撤去する.



%\subsection{PCの管理}
%\subsubsection{PCの分配}
%4月初めの研究室のオリエンテーションの時期くらいに,4年生へPC・ディスプレイ・キーボード・マウスの配分を行う.PCの配分を行う際は,自分のノートPCを持っていない人を最優先とし,その後にMATLABやMathematicaなどの情報基盤センターや研究室がライセンスを持っているソフトを使用する予定の人へ配分を行う(基盤センターや研究室がライセンスを所有しているソフトは個人PCにインストールすることはできないため).これらの配分が終わった後に研究室のPCが欲しい者へ配分を行う.ディスプレイやキーボードなどは,適宜配分を行う.
%
%また管理係がM0の場合は,B4とM2が卒業した後にM0,M1へPCの再配分を行う.管理係がB4の場合は,卒業式前に誰がどのPCを使用するのかを決めておき,卒業後に在学生が適宜交換を行うと良い.その際,新しいPCやディスプレイへの交換を希望しない者に関してはそのままで良いが,新しいPCやディスプレイへの交換を希望する者に対しては,先に述べた要領で再分配を行う.
%
%\subsubsection{ディスプレイの管理}
%2018年度はディスプレイを購入したため,ディスプレイ割り当てシートへの登録と,ディスプレイ番号を示すシールを貼り付けた.

\subsubsection{備品の管理}
2020年度,リモートでの研究が求められるようになったため,希望者にウェブカメラ,ヘッドセットに加え,マウスとキーボードを購入した.また,購入した物品に関してはテプラで番号を記し,管理をした.
管理用の番号については管理係/2020年度備品係/備品管理番号.xlsxに記載した.

2022年度,前の年度に購入したPC等にテプラで番号を記し,管理した.管理したものについてはWorkspace2022/Work2022/Shere/備品管理に記載している.黄色のシールについては正面,全体,備品番号の画像を添付し,青いシールについては備品番号を貼った備品の画像を添付する必要がある.

\subsection{席決め}
新年度と年度末には,座席決めを行う.座席を決める際は,上の学年からあみだくじ等で座席を決めるのが慣例である.座席表は係引継ぎ時に渡すが,座席配置を決める際には「AR\_CAD」という2次元CADソフトを用いた(基本的には何を使っても良い).過去にはMicrosoft Office Visioを使っていたこともあったようである.VisioはMicrosoft DreamSpark Premiumにより,無料でダウンロードできる(学生個人のアカウントがある).

\subsection{靴箱}
新年度には靴箱の名札を更新する.名札はテプラで作成する.



\subsection{先生・学生不在時間割表の作成}
先生と学生が授業やTAなどで研究室に居ない時間を把握するための不在時間割表を用意する.なお,不在時間は先生・学生それぞれに書いてもらうため,管理係は不在時間を記入できるファイルをクォーターごとに作成するだけで良い.昨年は時間割表をworkspaceに置いておき,皆に記入してもらった.もしファイルをworkspaceに置くなら,フォルダとファイルの編集権限を「everyone」に与えるようにしておくこと.時間割表が確定したら,4部印刷しそのうちの2枚は野中先生と関口先生にお渡しし,残りの2枚はそれぞれ総研学生室と世田谷の4階に貼っておく.



\subsection{研究室の引っ越し計画・環境整備}
この節に書く内容は毎年行うものではないが,2013年度は鈴木先生のご退職による研究室のレイアウト変更,2014年度には研究室4階のリフォーム作業に伴う研究室の環境整備を行う必要があった.
また,未遂であるが2016年度には上階下階合併計画があった.
2017年度にはクウォーターが変わるごとに席替えを行った.
2019年度には世田谷キャンパス10号館の取り壊しを機に総研への完全移設を行った.
2020年度にはコロナウイルスの流行による分散登校のため,世田谷キャンパスと総合研究所を用いて密にならない席決めを行った.
研究室の引っ越し等がある場合には,管理係が計画の立案,作業時の指揮を行う.
計画の立案は余裕を持って行い,先生や経験のある先輩と相談したり,アドバイスをもらったりしながら進めると良い.
また,環境整備には多くの備品を調達したり,引っ越し日程を考えたりと作業内容が膨大となる場合がある.
そのような場合には,手の空いている研究室のメンバーと協力しながら進めるようにする.
\textcolor{red}{全て一人で行うと作業量が膨大であるため,室長などにも協力をしてもらいながら計画を進めていくようにする.}

参考までに,以下に昨年行った研究室改修工事に伴う一連の流れを書いておく.
%%
\begin{enumerate}
\item 改修工事で行う内容の決定(先生と業者)
\item 改修工事期間中は研究室を使用できないため,中2階へ引っ越すことにした
\item 中2階の座席レイアウト・電源配線・LAN配線計画の立案(必要なテーブルタップ・ハブ数等の確認)
\item 中2階への引っ越し日時と作業手順計画の立案・実行
\item 改修工事終了後に新たに必要となるタイルカーペット枚数の計算・発注
\item 改修工事後の4階の机配置,電源・LAN配線計画の立案
\item 改修工事により不要となった物品の洗い出し(捨てるものリストへ追加)
\item 机増設後に必要となる椅子の数の確認(廃棄品の置き換え含む)
\item カーペット貼り付け作業時に必要となる道具の確認・調達
\item 4階への引っ越し日時と作業手順計画の立案・実行
\end{enumerate}
%%
なお,電源は柱や壁に埋め込まれているコンセントを使用するのではなく,柱に取り付けられている白い配電盤から供給されているものを使用する.柱や壁に埋め込まれているコンセントを使用すると,ブレーカーが落ちることがある.ブレーカーが落ちると他の研究室の電源も使えなくなるため,柱や壁に埋め込まれているコンセントは極力使わないようにし,掃除機や加湿器程度に留めるのが良い.\\

また2019年度に実施された総研への完全移設における引っ越し実行までの一連の流れを以下に記す.
%%
\begin{enumerate}
\item 先生や古田さんと相談をして引っ越し日を決め,引っ越し業者に予約をする.
\item 机を処分するものと総研に置くものに分ける.このとき机は幅が100\,cm\,以上のものを総研のものとし学生の人数を考慮して不足している机を購入した.
\item 本館2階の備品を処分するもの,2階に残すもの,1階に移動させるものに分ける.
\item 学生室,本館1階,本館2階のレイアウトを作成する.このとき部屋の大きさを考慮した上で机の配置を決める.
\item 椅子を学生室,総研1階,総研2階,処分するものに分ける.
\item 電源配線,LAN配線を決める.このとき必要な電源タップ,ハブの数を確認する.
\item 引っ越し当日に会議室や踊り場に学生室の机や冷蔵庫を移動させるため,その配置を決める.\\以上の内容を引っ越しの前々日までに決める.
\item 引っ越しの前々日までに学生全体に机の引き出しが飛び出さないように養生テープで固定させる.
\item 引っ越し前日に以下のように配置を行った.\\実験室:学生室のディスプレイや荷物等\\会議室:学生室で使い続ける机や椅子と総研1階の机\\学生室:総研の2階で使用する机または椅子,会議室に入りきらなかった椅子\\総研踊り場:世田谷に運ぶ机と椅子
\item 各部屋に人数を配分して,引っ越しを実行
\end{enumerate}
%%
\ 引っ越し当日の主な内容としては,机や棚,不用品の運び出し,備品の移動,床下配線の再構築,机や棚の設置,新しい机や椅子の組み立て,床上配線の構築,窓掃除やエアコンのフィルターの掃除などといった清掃,各自の荷物の設置となる.\\
\ なお,処分をする机や椅子は全て世田谷キャンパスに移動させた.このとき世田谷のレイアウトも同じように決めておく.大がかりな机の移動であったため,移動場所を明記した養生テープを机に貼った.
引っ越し当日は必ず50分程度の休憩を挟むようにする.\\
\ また,総研に移設したばかりであるため,本館1階などの備品が不足していることがあるため,随時対応をできるようにする.


また,2020年度のコロナウイルスの流行を受けて,研究室では世田谷キャンパスと総合研究所への分散登校を行った.そのため,今までの席配置では無く,席をひとつおきに確保した.2019年度に実施された引っ越しと同様に席配置などをExcelファイルで作成し,机などの移動はほとんどすることなく,ソーシャルディスタンスを保った席配置を行った.参考として,管理係/2020年度備品係に席決めなどに用いたExcelファイルを置いておく.また,2021年1月にパーテーションに使うアクリル板を入手したため,今後パーテーションの設置を行う可能性がある.



\subsection{カーペット貼りの手順}
2014年度に行った研究室の改修工事の際に,カーペットを貼り直した.ここでは,改修工事後に行ったカーペット貼りの手順を説明する.

カーペットはオフィス用のタイルカーペットを使用する.また,貼り付けにはカーペット用接着剤を使用した.なお,2013年度にもカーペット貼りを行ったが,その際には一般の接着剤を使用した.カーペットを貼る際に使用した道具を以下に示す.
%\begin{center}
%カーペット貼りに使用した道具
%\end{center}
\begin{itemize}
\item タイルカーペット
\item 黒刃カッター
\item 床用接着剤
\item ハケ
\item 発泡スチロール製容器(接着剤取り分け用)
\end{itemize}
%%
ここで,黒刃カッターはカーペットを壁や床の形に合わせてカットするために用いた.カーペットは非常に切りづらいので,切れ味の良い黒刃カッターを使用すると良い.また,床用接着剤は大きな容器院入っていたため,小さな発泡スチロール製容器に取り分けて使用した.そして容器に取り分けた接着剤をハケですくい,床に塗布した.

次に,2014年度に行ったカーペットの貼り付け手順を以下に示す.
%%
%\begin{center}
%カーペット貼り付け手順(2014年度)
%\end{center}
%%
\begin{enumerate}
\item カーペットが貼り付けられていない状態で,床の掃き掃除及び拭き掃除を行う
\item カーペットを配置する
\item カーペット用接着剤を,カーペット1枚分ずつ開けて床にハケで塗る
\item 白い接着剤が半透明になるまで乾かす
\item 接着剤が塗られた床の上にカーペットを敷く
\end{enumerate}
%%
以上の手順でカーペットを敷いた.カーペットの敷き方は,使用するカーペットや接着剤により若干変わってくるので,接着剤の注意書きなどをよく読むと良い.なお,研究室には沢山の備品があるため,荷物を全て外に出してカーペットを敷くことは難しい.そのため部屋を半分に分けて片側に荷物を移動して寄せ,もう片方で作業するというように,半分ずつカーペットを敷いていった.また,カーペットの切断には時間がかかるので,カーペットを貼る役割の人と切る役割の人に分けて作業を行った.
%%
\subsection{世田谷キャンパスにおける定期的な断水・停電時の対応}
世田谷キャンパスでは定期的な点検のため大学全体で断水や停電を行う日がある.この日程については予め連絡されるので,確認してカレンダーに記入しておく.これに備えて各階で行うべき対応を以下に示しておく.また,作業を管理係一人でこなすのは大変なので,研究室のメンバーに協力を仰ぐとよい.
2019年度の停電,断水,ネットワーク停止は8月中旬から下旬に行われた.また,消防設備やガス設備の点検も同じ期間に行われた.

停電の際は,学生室に関しては,各々の机周辺のコンセントについて帰宅時に抜いてもらうようにしておくように連絡しておくのが良い.
ミーティング部屋およびフィールドについてもコンセントを抜く.
天井のコンセントは脚立を用いる必要があるので注意すること.
ただし,先生の部屋につながっているコンセントについては抜かなくてよいが,確認はしておくこと.(先生自身が居室内のものに関して処理されるはずなので)
また,サーバPC(workspace)に関しては,PC係にシャットダウンをお願いし,その後コンセントを抜くこと.
停電後は最初の活動日に原状復帰すること.

断水の際は,断水後の活動開始前に水道を30分ほど流しっぱなしにし,汚い水を流す.
 % %
 % \begin{figure}[tb]
 %   \centering
 %   \includegraphics[width=0.8\linewidth]{management/plug_arrangment.eps}
 %   \caption{研究室5Fの配線およびコンセント位置}
 %   \label{fig:plug_arrangement}
 % \end{figure}
 % %
 %
%
\subsubsection{世田谷5階}
停電に備えてコンセントを抜く.配線とコンセント位置に関しては大まかに
管理係の引き継ぎフォルダ内の「配線とコンセント位置5F.pdf」
% Fig.~\ref{fig:plug_arrangement}
に示すので参考にすること.

\subsubsection{世田谷4階}
壁や柱に刺さっているコンセントは抜き,配電盤のブレーカーを落としておく.
なお,ブレーカーを落とす際には,全員のパソコンの電源が落ちていることを確認する.(パソコンをつけっぱなしにしている人がいるため)

\subsubsection{プロジェクター}
世田谷4階のプロジェクターは電源に不具合があり,スリープ状態から復帰が出来ない.そのため,使用後冷却が完了してから電源プラグを抜き,使用時に電源プラグを差してから電源を入れることで対応している.



\subsection{プリンターの管理}
プリンタートナーの残量を管理する.プリンターのIPアドレスをwebで検索すると,詳細なプリンターの状態が見れるので,お気に入りに登録しておく.残量レベルが少なくなったら,関口先生にトナーの注文をお願いする.この時,ヨドバシカメラのURLをまとめたものを送る.使用されているプリンタのトナーはC310Hという型番のものを使用している.過去にリサイクルのトナーを使用したことにより論文が印刷できなかった事例がありました.そのため\color{red}{トナーは正常に印刷が行えるように,必ず純正品を購入するようにする.}\\
% 関口先生にトナーの注文をお願いする.この時,アマゾンのトナー販売ページのURLをまとめたものをメールすると良い.
\color{black}{
\ また,トナーを交換した際に使用したページ数をプリンター横にある表にメモしておく.
プリンター用トナーを交換した際の使用済みトナーは,交換したトナーが入っていた箱に入れ,使用済みシールを貼ったうえ,購買で回収してもらう.\\
\ 2020年3月現在,総研のトナーはリサイクルのものを使用しているが,これは無くなるまで全て使い切る.
}

\subsection{使用済みの電池の処分}
電池は裏門横のゴミ捨て場に持って行けば回収してもらえるはずです.
電池の研究室での回収場所は,先輩に聞いて下さい.(M0に場所がどこか伝えて卒業してください.)

\end{document}

\end{inputs}

\begin{inputs}
\usepackage{input}
  \newcommand{\test}{waya}
\begin{document}
\maketitle
\section{会計係}
その名の通り,お金の管理をする仕事です.
引継ぎ資料として,workspaceのoishi/会計係に普段使用していたエクセルを上げてあります.
下に示すことに気をつけて,仕事をすれば大丈夫です.
\begin{itemize}
\item 領収書またはレシートを必ず回収しましょう.
\item 全体の名簿を作成しましょう.参加費の回収時にチェックしましょう.
\item 告知しても大体払ってもらえないので,自分で回収しに行きましょう.
\item パーティー係と連携して仕事を行うと効率的です.
\item 居酒屋で飲み会をするときは,勘定が出来なくならないようにセーブして飲みましょう.
\item 収支をエクセルなどでまとめましょう.月に1度行うことをお勧めします.
\item 鍵付きの机をもらいましょう.
\item 総研,世田谷組で平等にお金を使いましょう.
\item 卒業祝い品のお金を貯めましょう.M2に3000円,B4に1500円分が目安です.
\\
以上の様な仕事が予定されていた.
\end{itemize}
\end{document}

\end{inputs}


%20170819_追加 (津野)...
\begin{inputs}
%研究室生活を円滑に進めるために,全員に周知しておきたい内容を記入

%更新日:20180115_津野
%更新日:20190321_柴田浩志

%% レポートテンプレート
\documentclass[onecolumn]{jsarticle}
%% パッケージの設定
\usepackage[dvipdfmx]{graphicx}
\usepackage{msethesis}
\usepackage{amssymb,amsmath}
\usepackage{bm}
\usepackage{url}
\usepackage[dvipdfmx]{color}
\usepackage{wrapfig}
\usepackage{comment}
\usepackage{input}

\begin{document}

\section{学部生の引継ぎ (\color{red}学部生要閲覧\color{black})}

\subsection{1年の流れ}

一年の流れについて列挙します.
略記:B(学部),M(修士),D(博士)
%
\begin{table}[h]
  \centering
  \caption{1年の流れ}
  \label{table:StreamOfTime}
  \begin{tabular}{|l|l|c|c|}\hline
月&時期&内容 &備考\\ \hline 
4月&オリエン期間&研究室オリエンテーション & 研究室始動\\ \hline 
4月&始動より約1週間後&卒論構想発表会 &対象B4 \\ \hline 
5月&GW明け&中間発表会 &対象B4~D \\ \hline 
6月&中旬頃&報告会 &対象B4~D \\ \hline 
7月&中旬~下旬&B3向け研究室紹介 &B4中心準備 \\ \hline 
8月&第1週&中間発表会 &対象B4~D \\ \hline
8月&第1週?&外部向けオープンキャンパス &B4中心準備 \\ \hline 
8月&下旬まで&夏休み&中間発表会の報告書提出後 \\ \hline
8月&下旬&修士論文中間発表会&学科主催 \\ \hline    
9月&中旬?&合宿 &B4旅行係計画 \\ \hline 
9月&下旬&B3研究室配属 &B4中心研究班紹介 \\ \hline
10月&上旬&報告会 &対象B4~D \\ \hline
10月&下旬&世田谷祭 &B4準備 \\ \hline 
11月&中旬&報告会 &対象B4~D \\ \hline
12月&クリスマス前&中間発表会 &対象B3~D \\ \hline
%1月&&@@ & \\ \hline 
2月&上旬&事例研究最終発表会 &対象B3 \\ \hline 
2月&中旬&修士論文公聴会 &学科主催 \\ \hline
2月&中旬&卒業論文公聴会 &対象B4 \\ \hline 
3月&&引継ぎ & \\ \hline 

  \end{tabular}
\end{table}

\subsection{発表会について}

ほぼ毎月一回発表会を行います.
各自発表内容について,パワーポイントを作成しその時々の時間に従って発表を行います.
発表会についての大まかな流れは以下に示します.
\begin{itemize}
  \item 発表会前にパワーポイントを先輩等の添削を貰いつつ作成
  \item 中間発表会の場合,発表前日までに報告書の仮提出を行う
  \item 作成したパワポを発表係が指定した締め切りまでに(/workspace/lab\_data/papers/西暦/指定フォルダ)まで提出する
  \item 発表会中は一人一回は質問を行う
  \item \color{red}発表会後,1週間以内に報告書の提出を行う\color{black}
\end{itemize}
報告書の提出場所は,その時の先生の指示に従ってください.(workspaceかWebClassに提出)
B4は2カラムの卒業論文概要集形式で報告書を提出.M1以上は報告書か論文などを提出.
12月中間発表会の後に,B4卒業論文の第一稿仮提出がありました.
ここよりB4は1カラムの文章となります.

外部の学会発表などの予定が発表会の日程の近くにある場合は,発表会を学会発表練習に置き換えることが出来ます.
発表時間,質疑時間は発表先の時間に合わせてください.

また,修士2年の修論発表(夏の中間発表も)では4年生の手伝いが必要です.
この手伝いは高機能研のときだけで結構です.
カメラ,質問のメモ,タイムキーパーに加えてマイクを運ぶ人が必要です.
これは質疑で挙手をされた先生のもとへマイクを持っていく係です.
本番までにあらかじめ決めておくとスムーズに運営できます.

\subsection{研究室紹介のパワーポイントについて}

研究室では,研究室紹介のパワーポイントが英語版と日本語版の2つあります.
英語版は実験やシミュレーション結果をたくさん張ったもので,日本語版は主にオープンキャンパスや3年生向けに使うものです.
なので,英語版では歴代の見栄えする動画を毎年載せていってTCU-ACSL-Best collectionみたいな感じにします.
今までのも最近の研究も併せて載せるように.逆に日本語版が使われる状況には時間制限がある場合が多いです.
なので,載せる結果を絞って載せてください.入った後でがっかりさせないように直近の結果の方がいいです.
また,研究室の紹介の要素も忘れずに.毎年,英語版は5月ごろM0が更新します.
日本語版は6,7月ごろ研究室紹介で全体紹介する人が更新します.
上記のことに注意して作成してください.

\section{修士生の引継ぎ (\color{red}修士生及びM0要閲覧\color{black})}

\subsection{1年の流れ}
表\ref{table:StreamOfTime}を参照.
修士生は主にB4のサポートがメインになります.
B4のときに担当していなかった係でなくとも,気にかけてあげてください.
「B4が聞いてこなかったから言わなかった.知っていると思った.」は許されません.

\subsection{発表会について}
編集中

\subsection{学会について}
\subsubsection{論文作成の注意点}
学会論文ではフォントを埋め込んだpdfファイルを提出することが要求されます.
埋め込まれていない場合,スマホや他のPCでの閲覧時や印刷時に文字化けを起こすことがあります.
従って,学会に論文を提出する場合は,以下の手順でフォントが埋め込まれていることを確認してください.
\begin{enumerate}
 \item 確認するPDFを任意のリーダー(Adobe Acrobat,Foxit など)で開く
 \item プロパティのフォントを選択,そのPDFで使用されているフォントの一覧が表示されます
 \item 全てのフォント名の横に''(埋め込みサブセット)''がついていることを確認
\end{enumerate}
もし,'(埋め込みサブセット)''がついていないフォントが存在していた場合,
「svn://192.168.120.100/lab/係引き継き資料/works/全体的な引継ぎ事項/TeXpdfフォント埋め込み方法他.txt」に解決方法が書かれているので参照してください.


\subsection{Presentation~Competitionについて}

Presentation~Competition~(PC)では,英語による論文・ポスター作成,発表技法の向上を目的として,機械システム工学専攻の修士1年がポスターを用いて発表を行う.
本項では,PCに参加するにあたって注意すべき項目をまとめておく.
また,「http://www.mse.tcu.ac.jp/student/pc」より,日時やフォーマット等の案内を確認すること.
\begin{itemize}

  \item 履修登録\\
	PCは大学院の授業科目である「機械システム工学専攻事例研究」として行われる.
	単位も付与されるため,履修登録時に忘れずに登録しておくこと.

  \item 発表日までの計画\\
	PCの発表日は例年,7月の中旬頃に予定される.
	それまでに後述する予稿・ポスターの作成や発表練習があるため,
	それらの添削及び発表練習日の計画を立てること.
	人によってはSICEなどの学会や,夏のインターンシップの申込,
	TAの業務などと重なるため,それらを踏まえた上で計画することが望ましい.
	
  \item 予稿\\
	例年,A4サイズの1ページで発表内容の概要を作成している.
	発表日の1週間程度前に提出期限がある.
	PCの発表内容の骨子となるため,早めに作成を始めて仕上げておくことが望ましい.

  \item ポスター\\
	学科webサイトのフォーマットに従い,作成すること.
	発表日前日は他研究室の学生が印刷するため,
	情報基盤センターが混雑する恐れがある.
	遅くとも発表日前々日には印刷してあることが望ましい.

  \item 発表練習\\
	ほとんどの修士1年にとって,PCが初めての英語発表の場となる.
	そのため,複数回の段階に分けて発表練習を行うことが望ましい.
	前年度~(2017年度)は4回の発表練習を行い,
	1回目から,日本語原稿・ポスター,
	日本語原稿・英語ポスター,英語原稿・ポスター,英語原稿・ポスターで行った.

  \item レポート\\
	PC発表後,指定された学生の発表内容をレポートにまとめて提出する必要がある.
	提出期限があるため,早めに提出しておくこと.

\end{itemize}

%その他詳細な引継ぎ事項は「」を参照.

\subsection{ATACSについて}
修士1年はATACSにおいて特に作業が無いため,
この節では修士2年を対象とした引継ぎ事項をまとめる.
また博士課程の学生については内容が概ね重複しているため省略する.
\par
2018年度のATACSにおける修士2年の活動内容を表\ref{table:atacs_schedule}にまとめる.
%%
\begin{table}[htb]
	\caption{ATACS2018:修士2年の活動内容}
	\centering
	\begin{tabular}{|c|l|}\hline
		日付 						& 内容											\\ \hline\hline
		9/25							& 発表タイトル締切(ATACS係)		\\ \hline
		9/30							& 発表タイトル締切(ATACS幹事)	\\ \hline
		10/4							& 予稿締切(ATACS係)					\\ \hline
		10/5							& 予稿締切(ATACS幹事)				\\ \hline
		10/11						& 発表練習									\\ \hline
		10/12 ~ 11/14 		& ATACS本番								\\ \hline
	\end{tabular}
 	\label{table:atacs_schedule}
\end{table}
%%
発表タイトル・予稿は「修士2年→ATACS係」と「ATACS係→ATACS幹事」のそれぞれ2回の提出期限が設けられている.
後者の提出についてはATACS係に一任しているため,発表者である修士2年は前者の提出締切日を守る事.
発表練習は予定が合えば研究室全体で行うため,日時や場所について発表会係と前もって相談する事.
予稿の体裁はA4・2カラム・4~6ページ(超過可)であり,予稿テンプレートは事前に幹事から送付されたものを使用する事.
修士2年の発表時間は17分(発表:12分,質疑応答:5分).

\subsection{TAについて}

TAを行うにあたって,いくつか書類を出す必要がある.
\begin{itemize}
  \item 振込依頼表\\
	TAの給与の振込口座を指定するために必要.
	初回給与支給日の七日前までに提出する必要があるが,
	年度の初めに次の書類とともに準備することが望ましい.

  \item 扶養控除申告書 or 扶養控除申告書 他事業所への提出報告\\
	該当するどちらかの書類を提出すること.
	現在アルバイトをしていて扶養控除を出した人は後者を提出.
	何それ?って人は前者を提出すること.
  
  \item TA出勤簿\\
	この書類のみは毎月提出する必要がある.
	従事した時刻と業務内容,先生および自分の押印を忘れずに.
\end{itemize}

以降は各授業のTAに対しての引継ぎを記す.

\subsubsection{Cプログラミング}
編集中

\subsubsection{機械システム応用実験}
編集中

\subsubsection{技術日本語表現技法}
第1回で担当する学生を決定する.
(人数で等分配してもよいが,1年生はかなり能力差が大きい.
そのため大抵の場合人数で等分配すると失敗するので注意が必要.)

講義の始まる10~15分前には1号館へ行き,鍵を借りて置く.
また,学生にプリントを予め配布しておく.
このとき,余分にPC端末を起動しておく.
あらかじめ何台かのPCに自分のアカウントでログインしておいて,
何か問題が発生した場合にWebclassだけ学生自身のアカウントでログインさせても良い.
講義が始まったら学生の出席を確認.遅刻した人はどの程度遅刻したかをチェックしておく.
欠席や遅刻の学生は野中先生があらかた講義を終えた際(演習中)に先生へ報告する.
要チェックな学生は名簿自体にわかりやすく目印をつけてい置くとよい.

欠席した人は学番の近い学生に聞いて何か知らないか確認.
後日講義に来た場合は欠席について本人から確認しておく.課題が提出されていない,
著しくクオリティが低い場合も同様.本人に直接話を聞くこと.

\subsubsection{電気基礎実験}

主な業務は
\begin{itemize}
  \item 実験の補助
  \item 提出レポートの添削
\end{itemize}
実験を円滑に進めるために,各実験の数日前にはTA内で事前実験を行うこと.
前半・後半に学生を分けるため同じ実験を2回行うが,事前実験は1回のみで良い.

\subsubsection{ロボット制御プログラミング}
第1回目は,WebClassに名前がない人をWebClassに追加する作業を行う.
そのため,1年前にロボット制御プログラミングのTAを行っていた先輩と相談し,第1回目に手伝ってもらうようにする.
ただ,このときの先輩のTAの給料は基本的に出ず,ボランティアになるので注意.
第1回目は,担当者がWebClassへの追加を行っている最中は,先輩が質問への対応を行い,追加が終わり次第2人で質問への対応を行う.
\par
講義が始まる前は,野中先生と一緒に研究室から出ていき,学科事務室の池田さんのところで配布資料を受け取ったあとに教室の鍵と机のロックを解除するための鍵の2つを取りに行く.
先生が一緒に行く場合は,先生が基本的にやるので一緒についていくだけでよい.
ただ,先生がお忙しい場合は,担当者が先に行く必要があるため,学科事務室の池田さんのところでロボット制御の配布資料を野中先生に言われて取りに来た旨を伝え,資料をもらうことと,学生支援センターで教室の鍵と机のロックを解除する鍵の2種類をもらう.
その後,教室へ行き,机のロックを解除し,プロジェクターを降ろす作業を行う.
また,毎回の授業の最初に小テストを行う事が多いので,その場合はすぐにテスト用紙を配れるように各列の席数分ずつで用紙を分けておくこと.
授業で使用する資料については,小テスト中に用紙を分ければ間に合うが,小テストは開始後すぐにやるので,しっかりと準備する.
授業中は,先生が説明しているときには教室の後ろで立っているか,一番前の席の廊下側の席(TAの担当者の席)に座って待機する.
先生の説明が終わり,授業課題を行う際に教室内を回って,質問へ対応する.
このときには,手を挙げている学生への対応はもちろんのこと,回る最中にすこし画面を見て苦戦していたり,悩んでいたりする人にアドバイスをするとよい.
\par
基本的に,余った資料はTAが管理する.
2回分くらい前の資料は,毎回の授業で持参すると,前回の授業に出席していなかった人の対応が楽になるのでおすすめ.
また,資料が足りない場合はとりあえずの応急処置で周りの人に見せてもらうようにしてもらう.
最初の授業での出席者をもとに枚数を計算するので,毎回修正はするものの足りなくなることがたまにあるので注意.
\par
採点については,基本的に担当者はしなくてもよい.
野中先生の採点基準で採点を行う事が多いので,採点しなくてもよい.
ただ,採点基準が明確だったりする場合は採点してもいい.

\end{document}
\end{inputs}
%...20170819_追加 (津野)

 %{\bf inputで読み込んだファイルの中身ここまで}\\
\end{document}

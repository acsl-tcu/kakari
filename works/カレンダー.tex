\usepackage{input}
  \newcommand{\test}{waya}
\begin{document}
\maketitle
\section{カレンダー}
研究室員が共有するグーグルカレンダーがあります.そこにゼミ・報告会などの研究室に関係するイベントの日付を「高機能研行事」として書き込んでください.また,学会に関する日付(学会を行う日付や論文締切等)は「学会」としてカレンダーに書き込んでください.なお,自分の場合学会の日付は先輩に頼まれてから書いていました.去年分のカレンダーをみれば,書いた分があるので良ければ参考にしてください.あと,ミーティングのときにはメモをとり,日付を確認したらなるべく早くカレンダーに書き込むようにしていました.

\end{document}

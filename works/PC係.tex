\usepackage{input}
\usepackage[dvipdfmx]{hyperref}
\usepackage{wrapfig}
\usepackage[dvipdfmx]{graphicx}
\usepackage[dvips]{graphicx}
\usepackage{listings}
\begin{document}
\maketitle
\section{PC係}
PC係の仕事は大きく分けてパソコンの保守管理と覚えておくべき知識から構成されます.以下にそれぞれ書いていきます.
\subsection{パソコンの保守管理}
パソコンの保守管理は2つに大別できて,1つは研究室に関係するパソコンの管理,もう1つは研究室のサーバ管理です.
\subsubsection{パソコンの管理}
セキュリティ対策のために,この研究室の研究生にWindows Updateはこまめに更新確認をし,
必要であれば適宜更新をするよう勧告してください.
研究室のPCも更新確認をし,必要であれば適宜更新するようにしてください.
ただし,Windows Updateを更新することによりパソコンに不具合が生じることが稀にありますので,
ネットでWindows Update関係のニュースを確認するようにし,問題がなければ更新するようにしましょう.
また,各種ソフトの緊急アップデートなどもあるので,情報の収集に務めてください.
Windows Updateを先延ばしにできても適用しないことは出来ないようになっているので,
しっかり情報収集を行ってください.

研究室で所持している各PCのスペックの管理と誰が使っているか,使っていないPCはどこにあるのを把握することもPC係の仕事となります.
また,PCの他にディスプレイ,LANケーブルに関しても管理しているので注意してください.
基本的にはExcelに一覧として纏まっています.また,PC,ディスプレイを新規購入したり,既存のものと交換した際には新しく管理番号の追加,編集を適宜行うようにしてください.
このExcelファイルは自分のローカルではなくOnedriveのworkspaceに保存しておき,常に最新版が他の人も必要に応じて閲覧できるようにしてください.
研究室で貸し出しているディスプレイに関してもPC同様に管理を怠らないようにしてください.
また,リモート環境になったことで,研究室所有のPCのIPアドレスの固定を進めています.
こちらもExcelで管理しているので適宜更新してください.

研究室で所持しているPCと個人で使用しているPCどちらにおいても,ウイルス対策ソフトの導入を必ず行わせるようにしてください.
個人で使用しているPCに関しては所持している人各々でソフトの導入を行うようにしてください.
ESETが廃止されたため現在はWindowsDefenderのみとなっています.

年に1回程度総研,世田谷それぞれですべてのPCの内部を清掃しています.清掃には総研では本館1Fに置いてあるエアコンプレッサーを使って埃を吹き飛ばしていました.研究室のメンバーに協力してもらい,PC係は清掃の指揮をお願いします.特にリモートデスクトップで使用するため24時間稼働しているPCも多く,埃が溜まりやすくなっているので注意してください.

\subsubsection{サーバ管理}
2022年3月現在,研究室には2台のサーバがあります.
総研学生室に設置してある計算機サーバと総研本館1Fに設置してあるworkspaceサーバです.
workspaceサーバに関してはPC係になったらsudo権限をもらってください.
sudo権限を得ると,workspaceへの新規アカウントの追加,subversionの管理(commit権限の付与,Paper20**の作成)などができるようになります.
commit権限の付与は/lab/PC設定/src/any/subversion-add-user.txtを参照してください.
また,Paper20**の作成は/lab/PC設定/src/any/subversion-commands.txtを参照してください.
また,適宜ログインしてセキュリティアップデートがきていないか確認し,あれば適用するようにしてください.
2台のサーバのOSはUbuntuを使用しています.Ubuntuの場合のソフトウェアアップデートのコマンドは以下になります.Tera Term等でログインし,実行してください.
\begin{lstlisting}
	$ sudo apt update && sudo apt upgrade -y
\end{lstlisting}
workspaceサーバは研究室のHPも兼任しています.
研究室のHPについては,管理は関口先生が行っています.
しかし,研究室のHPにあるPublicationタブ内の各論文のDOIなどの各種情報はPC係がエクセルファイルを追記して対応します.
エクセルファイルは,OneDriveの「LAB/Publication」内の3つのファイルです.
LABフォルダへのアクセス権は関口先生に相談してください.
エクセルファイルを追記したら,その変更を関口先生が取り込むので,適宜先生に報告してください.

計算機サーバについてですが,管理を学生に渡されたのでこちらについてもsudo権限をもらい管理してください.
特にアカウントの作成等をやることはないと思いますが,workspaceサーバと同様にセキュリティアップデートの確認と適用をしてください.
ROS(Robot Operating System)が入っているので,ROSの知識についてもある程度持っておくとよいと思います.
アップデートの際には,ROSを使用している学生と相談をして行ってください.
また,計算機サーバの使い方に関しても覚えておいてください.
tera termとXmingというソフトを使って現在は使用方法を確立しています.
詳しい使い方は/lab/PC設定/計算機サーバの使い方.pdfに纏めてあるのでそちらを参照してください.

\subsection{覚えておく知識}
\subsubsection{プリンタの管理}
プリンターに関しては,以下のURLから状態を確認することができます.\\\\
http://(プリンターのIPアドレス)\\\\
トナーの残量レベルが低くなったら交換時期です.
美化備品係と相談をして購入するようにしてください.
交換する際は,色のトナーを野中先生に申告し,注文してもらってください.
数日で届きますがすぐには交換せず,しばらくは強制印刷で凌いでください.
強制印刷ができなくなったら交換をしましょう.

印刷用紙が少なくなってきたら売店で購入してください.お金は必要ありません.
A4の印刷用紙の箱をレジまで持って行き,研究費で購入したいという旨を述べ,
学科名と研究室名を言えば領収書を発行してくれます.
その領収書に自分の名前と学籍番号を明記すれば会計終了です.
領収書を受け取ったら速やかに関口先生に提出してください.
これらの作業は美化備品係と協力してください.
また,印刷ができない状態が長期間続かないようにすること,
トナーや用紙を購入しすぎてしまうことがないように心掛けてください.

\subsubsection{Subversionの使い方}
新4年生の最初の仕事はSubversionと計算機サーバの講習会です.
詳しい方法については/lab/PC設定/Subversionを参照してください.

\subsubsection{研究室における必要最低限のソフトウェア}
新しいパソコンを研究室で運用する場合,必要最低限として以下のソフトウェアをインストールしてください.
\begin{itemize}
\item TexWorks
\item Visual Studio Code
\item Tera Term
\item subversion
\item MATLAB
\end{itemize}
それぞれのインストール方法,注意点などは別途テキストに纏めてあるのでそちらを参照してください.
さらに,LaTeXのインストールでは必ずインストールしたPCがタイプセットを行う際に描画に関する動作が適切であることを確認することをお勧めします.
インストールの方法は方法はPC設定に資料がすべてまとめてあるのでそちらを参照するようにしてください.
PC係はそれらの資料を適宜更新していくようにしてください.
MATLABは高機能機械制御研究室において,必須といえるソフトウェアであるため,インストールは必ずしましょう.その際にインストールするバージョンは先生や先輩方に相談しましょう.
一昨年まではVisual Studio 2017をインストールしてもらっていましたが,MATLABを使用する学生が大半となったため,Visual Studioに関しては必要な際にインストールする形にしてください.
(lab/PC設定/src/事例研究PC係に確認用ファイル有)

\subsubsection{新しい研究室メンバーが配属された際の注意点}
仮配属等で新しく学生が研究室に入ってくる際に,PC係はアカウント作成とソフトウェアのインストールを担当します.
初めに,配属会後に研究室からの連絡を受け取るGmailアカウントとMicrosoft Educationアカウントを登録させます.
続いて,事例研究の初回授業までにソフトウェアのインストールをしてきてもらいます.

次に,事例研究の初回授業ではworkspaceサーバへのアカウント作成とソフトウェアの確認を行います.
その際,新しく入ってくる学生にパスワードを考えてもらいます.
そのとき考えてもらうパスワードは,セキュリティの観点から他のところで使用していないもの,
また個人を特定できないようなものにすることを必ずアナウンスした上で考えさせてください.
また,そのパスワードをメモするようなことはなるべく避けさせてください.
個人的にメモをしておきたい学生にはメモしたものの管理を徹底させてください.
(/lab/PC設定/src/事例研究PC係/新メンバーworkspace登録手順.txtを参照)
また,ソフトウェアの確認は4年生主体でM1とM2のPC係がサポートとして参加します.例年TeXで問題が発生するため,4年生全体に予め手順を確認するよう説明しておいてください.

\subsubsection{OSのアップグレードの注意点}
OSのアップグレード(例:Windows7からWindows10にする)を行う場合,
研究室における必要最低限のソフトウェアの動作が問題なく行えるかを確認した上で行ってください.
個人のPCに関しては自己責任で行うようアナウンスを行ってください.
特に研究室で管理しているパソコン(Excelファイルに纏めているもの)に関してはしっかり確認を行い,PC係主導で行ってください.
また最近はアップグレードを行う場合,必要なデータをネットワーク経由で落とすことが多いのでアップグレードを実行する際は,
スケジュールをしっかり組んで研究室のネットワークに負担がかからないように行うといいです.

\subsubsection{停電時の対応}
総合研究所,世田谷キャンパスが停電となる際には研究室にあるコンセントはスケジュール全て外しておく必要があります.


\subsection{その他連絡事項}
2019年度から研究室が完全に総合研究所に移動しました.
それに伴い,世田谷と総合研究所間での通信はVPNが必要となります.
さらに総合研究所で利用する研究室PCは年度初めにクリーンインストールを予定しています.
そのため,クリーンインストール後に各種ドライバのインストールも必要になります.
基本的にWindowsUpdateを実行すればインストールできると思いますが,ドライバはPCによって異なるのでインストールできなかったときのことを考えて行動しましょう.
また,研究室のネットワーク,配線などについてもPC係は可能な限り把握しておくと,何かあったときの対応がすぐにできるかと思います.頑張ってください.困ったときはM1とM2のPC係に早めに相談しましょう.
% また,2018年度から高機能機械制御研究室のYouTubeのチャンネルができました.
% YouTubeで「tcu acsl」と調べればチャンネルが検索に引っかかります.
% PC係はこのチャンネルの管理者権限を関口先生から付与されるため,チャンネルへの動画のアップロードを適宜行ってください.
% 基本的に,学生からアップロードしたい旨を受けたら,先生に相談しているか尋ね,相談していないようなら相談するように促してください.
% 先生からアップロードの許可を得ているものに関しては,アップロードしてください.
% 研究室のチャンネルとなるため,載せる内容は注意して選定してください.

%Thank you


\subsection{Prime関連}
\subsubsection{2017年度まで}
Primeの保守点検をおこなう係です.
役職に就く人間はPrimeを使用する人の方がいいと思います.
Motiveのバージョンが新しくリリースされた場合アップデートを行ってください.
また,アップデートをした時にパケットクライアントが新しくなると思うので,それを書き換えるようにお願いします.
新しいパケットクライアントはOptitrackのHPにあります.
また係の仕事として10号館5Fのフィールドの使用者を管理する仕事があります.
具体的にはGoogleカレンダーで使用者が書き込むシステムです.
17年度に作製したアカウントがあるのでそれを載せておきます.
ユーザーIDはtcuprime@gmail.comです.
パスワードは「Only北海道」とググれば出てくると思います.

\subsubsection{2018年度}
今年から総合研究所に移動となり,世田谷キャンパスと等々力キャンパスどちらにもPrimeがあります.
総研にあるPrimeの台数が30台なので
役職に就くのは総研でPrimeを使用する人のほうがいいと思います.
この役職では,昨年に引き続きPrimeの保守点検を行います.
また,Primeからのデータを取得するためのプログラム(通称Packet Client)の作成から保守までを行います.
今年からMATLABが個人のPCにインストールできるようになったため,2017年度まで使っていた
Cプログラムで記述されたPacket Clientではなく,MATLAB用のPacket Clientを作成しました.
雛形を作成してあるので基本的にはそれを更新していくようになると思います.
元のプログラムはOptitrackのホームページのNatNetSDKをダウンロードすればわかると思います.
アップデートが入るたびにデータの形式が変わっていないか確認したほうがいいと思います.
わからないことがある場合はOptitrackのwikiを検索すると載っていると思います.

\end{document}

\usepackage{input}
  \newcommand{\test}{waya}
\begin{document}
\maketitle
\section{安全管理係}
実験室管理係は,実験室使用予定や実験室備品等を管理する係です.ここでの実験室は,世田谷キャンパスの研究室5階フィールド,総研の実験室が該当します.


\subsection{実験室保守}
フィールドや計測機器など,実験室の保守管理を行ってください.世田谷と総研に研究室が分かれている都合上,自力で管理できるのはどちらか一つだと思います.もう一方は誰かに依頼しておくのをお勧めします.

総研実験室に引かれたカーペットの配置には意図があります.掃除などで一度片づける際は,実験で使う人に確認を取りながら並べてください.2018年度は,カーペット張り替えは毎月初回の掃除の日に行っていました.美化係と話し合って決定してください.

総研実験室のPrimeの配置は,カーペット同様実験で使う人と相談しながら決めてください.ただ,定期的な配置変更などは行っていません.必要になった時だけ調節するようにしてください.

総研実験室のサーバーラックの中身に関しては,PC係が主に管理していました.PC係がいなくても中身がどうなっているか分かるよう,把握はしておいてください.

総研実験室にある学生用の机周辺はコードが散らかりやすい空間になっています.モニターに使うHDMIケーブルやLANケーブルは,机の下にある赤い箱の中に入っています.使う時だけ取り出し,使い終わったら戻すよう周知し,徹底してください.


\subsection{実験室使用予定管理}
研究室メンバーの実験室使用予定を管理します.予定管理は,研究室行事等のカレンダーと同様,Googleカレンダーで行います.カレンダー名は『実験室使用予定』です.研究室メンバーにカレンダーを共有し,実験室を使う際は各個人で該当する時間に予定を書き込んでもらうシステムです.2018年度は,以下のようにカレンダーへの記入を指定しました.〇〇は使用者の名前です


~~・世田谷5階フィールド→『5階 〇〇』

~~・総研実験室全体→『実験室全体 〇〇』

~~・総研実験室手前のみ(フォースプレート側)→『実験室手前 〇〇』

~~・総研実験室奥側のみ(カーペット側)→『実験室奥 〇〇』


研究室行事が被らない限り,基本的に実験室使用権利は最初に予定を入れた人が優先です.記入方法,予定に関するルールをメールで共有してください.新B4生のアクセス権限追加も忘れずに行ってください.

引き継いだ時点で,カレンダーのアクセス権限には前年度B4,M2の卒業生も含まれています.設定画面からアクセス権限を開き,卒業生のアクセス権限を解除してください.卒業生のメールアドレスはOB会係が把握しています.2019年度はM1の安部に確認し,カレンダーのアクセス権限の欄から卒業生のメールアドレスを探してください.

カレンダー用アカウントのメールアドレスおよびパスワードはここに記入出来ないため,紙に書いて2020年度M1生の松浦に託しました.受け取ってください.

\color{red}{\bf 総研の実験室は高機能研として使用しているものではなく,『インテリジェントロボティクスセンター』としてロボ研と共同で使用しているものになります.使用予定カレンダーに関する連絡は,ロボ研に所属している総研の実験室を使う学生にも必ず共有してください.特に注意しなければいけないのが,高機能研の行事で実験室を使用する場合です.事例研究や総研でのデモなど,高機能研行事カレンダーに書いてあってもそれをロボ研の学生は見ることが出来ません.総研実験室の使用が出来ない日や時間帯は,必ず『〇〇のため総研実験室使用不可』などの予定を使用予定カレンダーに書いてください.}\color{black}

\subsection{Prime管理}
今年から総合研究所に移動となり,世田谷キャンパスと等々力キャンパスどちらにもPrimeがあります.Primeの保守点検を行ってください.また,Primeからのデータを取得するためのプログラム(通称Packet Client)の作成から保守までを行います.2018年度からMATLABが個人のPCにインストールできるようになったため,2017年度まで使っていたCプログラムで記述されたPacket Clientではなく,MATLAB用のPacket Clientを作成しました.雛形を作成してあるので基本的にはそれを更新していってください.元のプログラムはOptitrackのホームページのNatNetSDKをダウンロードすればわかると思います.アップデートが入るたびにデータの形式が変わっていないか確認してください.

不明な点がある場合はOptitrackのwikiを検索してみてください.


\subsection{その他連絡事項}
(2019/3/1版)

(1)総研実験室において,Primeのマーカーの管理に関する管理方法が曖昧なままでした.車椅子やUAV,障害物やフラフープなど,固定ではなく張り替えたりして使用しているため,使い終わっても付けっぱなしであることがありました.稀に机に放置されていることもあったため,何か管理方法は確立した方がいいと思います.

(2)実験室予定に名前があるにも関わらず,その時間になっても使用者が研究室に来ていないということが何度かありました.その間実験室の利用が出来なくなってしまうため,その点にも何かルール(何分来なかったらその日の予定はカレンダーから削除し優先権は他の希望者に譲渡など)を決めた方がいいかもしれません.実験室を使っていなくても研究室にいるような場合は直接理由を聞けるので問題はないと思います.

\end{document}

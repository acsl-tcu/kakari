\usepackage{input}
  \newcommand{\test}{waya}
\begin{document}
\maketitle
\section{Prime係}
\subsection{2017年度まで}
Primeの保守点検をおこなう係です.
役職に就く人間はPrimeを使用する人の方がいいと思います.
Motiveのバージョンが新しくリリースされた場合アップデートを行ってください.
また,アップデートをした時にパケットクライアントが新しくなると思うので,それを書き換えるようにお願いします.
新しいパケットクライアントはOptitrackのHPにあります.
また係の仕事として10号館5Fのフィールドの使用者を管理する仕事があります.
具体的にはGoogleカレンダーで使用者が書き込むシステムです.
17年度に作製したアカウントがあるのでそれを載せておきます.
ユーザーIDはtcuprime@gmail.comです.
パスワードは「Only北海道」とググれば出てくると思います.

\subsection{2018年度}
今年から総合研究所に移動となり,世田谷キャンパスと等々力キャンパスどちらにもPrimeがあります.
総研にあるPrimeの台数が30台なので
役職に就くのは総研でPrimeを使用する人のほうがいいと思います.
この役職では,昨年に引き続きPrimeの保守点検を行います.
また,Primeからのデータを取得するためのプログラム(通称Packet Client)の作成から保守までを行います.
今年からMATLABが個人のPCにインストールできるようになったため,2017年度まで使っていた
Cプログラムで記述されたPacket Clientではなく,MATLAB用のPacket Clientを作成しました.
雛形を作成してあるので基本的にはそれを更新していくようになると思います.
元のプログラムはOptitrackのホームページのNatNetSDKをダウンロードすればわかると思います.
アップデートが入るたびにデータの形式が変わっていないか確認したほうがいいと思います.
わからないことがある場合はOptitrackのwikiを検索すると載っていると思います.

\end{document}

\usepackage{input}
  \newcommand{\test}{waya}
\begin{document}
\maketitle
\section{メッセージ係}
\subsection{MESSAGEについて}
\begin{center}
MESSAGE\\
\textcolor{red}{ME}chanical \textcolor{red}{S}ystems \textcolor{red}{S}ymposium by all \textcolor{red}{AGE}s\\
\end{center}

MESSAGEとは,機械システム工学科が主催するシンポジウムの事である.
このシンポジウムは卒業生と在学生の交流を目的に開催されており,機械システム工学科を卒業し,様々な職場で活躍されているOB,OGの方がご来校下さります.
在学生が卒業生の活躍を知ることで,目的意識をもって学習に取り組むことができるようになります.
卒業生による講演だけではなく,開催年度に就職活動を体験し,社会人としての第一歩を踏み出そうとしている在学生による講演も行われます.
本研究室においても,多くの卒業生,先輩方が講演をなさってこられました.

このMESSAGEは各研究室のM1,M0の学生計12名と幹事研究室の教授がMESSAGE実行委員となり行われます.
``MESSAGE係''とは,高機能機械制御研究室を代表しMESSAGE及びCAP\,(Campus Amusement Park)の運営をする係です.
※ただし,研究室展示のことをCAPと呼ぶと一部の先生にとって違和感になるようなので注意.

\subsection{主な仕事内容}
MESSAGE係の研究室での主な仕事内容は,研究室展示の内容の検討,卒業生への周知などがあります.
また講演していただく卒業生は,野中先生と相談して,お願いをします.

MESSAGE実行委員会ではそれぞれの研究室に役職が割り振られ,担当する役職の役割を果たします.
実行委員会は夏季休業前に顔合わせを行い,話し合いを進めながら各研究室で本番へ向け準備を進めていきます.
細かい仕事内容については年度ごとに変化します.
基本的に実行委員長の先生が招集をかけ,話し合いを行いますが,先生によってはこの企画自体に積極的ではないので,自分たちで動かなくてはいけない.

\subsubsection{プログラム係}

プログラム係は,OBの方へ告知をする際に添付する大まかなプログラムと,当日どの研究室がいつどこでどんな仕事をするといった詳細シフトを作製します.詳細シフトは直前になっても大丈夫ですが,OBの方へ送るプログラムは早めにできていないとまずいです.プログラム待ちで連絡できませんといった状況は避けましょう.

\subsection{注意点}
MESSAGE開催までに注意することとして,MESSAGEに来ていただける卒業生は今まさに社会で活躍されている方々です.そのためMESSAGEに関する情報の周知が遅い場合,大変迷惑がかかります.なので,卒業生の方々への周知は早めにする必要があります.また,MESSAGEの具体的な内容の決定が遅くなると各方面に大変迷惑がかかります.なので,早めにMESSAGEで行う内容を決定する必要があります.理想的なスケジュールとしては,7月中に企画内容やテーマを決めてお盆までには出演依頼のメールをOBの方に送れているといいです.あと,その際の注意点では,連絡先を知らなくても野中先生が知っていることが多いので野中先生とよく話をしましょう.

また,MESSAGE当日は世田谷キャンパス学園祭の一日目に開催されるため,研究室の出店などへの参加が難しくなります.
そのため会場の受付など,研究室のメンバーに仕事をお願いする必要があるため,学園祭係との打ち合わせを行い,シフトを作成する必要があります.

\subsection{過去企画}

\subsubsection{基調講演}

ほとんど毎年実施.OBの方に講演をしていただく企画です.話してほしいテーマと時間をお伝えしてスライドを作ってきていただき,講演していただきます.2016年度は社会に出てから10年以上たつベテランの人にお話をしていただくというコンセプトでやりました.中堅1人,若手1人の構成で,僕たちはもちろんOBの方にも世代の異なる人の意見に触れるいい機会にしていただこうという感じでした.

\subsubsection{パネルディスカッション}

2016年度実施.各研究室から1人ずつパネラーのOBの方を出して,壇上でディスカッションしていただく企画.自己紹介用として2ページ程度のスライドテンプレを渡してあらかじめ作ってきていただいて,それをもとに企画が始まる.ちなみに2016年度は司会は幹事研究室の先生.その後はこちらからパネラーの方々に聞きたい質問をアンケートをもとに用意しておいて,それに対して喋っていただく.注意点としては,司会が聴衆からの質問を頻繁に受け付けないといまいち盛り上がらない.あと,ぶっちゃける先輩がいるとすごい盛り上がる.

\subsubsection{ブース}

2015年度に実施.各OBさんが自分の企業のブースを作ってそこにみんながお話をしに行くというもの.講演やディスカッションと違い近い距離で話ができる,自由な質問ができるというのが売り.ただし,企業の人気不人気はどうしても存在するのでブースごとに人の粗密ができてしまう.あとは1教室内で複数ブースがあると音が混じってうるさいらしい.なので,この2点に関しては要工夫.

\subsubsection{懇親会}

毎年実施.メモリアルホールでケータリングを用意して懇親会をする.ただし,学祭では教室内飲食禁止なのでメモリアルホールが取れなかったときは悲惨なことになる.必ず早めに抑えておくこと.予め担当研究室の指揮の下,会場レイアウトとケータリング搬入を済ませておくこと.基本的にMESSAGEのメインイベント.だが,各研究室でも夜に各々で飲み会を企画しているので,夕方ごろ終了になるのが望ましい.結局,それぞれの研究室の同窓会みたいになる上,かなり前のOBは先生くらいしか話せる人がいないのでそのあたりのケアが課題としてある.

\subsection{$2020$年度の活動}
コロナウイルスの影響により世田谷祭が中止となり,MESSAGEも中止となった.そのため,$2020$年度は準備等含め活動していない.MESSAGEについては関口先生から連絡があると前任者から伺いました.

\subsection{$2021$年度の活動}
本年はコロナの感染拡大状況を鑑み,初のオンライン開催となった.以下に当日までの流れを順に示す.

\subsubsection{顔合わせ~事前準備}
OB・OG招待まで \\
本年は熱流体システム研究室の永野先生が主催を務めていた.主催研究室の先生との連絡は基本メール,または先生により開催されたzoomでのミーティング時に行った.コロナ禍であったこともあり顔合わせが例年に比べて大分遅く,初回zoomミーティングは8月の終盤であった.

本年は例年と比べて研究室間の連携が幾分取りづらいものとして,OB・OGに送付する招待状やパネルディスカッション等への参加依頼文はすべて永野先生が用意してくださり,各研究室ごとに連絡代表者や研究室名を変更するという形であった.そのため,学生側の主な仕事はフォーマットが届くまでにパネルディスカッションを依頼するOB・OG(1名)と,グループディスカッションを依頼するOB・OG(2~3名)について,前年度の先輩を交えて候補者を決め,野中先生と関口先生に確認を取ることであった.

文面のフォーマットが届いたのは9月の初週終わりくらいであり,各研究室で文面を整えPDF化してメールに添付する形での送付となった.尚,この際の招待はOB・OGに対して一括で行うため,事前に野中先生乃至は関口先生に送付を依頼しておく必要がある.(招待を送付するのは先生でしたが,以降のプログラム送付やOB・OGの方とのやり取りは学生側で行います.)

最終的な招待は9月中旬ごろに行い,返信の〆切としては10月初週を提示していた.(例年はそもそも夏休み前から準備が始まっているものらしいので,この期間に関しては本年のは参考にしすぎない方がいいかもです.きちんと当代の主催の先生に確認を取ってください.)

その後,講演について了承してくださったOB・OGの方には改めてお礼を述べた上,当日のプログラム送付と合わせてグループディスカッション用のスライド作成を依頼した.(枚数の指定は必要ありませんが,予めどういう話について纏めてほしいかは提示しましょう.ちなみに本年は企業説明や紹介等ではなくOB・OGの皆様の経験談を軸に企業に勤めてからのお話をして頂きました.)

※招待状文面に関しては多分に主催担当の先生の裁量に左右されると思われるため,次年度から例年通り文面も学生側で考案する必要があることも念頭に置いて準備を進めた方がいい. \\

聴講側のOB・OGの招待 \\
10月下旬半ば頃に.永野先生から講演する側ではなく聴講者として参加するOB・OGを招待してほしいと連絡が来た.招待の送付を研究室の教授・准教授がするか,または学生がするかは各研究室内で決めるようにとのことで,本年は関口先生に卒業生の連絡先一覧を頂いて学生側で送付する形を取った.この際も招待文のフォーマットは永野先生が用意してくれていたので,学生側は貰った連絡先をBCCに追加し続けるのと文面に少し手を加えるくらいしか仕事は無かった.(Toに自分の学番メールを設定すると,不具合なく届いたか確認しやすくて便利です.) \\

アンケートの送付 \\
10月最終日,開催一週間前にして永野先生からOB・OGへのアンケートが届く.この際もやはり依頼文は予め用意してくれたので,研究室用にちょこちょこ直して送付という形になった.(事前にこういう追加をしてくる先生がみんな依頼文を用意してくれるとは限らないので,不安な場合は自分から率先して主催の先生に聞いておいた方が良いです.面倒でも連絡はこまめに取りましょう.) \\

オンライン開催に向けた準備 \\
本年は高機能研関口班の朝ミーティングでも使っているSpatialChatにて開催するとのことで,各研究室は上記のアンケート送付の時期と同じくしてチャットルームの背景に使用する研究室写真の提供を求められた.本年は学生室・実験室の写真を撮って永野先生に送付したが,過去の世田谷祭における出店の写真とかでもいいらしい.(ただし,あんまりはっきりと関係ない人の顔が写ってしまっているものは避けるべき.) \\

質疑考案 \\
パネルディスカッションに関しては,当日ももちろん参加する在学生に質問してもらうことになるが,質問が途切れないとは限らないので予めいくつかは担当の学生間で質問を用意しておく必要がある.この際,他の研究室の学生と質問が被るのも避けなければいけないため,事前に割り振られたペアとなる研究室の担当者と質問内容を共有しておいた方が良い.(今年はロボ研とペアでした.)

\subsubsection{当日の流れ}
当日は例年通り世田谷祭の一日目に開催した.開始直前にホストの渡し方が行き届いていなかったり,一度に部屋に入れる人数に制限があったりと若干の手間取りがあったものの,概ねマイク等の機器故障もなく開始.永野先生の司会進行に沿ってパネルディスカッションを終えた.質問は,用意していたものから聞いたものも幾つかはあったが,関口先生曰くオンラインでカメラも無いせいか例年よりみんな積極的だったと好評だった.

その後,午後の部までの昼休みの間にMESSAGE担当学生と永野先生とで簡単なミーティングをし,午後の部開始時の全体連絡と要所要所のタイムキーパーを永野先生に依頼した.

午後からは各研究室用に作られたzoomで言うところのブレイクアウトルームのような小部屋で待機し,永野先生の全体放送が終わり次第各研究室ごとのグループディスカッションとなった.この際の司会進行は学生が務めることになるため,段取りは事前によくよく見直した方が良い.

また,研究室内でのタイムキーパーはMESSAGE担当学生側で担うことになるため,スライドによる発表をお願いした後の質疑応答時には時間配分に気を配る必要がある.

\subsubsection{その他}
主催の先生が新しいツールを持ってきたときには,細部の仕組みまで掴み切れていない場合もある.こういうのを使うと話が出たら,一通り自分でも操作しておいた方が当日不備があった時に補助しやすいのでおすすめ.
\end{document}

\usepackage{input}
  \newcommand{\test}{waya}
\begin{document}
\maketitle
\section{副室長}
%%%%%%%%%%%%%%%%%%%%%%%%%%%%%%%%%%%%%

本章では副室長が行っていた業務を記しています.

\subsection{全体ミーティングの議事録作成}
	毎週行われる全体ミーティングは,
	室長が司会となり副室長は議事録の作成を行います.
	記載する内容は
	\begin{itemize}
		\item ミーティング日
		\item 「誰」が「何」を話したか
	\end{itemize} 
	です.
	
	細かく記載する必要はなく,
	後で他の研究室メンバーが確認してわかる程度に箇条書きしましょう.
	作成した議事録はファイル形式をPDFにし,
	研究室Slakのacsl\_genelanチャンネルに投稿します.

	全体ミーティングは毎週行われるため事前にフォーマットを作成しておくことをお勧めします.
	25年度はWordで大まかに形を作成しておき,
	ミーティングの度に日付などを変更して使用していました.
	議事録の書き方の指定は特にありませんでした.
	困ったら過去の議事録がacsl\_genelanチャンネルにあるので参考にしてください.

\subsection{室長不在時の代打}
	全体ミーティング日に室長が不在のときには代わりに副室長が司会を行います.
	このとき議事録は後で作成するか,
	他の人に頼んで作成してもらいましょう.

\subsection{MSE Hubの予約}
	研究活動をしているとMSE Hubを使用することが多くあります.
	MSE Hubは他研究室の人も利用するため予約システムがありますが,
	予約システムは先生と室長・副室長のみが使用できるようになっています(25年度現在).
	発表練習やミーティングなどで使うことになったら予約をしましょう.
	また研究や就活の一環で個人的に使用したい人もいるため,
	相談されたら代わりに予約をしてあげましょう.

\subsection{過去の業務}
上の3項目が25年度の主な副室長の業務でした.
年度によって必要な業務に変化が生じる可能性があるので過去の副室長の業務を記載します.
\begin{itemize}
	\item 共有カレンダーの管理
	\item 勉強会のスケジュール管理
\end{itemize}
25年度に関しては,
共有カレンダーは各自が管理し,
勉強会は各担当の先生ごとに開催されていました.

\subsection{最後に}
	各年度によって特色が変わると思います.
	ここまでに記した以外に研究室としてやらなくてはいけないことも変わると思います.
	そのときには室長とよく話し合って役割分担をしてください.
	頑張ってください!

\end{document}

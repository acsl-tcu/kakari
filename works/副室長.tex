\usepackage{input}
  \newcommand{\test}{waya}
\begin{document}
\maketitle
\section{副室長}
%%%%%%%%%%%%%%%%%%%%%%%%%%%%%%%%%%%%%

本章では2020年度の副室長が行っていた業務を記しています.
リモート体制にになったり,ロボ研と合同になったりしたので,数点例年と異なります.
\begin{description}
 \item[~室長の補佐~]\mbox{}\\
	    副室長のメインとなる業務です.主な内容としては
	\begin{itemize}
		\item 全体ミーティングのとき室長が司会進行なので内容をメモる
		\item 室長の仕事を手伝う
		\item 室長不在の時には室長の仕事を行う(ミーティングの司会,夜間申請の提出など)
	\end{itemize}
です.

室長は仕事が沢山あります.また,イレギュラーな仕事も沢山出てきます.
しっかり連絡取り合って,出来る限りのサポートしましょう.
そのためにも,室長の仕事を把握する必要があるので,この引継ぎ資料の室長の項目も読んでおきましょう.
スラックでは野中先生関口先生室長副室長,関口先生室長副室長のグループを作って業務連絡などしていました.
その方が室長が副室長に仕事の説明をする手間が省けます.

全体ミーティングのメモは,ミーティング後に室長がGmailで研究室全体に全体ミーティングの内容を送るときに使って貰っていました.タイピングなどに慣れるまでは,メモ役はもう1人募ってもいいかもしれません.


\item[~カレンダーの管理~]\mbox{}\\
各自で記入してくれたので,特にこれといったことはしていませんでした.
気づいたときに書いていない行事が無いかの確認くらいでした.
室長副室長は,様々な仕事で先生方とコンタクト取る必要があったので,野中先生関口先生のGoogleカレンダーの予定(tcu関連のお仕事)も特別に共有して頂きました.


\item[~勉強会のスケジュール管理~]\mbox{}\\
例年は,院ゼミでは院生が学部ゼミでは学部生が発表していましたが2020年度はロボ研と合同になったため院生のみ発表でした.名前も院ゼミから勉強会に代わりました.
スケジュールは基本1人で組んでいましたが,月末に開催する場合は報告会の週と被らないように発表係と一緒に開催日を組んでいました.

 \begin{description}
  \item[勉強会は1週間に1回]\mbox{}\\
 その年度の研究室の人数で変わりますが基本的には1週間に1回2人ずつ発表することを意識してスケジュールを立てましょう.また,例年は11月中に終了が目処でしたが,2020年度から勉強会の終了時期は夏休み前がマストになりました.これが決まったのが7月始め辺りだったので,夏休みに食い込み,8月中旬に終わりました.2021年度以降からも同様の体制ならば7月中に終わらせましょう.
 開催時期は4月はm1は研究が固まっていない,m2は就活関連であまり時間がないなどがあり5月初週から始めました.

 また,勉強会より発表会報告会が優先であるため発表会係と連絡を取り,早めに決めていく必要があります.
 そのため,発表会係に早めに決めてもらうように促すようにすることも大事です.
 目安としては,勉強会の発表2,3週間前には発表者に伝える必要があるので3週間前には日程が決まっているとベストです.発表者はスライド作成やプログラム作成などがあるので.

 研究室は多くの行事があるため,先延ばしにすると休日や長期休みに開催なんてこともあり得ます.
 月にゼミを何度行うか決めたら守るように頑張りましょう.


 \item[勉強会の発表順番]\mbox{}\\
 4年の序盤は研究室に関する知識等がないので,勉強会の発表順番は先輩方に決めてもらいました.院生全員と副室長をメンバーにした勉強会用のスラックのチャンネルを作成し,そこで基本先輩方で動いてもらいました.
 ここのグループで副室長が動いていたことは,発表順番以外に関する連絡です.
 2021年度からスラックにロボ研が加わったのでロボ研の方もチャンネルに入れると便利かもしれません.

 発表順番は,高機能とロボ研が交互に発表していました.勉強会は高機能のみのイベントかつ高機能の方が院生の人数が多かったため序盤は高機能が連続して行い,残りの人数がロボ研と同じになったら交互にしていました.ここは年度によって臨機応変にして下さい.

  \item[他行事との調整]\mbox{}\\
   研究室の行事は報告会やPC(Presentation Competition)など色々あります.研究の発表会や学会の発表練習などを行う週には勉強会を入れないようにしましょう.発表会係や先輩と話し合いながら調整して下さい.調整した日で開催可能か最終確認として先生方とロボ研に確認取ってください.佐藤先生にはロボ研の人に確認を取って貰っていました.

 \item[勉強会を行える日時を把握する]\mbox{}\\
   先生は授業や会議などで忙しいです.授業の曜日は決まっているため,自然とゼミを行える曜日は決まっていきます.基本的に行える曜日・時間を把握しておくと日程が組みやすいです.また2020年度は2019年度までと異なりリモートだったのでゼミの日程はとりにくいとは感じませんでした.\\
   ちなみに,2020年度は前期後期共に基本木曜日の3限に行っていました.

 \item[担当されていたゼミ当日に風邪や体調不良で欠席した人が出た場合]\mbox{}\\
   担当していたゼミ当日に風邪や体調不良などの原因によって欠席される人もいます.その場合は代わりにどの日でやるのかということを話し合って,早いうちにスケジュールに反映させましょう.
 \end{description}


研究室内の人とマメにコミュニケーションをとって,「いつ」「誰が」「何を」行うのか把握しておくとスケジュールが組みやすいです.
多くの先輩が学会に参加したり授業があったりするので,日程が決まったら早めに連絡しましょう.

その際に先輩から助言を頂くのも大いにありだと思います.
\end{description}

2020年度の副室長の業務内容の説明は以上です.

色んな人と連絡取りあって情報共有をしましょう.
リモート体制なら週に1回でも4年全体でzoomをしてもいいかもしれないです.
2020年度はこれをしなくて特定の人にだけ仕事が多くなるなんてことがありました.

また,ここに記した以外の仕事もよく入ってきます.
先輩や同期など色んな人と連絡取って上手くこなしてください.特に,室長とはよく連絡取り合ってください.








**以下のものは対面で活動できた2019年度のものです.**

\begin{description}
 \item[~室長の補佐~]\mbox{}\\
	    副室長のメインとなる業務です.主な内容としては
	\begin{itemize}
		\item ミーティングのときは室長と一緒に真ん中に立ちメモを取る
		\item ミーティングの内容を室長と確認
		\item 室長不在の時には室長の仕事を行う(ミーティングの司会,夜間申請の提出など)
	\end{itemize}
です.
室長の仕事は多いので時と場合に応じてサポートや代理を務められるようにしておいて下さい.
そのためには室長の業務を把握している必要があります.
副室長に任命された方は係引き継ぎの室長の欄を確認しておいて下さい.
% また2018年度から室長と副室長は世田谷と総研で別れました.
% 副室長はどちらかの研究室において一部業務が室長と同じ扱いになるので注意しておいてください.\\
% (例)2018年度では室長は総研,副室長は世田谷にいたので室長の仕事で野中先生と日程調整が必要なものは副室長が野中先生と日程調整などを行っていました.また夜間申請に関しては室長が世田谷にいないので副室長が毎回申請を行っていました.
また,2019年度は室長が総研,副室長が世田谷の夜間申請の書類を作成していました.
来年度は副室長が修士にいるので去年の副室長から夜間申請の書類を引き継いでください.
 \item[~カレンダーの管理~]\mbox{}\\
	研究室で共有しているGoogleカレンダーの管理です.
学会や休暇表は各自で記入しているため,書いていない行事が無いかの確認くらいで十分だと思います.
 \item[~院・学部ゼミのスケジュール管理~]\mbox{}\\
院・学部ゼミの内容自体については研究室オリエンテーションで説明されるため,割愛させていただきます.\\
ゼミの資料とかに関しては副室長が用意,または購入してもらうものを周知するために年度初めに関口先生に何が必要か聞いておきましょう.
2019度の院ゼミ・学部ゼミは副室長がスケジュールの管理を行っていました.
研究室内外での行事とかぶらないように調整が必要となるため副室長の業務の中でも最も大変な業務だと思います.
昨年度の副室長がゼミのスケジュールを決める上で気を付けた点をいくつか示しておきます.
	\begin{description}
	 \item[ゼミは1週間に1回]\mbox{}\\
  研究室内の人数で変わりますが基本的には1週間に1回2人ずつ発表するようにすることを意識してスケジュールを立てましょう.
  % この原因として,一つ目に夏休み中に連絡をとり,長期休み明けにゼミを開けなかったこと.2つ目に2018年度は年に一人一回発表だったため楽観視してしまい,前期中にゼミを入れられなかった点があげられます.

  夏休み中に連絡を取りづらい方もいるかもしれないですが,きちんと先生や発表者と連絡を取り休み明けにはすぐにゼミを始められるようにしましょう.また春のオリエンテーションでゼミの回数が説明されると思いますが1,2月は卒論,修論の関係上忙しくなるので,通常で11月中,遅くても12月初旬にゼミを終わらせる計画を立てましょう.また前期中には就活やpcなどで忙しい人がたくさんいますが後期のほうが学会とかの関係上ゼミを開くのが難しくなっていくのでなるべく前期中にゼミの日程は詰め込みましょう.

また,ゼミより発表会報告会が優先であるため発表会係と連絡を取り,早めに決めていく必要があります.
そのため,発表会係に早めに決めてもらうように促すようにすることも大事です.
目安としては,ゼミの発表2週間前には発表者に伝える必要があるので3週間前には日程は確定しているようにしましょう.
研究室はいろいろと行事があるため,「今月は忙しいから来月に回せばいいや」と考えていると「土曜日や夏休みを使って一日の間に6人発表」なんてこともあり得ます.
月にゼミを何度行うか決めたら絶対に守るように頑張りましょう.
	 \item[他行事との調整]\mbox{}\\
		研究室の行事は報告会やpcなど色々あります.研究の報告会や発表会を行う週にはゼミを入れないようにしましょう.発表会係や先輩と話し合いながら調整して下さい.
	\item[ゼミを行える曜日・時間を把握する]\mbox{}\\
		先生は授業や会議などで忙しいです.授業の曜日は決まっているため,自然とゼミを行える曜日は決まっていきます.基本的に行える曜日・時間を把握しておくと日程が組みやすいです.また2019年度は野中先生が学科主任と総研の所長をやっていたためゼミの日程がとりにくい状況でした.なので毎週何曜日の何時限目に行いたいですというのを4月中に野中先生と関口先生に許可を取って決めておいたほうが良いかもしれません.\\
    ちなみに,昨年度は前期後期共に木曜日に行っていました.
  \item[担当されていたゼミ当日に風邪や体調不良で欠席した人が出た場合]\mbox{}\\
    担当していたゼミ当日に風邪や体調不良などの原因によって欠席される人もいます.その場合は代わりにどの日でやるのかということを話し合って,早いうちにスケジュールに反映させましょう.2019年度はこの部分を徹底しなかったせいで1月までゼミが伸びたので,特に気を付けて下さい.
	\end{description}
研究室内の人とマメにコミュニケーションをとって,「いつ」「誰が」「何を」行うのか把握しておくとスケジュールが組みやすいです.
特に多くの先輩が学会に参加しているので,日程が決まったら早めに連絡し先輩同士で話し合ってもらうこともあります.
その際に先輩から助言を頂くのも大いにありだと思います.
\end{description}
長くなりましたが,副室長の業務内容の説明は以上です.
副室長の役職は出来て数年のため,業務内容が他の係と分担・統合されるなど,大きく変更することが考えられます.
与えられた業務をよく確認して,室長のサポートをしつつ,時には室長にサポートしてもらいながら頑張って下さい.
\end{document}

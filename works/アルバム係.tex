\usepackage{input}
  \newcommand{\test}{waya}
\begin{document}
\maketitle
\section{アルバム係}
アルバム係は飲み会などの研究室の写真撮影や卒業アルバムの製作などをする係です.
アルバム委員に選ばれたり,選ばれなかったりで仕事の量が違いますが基本は以下の通りです.
2020年度では活動がオンラインになったためその点について最後に追加で記載しておきます.
\begin{itemize}
\item 発表会・報告会などでの動画を撮影します.撮影には関口先生の私物であるビデオカメラを使用してください.ただし,2019年3月2日現在バッテリーを用いた撮影は出来ないので,必ず電源ケーブルを用いて撮影してください.教室によってはコンセントが少なく,撮影場所から遠いこともあるのでその場合は延長コードを使ってください.また,ビデオカメラはアルバム係が管理することになりますが,総研で管理する場合発表会などの前日は一度家に持ち帰ることをお勧めします.発表会が午前からだと総研に寄る時間があまりないためです.午後からの場合,午前中に一度総研によるのであれば持ち帰る必要はないと思います.
\item 撮影に関してですが基本的に三脚を用いてカメラを固定します.世田谷と総研に1つずつミニ三脚があるのでそれを用いてください.発表中はスライドと発表者が映るように撮影します.ただし,カメラの画角が狭くこれらを全て映そうとすると,教室の大部分のスペースを占領してしまう形になるのでその場合はスライドのみの撮影でも構いません.撮影は発表係のアナウンス後に開始し,質問後に発表者があいさつしたところで撮影を終了します.
\item 発表会・報告会以外にもゼミや修論公聴会,学会の発表練習などがあるのでこちらも忘れずに撮影しましょう.修論公聴会は普段用いるミニ三脚ではなく,世田谷にある大きい三脚を用いて教室の後方で撮影します.
\item 動画撮影後,一人ごとに編集しworkspace/MoviePhoto/Hxx(xx:年度)のフォルダ内に保存します.動画形式は問いませんが,そのままだと重いので軽くしてください.参考に動画の圧縮ソフトとして「Aiseesoft フリー動画変換」があります.これを用いることで容量を削減することができますが,windowsの「フォト」などでトリミングが必要となります.
\item workspace内の容量を節約するため,2ヶ月で保存している発表会・報告会の動画は削除してください.
\item 研究室でのイベントの写真撮影‐卒業アルバムの研究室ページの作製に使います.まんべんなくとりましょう.また,研究室のメンバーを過不足なくとるようにしましょう.
\item 節目のタイミングで研究室の集合写真の撮影を行います.まず,新年度になってからホームページ掲載用に研究室の集合写真を撮影します.1号館の教室で撮ることが多いと思いますが,特に場所の決まりはありません.次にその年度の最初の事例研究で集合写真の撮影があります.全体の集合写真とB3生のみの集合写真を撮影します.B3生の集合写真は撮影後B3生の名前(ふりがな込み)を入れたものを4部印刷します.1つは野中先生用,1つは関口先生用,残り2部は世田谷用,総研用となります.最後は卒論公聴会です.卒論公聴会後に全体の集合写真の撮影をします.
\item 研究室用とは別に卒業アルバム用の研究室集合写真の撮影を行います.秋ごろにアルバム委員に研究室の集合写真の日時と場所を決めてくださいと言われるので,先生の日程を確認し日時を決定してください.日時と場所は学支で予約します.撮影可能な日時が決まっているので予約開始日に一度学支で撮影可能な日時を確認しておくと良いと思います.場所はどこでもいいです.自分がここだと思ったところにしてください.当日はカメラマンが撮影します.ちなみに2017年と2018年は3号館の中庭で撮影をしました.2018年は報告会の途中での撮影で当初3号館前での撮影の予定でしたが,時間帯的に日差しがまぶしかったため中庭での撮影に変更となりました.
\item 卒業アルバムの研究室ページの作製‐12月頃に今までとった写真をつかって1ページの研究室紹介ページを作成します.4年生中心で4年全員と野中先生,関口先生,鈴木先生が乗るようにしてください.2018年はパワポで2017年は5階研究室のレクザに入ってるphotshotを使って作成しました.ちなみに2018年の卒業アルバムはworkspaceのMoviephoto->H30->卒業アルバムにあるので参考にしてください.
\item あとは卒業アルバム委員に言われたことを適宜行ってください.
\item 使用するカメラについては特に決まりはありません.スマートフォンのカメラでも十分だと思います.もし,一眼レフカメラなどを研究室で持っている人がいればその方に集合写真などの撮影をお願いするのも1つの手です.
\item 最後に飲み会にカメラを忘れない,写真を撮ることを忘れないようにしてください.
\item ここからオンラインでの仕事内容について記します.基本的には上記の仕事内容に変更はありません.主に変更となったのはにオンライン報告会でしたので,変更となった点について記します.オンライン報告会はZOOMで行い,録画は自分のPCで行うようになりました.報告会が始まる前にホストとなる先生に録画の許可を自分のPCに貰ってください.その後,各発表者の題目を読む前からスタートし,質疑応答の終わりで停止します.この際の注意として録画中はマークが表示されるため確認すること.また,停止は一時停止ではなく停止を行ってください.その方が編集する必要が無くなり後々楽になります.報告会が終わりZOOMを抜けると自動で動画の変換が行われます.この際自分のPCの容量が少ないと変換が行われないため空き容量には気を付けてください.動画が出来ましたら,各動画に日付や報告会名,発表者名等を付けます.そして動画をアップするという流れになります.アップする場所なのですが通常の仕事内容で記載されている場所に加え,Microsoft Stream,URL[https://www.microsoft.com/ja-jp/microsoft-365/microsoft-stream]にアップしてください.Microsoft Stream のチャンネルは,高機能・RLS2020のメンバーのみアクセスできるため,サインインする際は,g学籍番号@w.tcu.ac.jpのアカウントでサインインをお願い致します.この中で動画をMicrosoft Streamの報告会.発表会動画というチャンネルにアップロードしてください.チャンネルが無ければ作成しましょう.以上がオンライン報告会での仕事内容になります.
\end{itemize}
\end{document}

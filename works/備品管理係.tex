\usepackage{input}	
  \newcommand{\test}{waya}
\begin{document}
\maketitle
\section{備品管理係}
\subsection{蛍光ランプの交換}
管理係は,研究室内の切れた蛍光ランプの交換を行う.
%%
\subsubsection{蛍光ランプの交換手順}
世田谷での蛍光ランプの交換手順は以下の通りである.
\begin{enumerate}
\item 脚立等を使い,蛍光ランプを取り外す
\item 取り外した蛍光ランプを1号館の施設管理課へ持っていく
\item 切れた蛍光ランプを持っていくと施設管理課から切れた本数分の蛍光ランプがもらえるので,それを受け取る
\item 脚立等を使い,蛍光ランプを取り付ける
\end{enumerate}

総研での蛍光ランプの交換手順は以下の通りである.
\begin{enumerate}
\item 脚立等を使い,蛍光ランプを取り外す
\item 取り外した蛍光ランプを総研本館の事務室へ持っていく
\item 切れた蛍光ランプを持っていくと事務室から持って行った本数分の蛍光ランプがもらえるので,それを受け取る
\item 脚立等を使い,蛍光ランプを取り付ける
\end{enumerate}
総研の脚立を片付けるときは,指を挟まないように注意する.寝かせて片付けるのが良い.
%%
\subsubsection{蛍光ランプの取り外しと取り付け}
研究室の蛍光ランプが取り付けられている照明器具は,電池を外すようにランプを片側に寄せてから下方向に引いて外す.取り付け時は逆の手順で行う.総研実験室の蛍光ランプにはカバーがついている.真ん中にある引っかかりを押しながら外す.

\subsubsection{その他の留意事項}
総研では2019年度に蛍光灯の取り換えを多く行ったが,2020年度以降も取り換えが頻繁に起こる可能性があるため注意すること.世田谷の研究室では,4階の照明はLEDなので交換の必要は無く,交換が必要なのは5階のみである.また研究室の天井は高いため,一人で交換を行うと蛍光ランプを外したり取り付けたりする際にランプを持ったまま脚立を昇降しなければならず,非効率な上にやや危険である.よって,ランプの交換は2人で行う方が良い.その際,一人が脚立に登って交換作業をし,もう一人が脚立の下で外した蛍光ランプの受け取りや新しい蛍光ランプの受け渡しを行うと効率的である.


\subsection{備品の調達}
研究室で使用しているホワイトボードマーカーや養生テープなど,研究室で必要となる備品を購入する.総研関係の備品,世田谷関係の備品共に購入の際は事前に購入希望物品とその理由を野中先生か関口先生に伝える(可能であれば双方に伝える).基本的に月に一回まとめて購入する.調達時の支払方法は仮納品書と立て替え払いの2種類がある.

\subsubsection{経費での購入可能品目}
以下に代表的な研究室運営費で購入できる備品を示す.

\begin{enumerate}
\item ゴミ袋(45L以上)
\item ホワイトボードマーカーの替えインク
\item 印刷用コピー用紙
\item プリンタートナー(純正品)
\item 消毒用アルコール(衛生維持のため)
\item 清掃用スポンジ(激落ちくん等)
\item 雑巾
\item ビニール紐(資材梱包・研究のため)
\item 空気清浄機フィルタ(精密機器保護のため)
\item セロテープ(50mm幅、資材梱包・論文作成のため)
\item ドラフティングテープ(掲示物貼り付けのため)
\item 両面テープ(薄型、プライム用マーカー貼り付けのため)
\item 養生テープ(資材梱包・マーキングのため)
\item 結束バンド
\item テプラカートリッジ
\end{enumerate}

\subsubsection{経費での購入不可品目}
以下に示す備品は,研究室経費では落とせない

\begin{enumerate}
\item ベープ詰め替え
\item ポット洗浄中
\item 害虫駆除関連
\item ティッシュ
\item ウェットティッシュ(アルコールシート)
\item 食器用洗剤
\item ハンドソープ
\item スポンジ
\item 水切りネット
\item 台ふきん
\item 個人的に使用するもの
\end{enumerate}

\subsubsection{仮納品書での購入}
等々力,世田谷キャンパスの文具ストアで購入する際は,仮納品書を店員に書いてもらい,購入する.書いてもらった仮納品書は野中先生か関口先生に渡す.この場合,管理係が立て替え払いを行う必要は無い.

\subsubsection{立て替え払い}
キャンパス内の文具ストア以外の一般の店舗で購入を行う場合は,管理係が立て替え払いをする.この時,領収書を店員に切ってもらうことを忘れないようにする.領収書を切ってもらう際には,以下の「領収書に関する覚書」を参照されたい.また,立て替え金を自分で用意できない場合は,先生に相談すること.
\begin{shadebox}
%%
\begin{center}
領収書に関する覚書
\end{center}
%%
\begin{itemize}
\item 金額は5万円未満,合計で3万円以上になる場合は明細を示す
\item 宛名は「東京都市大学」
\item 品目は具体的に.例:抵抗,アクリル板,アルミ板他
\item レシートは不可.社印が必要.\underline{3万円以上は収入印紙が必要}
\item 可能な限り早く清算する
\item 可能な限り領収書の数を少なくする
\item 文具ストア(ハヤト商事)で買えるのなら,先生がそちらで購入
\item 東急ハンズで購入する場合は割引カードを使用する
\end{itemize}
%%
\end{shadebox}
%%

\subsubsection{購入希望備品情報の収集}
管理係が研究室で必要となっている備品の情報をすべて把握することは困難である.そのため,研究室のメンバーから必要な備品の報告を受け,それを基に備品を購入するのが慣例である.しかし,口答でのみ報告をされると購入希望備品が漏れ落ちることもある.こういったことを防ぐため,購入備品リストを作って管理したり,購入してほしい備品を研究室のメンバーが自由に記入できるようなファイルをworkspaceに置いておき,随時記入してもらうなどの方法を検討すると良いかもしれない.

\subsubsection{その他の留意事項}
冷蔵庫等の大きな備品を購入する際は,購入により捨てることになる古い備品の処分方法や,新しく購入する備品を廃棄する際のことを考えて購入する.
研究室運営費で購入できないものは,会計係で余っているお金を使用させてもらうことが慣例となっているようである.\\
%しかし,2018年度からはキャンパスが分かれたため,管理係が立て替えておき,適宜該当キャンパスの学生に請求することをおすすめする.
\ キャンパス内の文具ストア以外の一般の店舗で購入を行うとき,領収書に商品名が記載されていないことがある.そのようなときには必ずレシートも一緒に先生に渡すようにする.




\subsection{捨てるものリストの作成}
年度末に,大型備品の廃棄を行う.(2016年度の日程は,申請書の提出期限は1月17日,運び出し期間は2月16日頃である.先生がこの日程を把握してない可能性があるので,時期が近付いたら確認すること.)
その際に廃棄する物をリストにまとめ,先生に渡す.捨てるものリストは一度に作成すると大変なので,捨てるものがあればこまめに作成を行うと良い.また,こちらで捨てるものを全て把握することは難しいので,先生や他の研究室のメンバーから報告を受けるようにする.
備品番号は上から3,4桁目が購入年度になっている.購入から10年経たないと捨てることはできない.
%(例:2017年度->M05\textcolor{red}{07}までは捨てることが可能)



\subsection{不要掲示物の撤去}
締め切りの過ぎている学会参加募集や,インターンシップのお知らせなどを適宜撤去する.



%\subsection{PCの管理}
%\subsubsection{PCの分配}
%4月初めの研究室のオリエンテーションの時期くらいに,4年生へPC・ディスプレイ・キーボード・マウスの配分を行う.PCの配分を行う際は,自分のノートPCを持っていない人を最優先とし,その後にMATLABやMathematicaなどの情報基盤センターや研究室がライセンスを持っているソフトを使用する予定の人へ配分を行う(基盤センターや研究室がライセンスを所有しているソフトは個人PCにインストールすることはできないため).これらの配分が終わった後に研究室のPCが欲しい者へ配分を行う.ディスプレイやキーボードなどは,適宜配分を行う.
%
%また管理係がM0の場合は,B4とM2が卒業した後にM0,M1へPCの再配分を行う.管理係がB4の場合は,卒業式前に誰がどのPCを使用するのかを決めておき,卒業後に在学生が適宜交換を行うと良い.その際,新しいPCやディスプレイへの交換を希望しない者に関してはそのままで良いが,新しいPCやディスプレイへの交換を希望する者に対しては,先に述べた要領で再分配を行う.
%
%\subsubsection{ディスプレイの管理}
%2018年度はディスプレイを購入したため,ディスプレイ割り当てシートへの登録と,ディスプレイ番号を示すシールを貼り付けた.

\subsubsection{備品の管理}
2020年度,リモートでの研究が求められるようになったため,希望者にウェブカメラ,ヘッドセットに加え,マウスとキーボードを購入した.また,購入した物品に関してはテプラで番号を記し,管理をした.
管理用の番号については管理係/2020年度備品係/備品管理番号.xlsxに記載した.

2022年度,前の年度に購入したPC等にテプラで番号を記し,管理した.管理したものについてはWorkspace2022/Work2022/Shere/備品管理に記載している.黄色のシールについては正面,全体,備品番号の画像を添付し,青いシールについては備品番号を貼った備品の画像を添付する必要がある.

\subsection{席決め}
新年度と年度末には,座席決めを行う.座席を決める際は,上の学年からあみだくじ等で座席を決めるのが慣例である.座席表は係引継ぎ時に渡すが,座席配置を決める際には「AR\_CAD」という2次元CADソフトを用いた(基本的には何を使っても良い).過去にはMicrosoft Office Visioを使っていたこともあったようである.VisioはMicrosoft DreamSpark Premiumにより,無料でダウンロードできる(学生個人のアカウントがある).

\subsection{靴箱}
新年度には靴箱の名札を更新する.名札はテプラで作成する.



\subsection{先生・学生不在時間割表の作成}
先生と学生が授業やTAなどで研究室に居ない時間を把握するための不在時間割表を用意する.なお,不在時間は先生・学生それぞれに書いてもらうため,管理係は不在時間を記入できるファイルをクォーターごとに作成するだけで良い.昨年は時間割表をworkspaceに置いておき,皆に記入してもらった.もしファイルをworkspaceに置くなら,フォルダとファイルの編集権限を「everyone」に与えるようにしておくこと.時間割表が確定したら,4部印刷しそのうちの2枚は野中先生と関口先生にお渡しし,残りの2枚はそれぞれ総研学生室と世田谷の4階に貼っておく.



\subsection{研究室の引っ越し計画・環境整備}
この節に書く内容は毎年行うものではないが,2013年度は鈴木先生のご退職による研究室のレイアウト変更,2014年度には研究室4階のリフォーム作業に伴う研究室の環境整備を行う必要があった.
また,未遂であるが2016年度には上階下階合併計画があった.
2017年度にはクウォーターが変わるごとに席替えを行った.
2019年度には世田谷キャンパス10号館の取り壊しを機に総研への完全移設を行った.
2020年度にはコロナウイルスの流行による分散登校のため,世田谷キャンパスと総合研究所を用いて密にならない席決めを行った.
研究室の引っ越し等がある場合には,管理係が計画の立案,作業時の指揮を行う.
計画の立案は余裕を持って行い,先生や経験のある先輩と相談したり,アドバイスをもらったりしながら進めると良い.
また,環境整備には多くの備品を調達したり,引っ越し日程を考えたりと作業内容が膨大となる場合がある.
そのような場合には,手の空いている研究室のメンバーと協力しながら進めるようにする.
\textcolor{red}{全て一人で行うと作業量が膨大であるため,室長などにも協力をしてもらいながら計画を進めていくようにする.}

参考までに,以下に昨年行った研究室改修工事に伴う一連の流れを書いておく.
%%
\begin{enumerate}
\item 改修工事で行う内容の決定(先生と業者)
\item 改修工事期間中は研究室を使用できないため,中2階へ引っ越すことにした
\item 中2階の座席レイアウト・電源配線・LAN配線計画の立案(必要なテーブルタップ・ハブ数等の確認)
\item 中2階への引っ越し日時と作業手順計画の立案・実行
\item 改修工事終了後に新たに必要となるタイルカーペット枚数の計算・発注
\item 改修工事後の4階の机配置,電源・LAN配線計画の立案
\item 改修工事により不要となった物品の洗い出し(捨てるものリストへ追加)
\item 机増設後に必要となる椅子の数の確認(廃棄品の置き換え含む)
\item カーペット貼り付け作業時に必要となる道具の確認・調達
\item 4階への引っ越し日時と作業手順計画の立案・実行
\end{enumerate}
%%
なお,電源は柱や壁に埋め込まれているコンセントを使用するのではなく,柱に取り付けられている白い配電盤から供給されているものを使用する.柱や壁に埋め込まれているコンセントを使用すると,ブレーカーが落ちることがある.ブレーカーが落ちると他の研究室の電源も使えなくなるため,柱や壁に埋め込まれているコンセントは極力使わないようにし,掃除機や加湿器程度に留めるのが良い.\\

また2019年度に実施された総研への完全移設における引っ越し実行までの一連の流れを以下に記す.
%%
\begin{enumerate}
\item 先生や古田さんと相談をして引っ越し日を決め,引っ越し業者に予約をする.
\item 机を処分するものと総研に置くものに分ける.このとき机は幅が100\,cm\,以上のものを総研のものとし学生の人数を考慮して不足している机を購入した.
\item 本館2階の備品を処分するもの,2階に残すもの,1階に移動させるものに分ける.
\item 学生室,本館1階,本館2階のレイアウトを作成する.このとき部屋の大きさを考慮した上で机の配置を決める.
\item 椅子を学生室,総研1階,総研2階,処分するものに分ける.
\item 電源配線,LAN配線を決める.このとき必要な電源タップ,ハブの数を確認する.
\item 引っ越し当日に会議室や踊り場に学生室の机や冷蔵庫を移動させるため,その配置を決める.\\以上の内容を引っ越しの前々日までに決める.
\item 引っ越しの前々日までに学生全体に机の引き出しが飛び出さないように養生テープで固定させる.
\item 引っ越し前日に以下のように配置を行った.\\実験室:学生室のディスプレイや荷物等\\会議室:学生室で使い続ける机や椅子と総研1階の机\\学生室:総研の2階で使用する机または椅子,会議室に入りきらなかった椅子\\総研踊り場:世田谷に運ぶ机と椅子
\item 各部屋に人数を配分して,引っ越しを実行
\end{enumerate}
%%
\ 引っ越し当日の主な内容としては,机や棚,不用品の運び出し,備品の移動,床下配線の再構築,机や棚の設置,新しい机や椅子の組み立て,床上配線の構築,窓掃除やエアコンのフィルターの掃除などといった清掃,各自の荷物の設置となる.\\
\ なお,処分をする机や椅子は全て世田谷キャンパスに移動させた.このとき世田谷のレイアウトも同じように決めておく.大がかりな机の移動であったため,移動場所を明記した養生テープを机に貼った.
引っ越し当日は必ず50分程度の休憩を挟むようにする.\\
\ また,総研に移設したばかりであるため,本館1階などの備品が不足していることがあるため,随時対応をできるようにする.


また,2020年度のコロナウイルスの流行を受けて,研究室では世田谷キャンパスと総合研究所への分散登校を行った.そのため,今までの席配置では無く,席をひとつおきに確保した.2019年度に実施された引っ越しと同様に席配置などをExcelファイルで作成し,机などの移動はほとんどすることなく,ソーシャルディスタンスを保った席配置を行った.参考として,管理係/2020年度備品係に席決めなどに用いたExcelファイルを置いておく.また,2021年1月にパーテーションに使うアクリル板を入手したため,今後パーテーションの設置を行う可能性がある.



\subsection{カーペット貼りの手順}
2014年度に行った研究室の改修工事の際に,カーペットを貼り直した.ここでは,改修工事後に行ったカーペット貼りの手順を説明する.

カーペットはオフィス用のタイルカーペットを使用する.また,貼り付けにはカーペット用接着剤を使用した.なお,2013年度にもカーペット貼りを行ったが,その際には一般の接着剤を使用した.カーペットを貼る際に使用した道具を以下に示す.
%\begin{center}
%カーペット貼りに使用した道具
%\end{center}
\begin{itemize}
\item タイルカーペット
\item 黒刃カッター
\item 床用接着剤
\item ハケ
\item 発泡スチロール製容器(接着剤取り分け用)
\end{itemize}
%%
ここで,黒刃カッターはカーペットを壁や床の形に合わせてカットするために用いた.カーペットは非常に切りづらいので,切れ味の良い黒刃カッターを使用すると良い.また,床用接着剤は大きな容器院入っていたため,小さな発泡スチロール製容器に取り分けて使用した.そして容器に取り分けた接着剤をハケですくい,床に塗布した.

次に,2014年度に行ったカーペットの貼り付け手順を以下に示す.
%%
%\begin{center}
%カーペット貼り付け手順(2014年度)
%\end{center}
%%
\begin{enumerate}
\item カーペットが貼り付けられていない状態で,床の掃き掃除及び拭き掃除を行う
\item カーペットを配置する
\item カーペット用接着剤を,カーペット1枚分ずつ開けて床にハケで塗る
\item 白い接着剤が半透明になるまで乾かす
\item 接着剤が塗られた床の上にカーペットを敷く
\end{enumerate}
%%
以上の手順でカーペットを敷いた.カーペットの敷き方は,使用するカーペットや接着剤により若干変わってくるので,接着剤の注意書きなどをよく読むと良い.なお,研究室には沢山の備品があるため,荷物を全て外に出してカーペットを敷くことは難しい.そのため部屋を半分に分けて片側に荷物を移動して寄せ,もう片方で作業するというように,半分ずつカーペットを敷いていった.また,カーペットの切断には時間がかかるので,カーペットを貼る役割の人と切る役割の人に分けて作業を行った.
%%
\subsection{世田谷キャンパスにおける定期的な断水・停電時の対応}
世田谷キャンパスでは定期的な点検のため大学全体で断水や停電を行う日がある.この日程については予め連絡されるので,確認してカレンダーに記入しておく.これに備えて各階で行うべき対応を以下に示しておく.また,作業を管理係一人でこなすのは大変なので,研究室のメンバーに協力を仰ぐとよい.
2019年度の停電,断水,ネットワーク停止は8月中旬から下旬に行われた.また,消防設備やガス設備の点検も同じ期間に行われた.

停電の際は,学生室に関しては,各々の机周辺のコンセントについて帰宅時に抜いてもらうようにしておくように連絡しておくのが良い.
ミーティング部屋およびフィールドについてもコンセントを抜く.
天井のコンセントは脚立を用いる必要があるので注意すること.
ただし,先生の部屋につながっているコンセントについては抜かなくてよいが,確認はしておくこと.(先生自身が居室内のものに関して処理されるはずなので)
また,サーバPC(workspace)に関しては,PC係にシャットダウンをお願いし,その後コンセントを抜くこと.
停電後は最初の活動日に原状復帰すること.

断水の際は,断水後の活動開始前に水道を30分ほど流しっぱなしにし,汚い水を流す.
 % %
 % \begin{figure}[tb]
 %   \centering
 %   \includegraphics[width=0.8\linewidth]{management/plug_arrangment.eps}
 %   \caption{研究室5Fの配線およびコンセント位置}
 %   \label{fig:plug_arrangement}
 % \end{figure}
 % %
 %
%
\subsubsection{世田谷5階}
停電に備えてコンセントを抜く.配線とコンセント位置に関しては大まかに
管理係の引き継ぎフォルダ内の「配線とコンセント位置5F.pdf」
% Fig.~\ref{fig:plug_arrangement}
に示すので参考にすること.

\subsubsection{世田谷4階}
壁や柱に刺さっているコンセントは抜き,配電盤のブレーカーを落としておく.
なお,ブレーカーを落とす際には,全員のパソコンの電源が落ちていることを確認する.(パソコンをつけっぱなしにしている人がいるため)

\subsubsection{プロジェクター}
世田谷4階のプロジェクターは電源に不具合があり,スリープ状態から復帰が出来ない.そのため,使用後冷却が完了してから電源プラグを抜き,使用時に電源プラグを差してから電源を入れることで対応している.



\subsection{プリンターの管理}
プリンタートナーの残量を管理する.プリンターのIPアドレスをwebで検索すると,詳細なプリンターの状態が見れるので,お気に入りに登録しておく.残量レベルが少なくなったら,関口先生にトナーの注文をお願いする.この時,ヨドバシカメラのURLをまとめたものを送る.使用されているプリンタのトナーはC310Hという型番のものを使用している.過去にリサイクルのトナーを使用したことにより論文が印刷できなかった事例がありました.そのため\color{red}{トナーは正常に印刷が行えるように,必ず純正品を購入するようにする.}\\
% 関口先生にトナーの注文をお願いする.この時,アマゾンのトナー販売ページのURLをまとめたものをメールすると良い.
\color{black}{
\ また,トナーを交換した際に使用したページ数をプリンター横にある表にメモしておく.
プリンター用トナーを交換した際の使用済みトナーは,交換したトナーが入っていた箱に入れ,使用済みシールを貼ったうえ,購買で回収してもらう.\\
\ 2020年3月現在,総研のトナーはリサイクルのものを使用しているが,これは無くなるまで全て使い切る.
}

\subsection{使用済みの電池の処分}
電池は裏門横のゴミ捨て場に持って行けば回収してもらえるはずです.
電池の研究室での回収場所は,先輩に聞いて下さい.(M0に場所がどこか伝えて卒業してください.)

\end{document}

\usepackage{input}
  \newcommand{\test}{waya}
\begin{document}
\maketitle
\section{学園祭係}
\subsection{主な仕事内容}
学園祭係の主な仕事内容は
\begin{enumerate}
  \item 世田谷祭の説明会への参加
  \item 各種書類への記入・提出
  \item 販売品目・価格の決定
  \item 当日の陣頭指揮
\end{enumerate}
である.詳しい内容は以降で説明する.

\subsection{世田谷祭の説明会について}
*H30年度修正

世田谷祭の説明会は当日までに3回ある.
第1回目説明会の日程については食堂などで告知しているので,よく確認してください.


説明会では世田谷祭参加団体への資料と各種書類が配布される.資料については運営委員がその場で読み上げるが,重要な箇所を読まないことがあるので,資料はよく読んでください.

学園祭係が代表責任者となり,ほかに副責任者,酒類取扱責任者がある.これらすべての責任者は当日に印鑑を押さなければならないので忘れないようにする.
また,片づけ日には代表責任者含め4人ほどはテントの片づけを手伝うことになっているが,雨天時は手伝わなくてよいことはマニュアルに書いていないので気を付ける.

\subsection{各種書類について}
説明会で配布される各種書類は確約書と申請書に分かれている.
確約書は内容に同意するかどうかの書類なので,よく読んで記入漏れがないかチェックしてください.
申請書は運営委員会への物品使用申請書類が主である.これもよく読んで必要事項に記入してください.ただ,火器の申請書には全て(ガスボンベ,カセットガスコンロ,炭,ガスバーナー)にチェックを入れることをお勧めする.理由として,消防法の関係で恐ろしく早い段階で火器の申請書を提出するが,販売品目の決定はもっと後になる.販売品目の選択肢を多くするためにも火器の申請書には全てチェックを入れるのが良い.

また,H30年度よりネット上の登録フォームによる申請に変更されている.
書類に記載されている期日までに申請し,登録完了の返信を確認すること.

\subsection{販売品目・価格}
販売品目は取り扱い可能食品を参考に決定すれば問題ない.取り扱い可能食品か判断できない場合は運営委員会に確認すること.

1, 2ヶ月前に,試食会などを行い,量や価格を設定し,宣伝用の写真をとり,ポスターなどの作成に用いる.
価格について多少高めに設定しても問題ないが,商品を半分売ったときに予算を回収できる利益設定が望ましい.
薄利多売方式はお勧めしない.これは予算回収のために多く売る必要があり,予想より売れ行きが悪い場合に予算回収できない可能性がある.


\subsection{当日の陣頭指揮}

当日の屋台番が混乱しないよう,事前にシフト表を作成することをお勧めする.シフト表を作成する際の留意点として
\begin{itemize}
  \item 1日目はMESSAGEに常時数人,一時的には10人以上の人員を割かなければいけない.
  \item 交代は一人ずつ行う.
  \item 各責任者・救命講習受講者は最低一人は配置.
\end{itemize}
などがあるので,これに注意して作成して欲しい.



また,全体への連絡手段をひとつ確保しておくと良い.

\subsection{設備に関して}
長机と長椅子は1団体に付き,それぞれ1つづつ貸与される.
出品するものによって異なるが,ものによってはスペースが足りなくなることがある.
H29年度では,研究室の廃棄予定の机を1つ拝借し(野中先生の許可済み),調理スペースと販売スペースの確保を行った.
必要そうだと判断した場合,早めに確保すること.

また材料を保存する際,研究室の冷蔵庫を使用することも可能である.
この場合,予め各員に連絡し,スペースの確保に協力してもらってください.
それが出来ない場合は,近くに住んでいる研究室のB4の人に頼み,冷蔵庫を貸してもらうなりすること.
保冷バッグは1夜持ちません.大変でした.

H29/30年度に購入した調理器具は研究室に寄付しているが,
他のメンバーが使用しているために,使用するのであれば承諾を得てから持ち出すこと.

\subsection{2020年度の対応}

日程\\
・夏の説明会(7月9日)\\
・秋の説明会(9月30日)\\
・最終説明会(10月28日)\\
・準備日(11月6日の午後から)\\
・世田谷祭(11月7日,8日)\\
・片付け(11月9日)\\



\end{document}

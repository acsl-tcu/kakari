%研究室生活を円滑に進めるために,全員に周知しておきたい内容を記入

%更新日:20180115_津野
%更新日:20190321_柴田浩志

%% レポートテンプレート
\documentclass[onecolumn]{jsarticle}
%% パッケージの設定
\usepackage[dvipdfmx]{graphicx}
\usepackage{msethesis}
\usepackage{amssymb,amsmath}
\usepackage{bm}
\usepackage{url}
\usepackage[dvipdfmx]{color}
\usepackage{wrapfig}
\usepackage{comment}
\usepackage{input}

\begin{document}

\section{学部生の引継ぎ (\color{red}学部生要閲覧\color{black})}

\subsection{1年の流れ}

一年の流れについて列挙します.
略記:B(学部),M(修士),D(博士)
%
\begin{table}[h]
  \centering
  \caption{1年の流れ}
  \label{table:StreamOfTime}
  \begin{tabular}{|l|l|c|c|}\hline
月&時期&内容 &備考\\ \hline 
4月&オリエン期間&研究室オリエンテーション & 研究室始動\\ \hline 
4月&始動より約1週間後&卒論構想発表会 &対象B4 \\ \hline 
5月&GW明け&中間発表会 &対象B4~D \\ \hline 
6月&中旬頃&報告会 &対象B4~D \\ \hline 
7月&中旬~下旬&B3向け研究室紹介 &B4中心準備 \\ \hline 
8月&第1週&中間発表会 &対象B4~D \\ \hline
8月&第1週?&外部向けオープンキャンパス &B4中心準備 \\ \hline 
8月&下旬まで&夏休み&中間発表会の報告書提出後 \\ \hline
8月&下旬&修士論文中間発表会&学科主催 \\ \hline    
9月&中旬?&合宿 &B4旅行係計画 \\ \hline 
9月&下旬&B3研究室配属 &B4中心研究班紹介 \\ \hline
10月&上旬&報告会 &対象B4~D \\ \hline
10月&下旬&世田谷祭 &B4準備 \\ \hline 
11月&中旬&報告会 &対象B4~D \\ \hline
12月&クリスマス前&中間発表会 &対象B3~D \\ \hline
%1月&&@@ & \\ \hline 
2月&上旬&事例研究最終発表会 &対象B3 \\ \hline 
2月&中旬&修士論文公聴会 &学科主催 \\ \hline
2月&中旬&卒業論文公聴会 &対象B4 \\ \hline 
3月&&引継ぎ & \\ \hline 

  \end{tabular}
\end{table}

\subsection{発表会について}

ほぼ毎月一回発表会を行います.
各自発表内容について,パワーポイントを作成しその時々の時間に従って発表を行います.
発表会についての大まかな流れは以下に示します.
\begin{itemize}
  \item 発表会前にパワーポイントを先輩等の添削を貰いつつ作成
  \item 中間発表会の場合,発表前日までに報告書の仮提出を行う
  \item 作成したパワポを発表係が指定した締め切りまでに(/workspace/lab\_data/papers/西暦/指定フォルダ)まで提出する
  \item 発表会中は一人一回は質問を行う
  \item \color{red}発表会後,1週間以内に報告書の提出を行う\color{black}
\end{itemize}
報告書の提出場所は,その時の先生の指示に従ってください.(workspaceかWebClassに提出)
B4は2カラムの卒業論文概要集形式で報告書を提出.M1以上は報告書か論文などを提出.
12月中間発表会の後に,B4卒業論文の第一稿仮提出がありました.
ここよりB4は1カラムの文章となります.

外部の学会発表などの予定が発表会の日程の近くにある場合は,発表会を学会発表練習に置き換えることが出来ます.
発表時間,質疑時間は発表先の時間に合わせてください.

また,修士2年の修論発表(夏の中間発表も)では4年生の手伝いが必要です.
この手伝いは高機能研のときだけで結構です.
カメラ,質問のメモ,タイムキーパーに加えてマイクを運ぶ人が必要です.
これは質疑で挙手をされた先生のもとへマイクを持っていく係です.
本番までにあらかじめ決めておくとスムーズに運営できます.

\subsection{研究室紹介のパワーポイントについて}

研究室では,研究室紹介のパワーポイントが英語版と日本語版の2つあります.
英語版は実験やシミュレーション結果をたくさん張ったもので,日本語版は主にオープンキャンパスや3年生向けに使うものです.
なので,英語版では歴代の見栄えする動画を毎年載せていってTCU-ACSL-Best collectionみたいな感じにします.
今までのも最近の研究も併せて載せるように.逆に日本語版が使われる状況には時間制限がある場合が多いです.
なので,載せる結果を絞って載せてください.入った後でがっかりさせないように直近の結果の方がいいです.
また,研究室の紹介の要素も忘れずに.毎年,英語版は5月ごろM0が更新します.
日本語版は6,7月ごろ研究室紹介で全体紹介する人が更新します.
上記のことに注意して作成してください.

\section{修士生の引継ぎ (\color{red}修士生及びM0要閲覧\color{black})}

\subsection{1年の流れ}
表\ref{table:StreamOfTime}を参照.
修士生は主にB4のサポートがメインになります.
B4のときに担当していなかった係でなくとも,気にかけてあげてください.
「B4が聞いてこなかったから言わなかった.知っていると思った.」は許されません.

\subsection{発表会について}
編集中

\subsection{学会について}
\subsubsection{論文作成の注意点}
学会論文ではフォントを埋め込んだpdfファイルを提出することが要求されます.
埋め込まれていない場合,スマホや他のPCでの閲覧時や印刷時に文字化けを起こすことがあります.
従って,学会に論文を提出する場合は,以下の手順でフォントが埋め込まれていることを確認してください.
\begin{enumerate}
 \item 確認するPDFを任意のリーダー(Adobe Acrobat,Foxit など)で開く
 \item プロパティのフォントを選択,そのPDFで使用されているフォントの一覧が表示されます
 \item 全てのフォント名の横に''(埋め込みサブセット)''がついていることを確認
\end{enumerate}
もし,'(埋め込みサブセット)''がついていないフォントが存在していた場合,
「svn://192.168.120.100/lab/係引き継き資料/works/全体的な引継ぎ事項/TeXpdfフォント埋め込み方法他.txt」に解決方法が書かれているので参照してください.


\subsection{Presentation~Competitionについて}

Presentation~Competition~(PC)では,英語による論文・ポスター作成,発表技法の向上を目的として,機械システム工学専攻の修士1年がポスターを用いて発表を行う.
本項では,PCに参加するにあたって注意すべき項目をまとめておく.
また,「http://www.mse.tcu.ac.jp/student/pc」より,日時やフォーマット等の案内を確認すること.
\begin{itemize}

  \item 履修登録\\
	PCは大学院の授業科目である「機械システム工学専攻事例研究」として行われる.
	単位も付与されるため,履修登録時に忘れずに登録しておくこと.

  \item 発表日までの計画\\
	PCの発表日は例年,7月の中旬頃に予定される.
	それまでに後述する予稿・ポスターの作成や発表練習があるため,
	それらの添削及び発表練習日の計画を立てること.
	人によってはSICEなどの学会や,夏のインターンシップの申込,
	TAの業務などと重なるため,それらを踏まえた上で計画することが望ましい.
	
  \item 予稿\\
	例年,A4サイズの1ページで発表内容の概要を作成している.
	発表日の1週間程度前に提出期限がある.
	PCの発表内容の骨子となるため,早めに作成を始めて仕上げておくことが望ましい.

  \item ポスター\\
	学科webサイトのフォーマットに従い,作成すること.
	発表日前日は他研究室の学生が印刷するため,
	情報基盤センターが混雑する恐れがある.
	遅くとも発表日前々日には印刷してあることが望ましい.

  \item 発表練習\\
	ほとんどの修士1年にとって,PCが初めての英語発表の場となる.
	そのため,複数回の段階に分けて発表練習を行うことが望ましい.
	前年度~(2017年度)は4回の発表練習を行い,
	1回目から,日本語原稿・ポスター,
	日本語原稿・英語ポスター,英語原稿・ポスター,英語原稿・ポスターで行った.

  \item レポート\\
	PC発表後,指定された学生の発表内容をレポートにまとめて提出する必要がある.
	提出期限があるため,早めに提出しておくこと.

\end{itemize}

%その他詳細な引継ぎ事項は「」を参照.

\subsection{ATACSについて}
修士1年はATACSにおいて特に作業が無いため,
この節では修士2年を対象とした引継ぎ事項をまとめる.
また博士課程の学生については内容が概ね重複しているため省略する.
\par
2018年度のATACSにおける修士2年の活動内容を表\ref{table:atacs_schedule}にまとめる.
%%
\begin{table}[htb]
	\caption{ATACS2018:修士2年の活動内容}
	\centering
	\begin{tabular}{|c|l|}\hline
		日付 						& 内容											\\ \hline\hline
		9/25							& 発表タイトル締切(ATACS係)		\\ \hline
		9/30							& 発表タイトル締切(ATACS幹事)	\\ \hline
		10/4							& 予稿締切(ATACS係)					\\ \hline
		10/5							& 予稿締切(ATACS幹事)				\\ \hline
		10/11						& 発表練習									\\ \hline
		10/12 ~ 11/14 		& ATACS本番								\\ \hline
	\end{tabular}
 	\label{table:atacs_schedule}
\end{table}
%%
発表タイトル・予稿は「修士2年→ATACS係」と「ATACS係→ATACS幹事」のそれぞれ2回の提出期限が設けられている.
後者の提出についてはATACS係に一任しているため,発表者である修士2年は前者の提出締切日を守る事.
発表練習は予定が合えば研究室全体で行うため,日時や場所について発表会係と前もって相談する事.
予稿の体裁はA4・2カラム・4~6ページ(超過可)であり,予稿テンプレートは事前に幹事から送付されたものを使用する事.
修士2年の発表時間は17分(発表:12分,質疑応答:5分).

\subsection{TAについて}

TAを行うにあたって,いくつか書類を出す必要がある.
\begin{itemize}
  \item 振込依頼表\\
	TAの給与の振込口座を指定するために必要.
	初回給与支給日の七日前までに提出する必要があるが,
	年度の初めに次の書類とともに準備することが望ましい.

  \item 扶養控除申告書 or 扶養控除申告書 他事業所への提出報告\\
	該当するどちらかの書類を提出すること.
	現在アルバイトをしていて扶養控除を出した人は後者を提出.
	何それ?って人は前者を提出すること.
  
  \item TA出勤簿\\
	この書類のみは毎月提出する必要がある.
	従事した時刻と業務内容,先生および自分の押印を忘れずに.
\end{itemize}

以降は各授業のTAに対しての引継ぎを記す.

\subsubsection{Cプログラミング}
編集中

\subsubsection{機械システム応用実験}
編集中

\subsubsection{技術日本語表現技法}
第1回で担当する学生を決定する.
(人数で等分配してもよいが,1年生はかなり能力差が大きい.
そのため大抵の場合人数で等分配すると失敗するので注意が必要.)

講義の始まる10~15分前には1号館へ行き,鍵を借りて置く.
また,学生にプリントを予め配布しておく.
このとき,余分にPC端末を起動しておく.
あらかじめ何台かのPCに自分のアカウントでログインしておいて,
何か問題が発生した場合にWebclassだけ学生自身のアカウントでログインさせても良い.
講義が始まったら学生の出席を確認.遅刻した人はどの程度遅刻したかをチェックしておく.
欠席や遅刻の学生は野中先生があらかた講義を終えた際(演習中)に先生へ報告する.
要チェックな学生は名簿自体にわかりやすく目印をつけてい置くとよい.

欠席した人は学番の近い学生に聞いて何か知らないか確認.
後日講義に来た場合は欠席について本人から確認しておく.課題が提出されていない,
著しくクオリティが低い場合も同様.本人に直接話を聞くこと.

\subsubsection{電気基礎実験}

主な業務は
\begin{itemize}
  \item 実験の補助
  \item 提出レポートの添削
\end{itemize}
実験を円滑に進めるために,各実験の数日前にはTA内で事前実験を行うこと.
前半・後半に学生を分けるため同じ実験を2回行うが,事前実験は1回のみで良い.

\subsubsection{ロボット制御プログラミング}
第1回目は,WebClassに名前がない人をWebClassに追加する作業を行う.
そのため,1年前にロボット制御プログラミングのTAを行っていた先輩と相談し,第1回目に手伝ってもらうようにする.
ただ,このときの先輩のTAの給料は基本的に出ず,ボランティアになるので注意.
第1回目は,担当者がWebClassへの追加を行っている最中は,先輩が質問への対応を行い,追加が終わり次第2人で質問への対応を行う.
\par
講義が始まる前は,野中先生と一緒に研究室から出ていき,学科事務室の池田さんのところで配布資料を受け取ったあとに教室の鍵と机のロックを解除するための鍵の2つを取りに行く.
先生が一緒に行く場合は,先生が基本的にやるので一緒についていくだけでよい.
ただ,先生がお忙しい場合は,担当者が先に行く必要があるため,学科事務室の池田さんのところでロボット制御の配布資料を野中先生に言われて取りに来た旨を伝え,資料をもらうことと,学生支援センターで教室の鍵と机のロックを解除する鍵の2種類をもらう.
その後,教室へ行き,机のロックを解除し,プロジェクターを降ろす作業を行う.
また,毎回の授業の最初に小テストを行う事が多いので,その場合はすぐにテスト用紙を配れるように各列の席数分ずつで用紙を分けておくこと.
授業で使用する資料については,小テスト中に用紙を分ければ間に合うが,小テストは開始後すぐにやるので,しっかりと準備する.
授業中は,先生が説明しているときには教室の後ろで立っているか,一番前の席の廊下側の席(TAの担当者の席)に座って待機する.
先生の説明が終わり,授業課題を行う際に教室内を回って,質問へ対応する.
このときには,手を挙げている学生への対応はもちろんのこと,回る最中にすこし画面を見て苦戦していたり,悩んでいたりする人にアドバイスをするとよい.
\par
基本的に,余った資料はTAが管理する.
2回分くらい前の資料は,毎回の授業で持参すると,前回の授業に出席していなかった人の対応が楽になるのでおすすめ.
また,資料が足りない場合はとりあえずの応急処置で周りの人に見せてもらうようにしてもらう.
最初の授業での出席者をもとに枚数を計算するので,毎回修正はするものの足りなくなることがたまにあるので注意.
\par
採点については,基本的に担当者はしなくてもよい.
野中先生の採点基準で採点を行う事が多いので,採点しなくてもよい.
ただ,採点基準が明確だったりする場合は採点してもいい.

\end{document}
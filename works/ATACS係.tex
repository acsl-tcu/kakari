\usepackage{input}
  \newcommand{\test}{waya}
\begin{document}
\maketitle
\section{ATACS係}
ATACS係はATACSの幹事との諸連絡や提出論文の管理など,
ATACS関連の作業全般を請け負う係です.

\begin{center}
	\it ATACS\\
	\it \textcolor{red}{A}dvanced \textcolor{red}{T}heory and \textcolor{red}{A}pplication of \textcolor{red}{C}ontrol \textcolor{red}{S}ystems
\end{center}
ATACSは制御に関する研究をしている研究室が集まり,発表しあう場です.
修士$1$年と学部$4$年はその発表を聞きます.

ATACSに関しては幹事の大学と連絡をとり,必要なことを聞きます.
基本的に,幹事の大学の指示通りに行っていけば大丈夫です.
幹事を行う大学によりますが,大体$5,6$月ごろに参加・発表予定者の確認の連絡が届きます.
その後は,幹事大学と連絡を取り合って,その指示に従ってください.
また,ATACSについて幹事様より情報が渡されたら,メーリングリストなどで研究室の全員にその内容を周知させてください.

\subsection{2017年度まで}
ATACSでは修士$2$年と博士課程の人達が発表を行います.
発表を行う人の論文を印刷をして,幹事の大学に郵送します.
相当な量の印刷を行うため,A$4$サイズの紙が入る箱を用意したおいたほうがいいです.
加えて,印刷論文の締切や,印刷機の借りて印刷する日,印刷物を並び替え,
纏める日,郵送日などのスケジュールも作成しておきましょう.
具体的な郵送締切日は幹事大学より随時連絡があるはずです.
参考程度に,$2015$年度は$6$人の論文を$240$部印刷したため,A$4$が入る箱を$2$箱使用しました.
また,論文を印刷する際に,奇数ページの人は最後に白紙し,次の人は新しい紙に印刷してください.
($1$枚の表裏で論文の著者が変わらないようにしてください.)
印刷を終えた論文すべてを穴あけパンチで穴を開けます.
その際に,論文タイトルの左側に穴が来るように穴を開けてください.
ただし,幹事大学より別途印刷物の纏め方について指示があった場合,それに従って下さい.
($2015$年度は穴あけパンチが不要の指示がありました)
そして,各論文を指定の順番に並び替えた状態で箱詰めし,郵送します.
基本的にこれらの手作業は学部$4$年が行ってください.
$2017$年度では発表者の論文をGoogle Drive上で管理を行いました.
ATACS関連のGoogle Driveのアカウントに論文提出日までに発表者全員の論文を集めアップロードしました.

\subsection{2018年度}
$2018$年度も2017年度と同様に発表者の論文をGoogle Drive上で管理を行いました.
2018年度のアタックスのスケジュールをTable{\ref{2018_ATACS}}に記載しておきます.
\begin{table}[b]
  \caption{2018年度の日程}
  \begin{tabular}{|c|c|} \hline
  Event & Date \\ \hline
  参加者名簿提出についてのメール & 6月19日 \\ \hline
  参加者名簿を提出 & 6月29日 \\ \hline
  発表タイトル提出についてのメール & 9月10日 \\ \hline
  発表タイトル提出 & 9月30日 \\ \hline
  論文提出 & 10月5日 \\ \hline
  参加費振込 & 10月9日まで \\ \hline
  ATACS当日 & 10月12から14日 \\ \hline
  \end{tabular}
  \label{2018_ATACS}
\end{table}
ATACS参加費振込では,振込名義人はATACS係の名前でいいです.
ATACSは,数日間行われるためATACS会場の宿泊施設に宿泊する人が多くなります.
宿泊の必要がある場合は,合宿$\cdot$研修$\cdot$旅行$\cdot$見学届を
1週間前までに学校に提出する必要があるため注意してください.

ATACS当日ですが,各研究室でセッションのチェアの人を出します.
チェアは司会進行や質問がでない時の繋ぎとして質問を行います.
チェアは最低限,司会と時間を見る人とマイクを質問者に届ける人の$3$名は必要です.
これも,基本的には学部$4$年から選出してください.

また,懇親会で行う出し物を学部$4$で行うため,それの準備の必要になります.
過去の出し物を参考に,制御に関わるネタを含んだ出し物を作成してください.
ATACSは大体$10$月に開催されるため,余裕をもって出し物を作成した方が良いです.
$2017$年度は寸劇を撮影したものを流して貰い,それを出し物としました.
過去には,カルマンフィルタを元にした寸劇をその場で行うなどもされたようです.
$2018$年度はMPCをテーマに撮影を行い,動画を流しました.
動画を流す場合はパソコンとデータを持っていく必要があるので注意してください.

もしも急病などの要因で,参加する予定だったはずの学生が参加できなくなってしまった場合,
直ぐに研究室の教授と幹事大学にその旨を連絡し,指示に従ってください.
連絡が遅れると参加費の払い込みがキャンセル不可になったりするので,
これらの連絡は迅速に行ってください.

\subsection{2019年度}
2019年度は,令和元年台風19号により中止になりました.ここでは,中止になるまでの流れ等を書きます.当日の流れに関しては,M2の先輩方に確認してください.
幹事さんとの連絡や幹事校からのお知らせは,全てSLACKにて行いました.
2019年度のアタックスのスケジュールをTable{\ref{2019_ATACS}}に記載しておきます.
\begin{table}[b]
 \centering
  \caption{2019年度の日程}
  \begin{tabular}{|c|c|} \hline
  Event & Date \\ \hline
 初めての連絡が来る & 5月23日 \\ \hline
  参加者名簿提出についてのメール & 7月31日 \\ \hline
  参加者名簿を提出 & 8月28日 \\ \hline
 参加者名簿提出期限 & 8月31日 \\ \hline
  発表タイトル提出についてのメール & 9月17日 \\ \hline
 発表タイトル修正締め切り & 10月6日 \\ \hline
  発表タイトル提出 & 9月30日 \\ \hline
 発表タイトル提出期限 & 9月30日 \\ \hline
  論文提出 & 10月7日 \\ \hline
  論文提出期限 & 10月7日 \\ \hline
  参加費振込 & 9月27日 \\ \hline
  参加費振込期限 & 9月30日 \\ \hline
 台風の影響による連絡 & 10月8日 \\ \hline
 ATACS中止連絡 & 10月11日 \\ \hline
  ATACS当日 & 10月12から13日 \\ \hline
  \end{tabular}
  \label{2019_ATACS}
\end{table}
ATACS参加費振込では,振込名義人はATACS係の名前でいいです.
博士の先輩に確認を取るのを忘れずにしましょう.
\end{document}

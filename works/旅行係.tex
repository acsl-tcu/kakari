\usepackage{input}
\usepackage{color}
  \newcommand{\test}{waya}
\begin{document}
\maketitle
%
\section{旅行係}
主な仕事内容は,旅行の企画・仕切りです.\\
サブバージョンにも資料があるので参考にしてください.\\
やること↓
%
\begin{itemize}
  \item 日にち・場所・内容決め
  \item 旅行会社との連絡の取り合い
  \item 研究室内へのアナウンス
  \item しおりの作成
  \item 当日の仕切り
 \item 費用の清算
\end{itemize}
%
\subsection{日にち・場所・内容決め}
日にちについては先生と相談して決める.早めに決めといたほうがいい.
4~5月くらいに野中先生と関口先生に相談して日程を決定する.2017年度は10月17日~18日に鬼怒川温泉.
2018年度は9月18日~19日に石和温泉,\color{red}2019年度は8月26日~27日に草津温泉\color{black}
場所内容については旅行係の独断で決定してもよい.
ミーティング等でみんなからやりたいこと,行きたい場所などの要望を聞いてそれを旅行会社に伝えると候補を挙げてくれる.
それをミーティングで報告しアンケートなどを採るなどして決める.
晴れ,雨パターンを作っておくとよい.\\
おそらく宴会をやると思うのでその手配もする.\\
バスやホテルの手配は旅行会社がやってくれる.
アクティビティに関して,翌日は基本コアタイムなので2日目にハードなアクティビティ(バレーボール等)があると,
翌日の作業に響く.このことを踏まえて充分に休息が取れる行程が望ましい.
%
\subsection{旅行会社との連絡の取り合い}
早めに電話等で連絡を取る.(昨年は5月ごろから)もしくは旅行会社(昨年は株式会社ジャパトラと京王観光)の人が研究室に一度あいさつに来るのでその後,
話の進め方については基本的には旅行会社とのメールのやり取りに従っていればよい.
%
\subsection{研究室内へのアナウンス}
場所や内容,予算等が決まったらミーティングで適宜報告する.また,予算の徴収も行う(\color{red}2018年度は旅費35000円に対し,
36000円を徴収した.\color{black}).予算は少し多めに徴収する.
また,合宿費用の徴収日を過ぎても連絡なしで支払いが滞っている者がいる場合は先生に相談する.
全額返金でキャンセルできる期限を調べておき,何かあれば(病欠などで長期間学校を欠席している物がいる等)それ以前に先生に相談する.
当日,欠席者が出や場合,旅行会社と相談し,ホテルや昼食の店などに即座に連絡する.時間によっては金が返ってくるため.
%
\subsection{しおりの作成}
サブバージョンを参照して作る.見やすいように工夫する.
%
\subsection{合宿ミーティング}
合宿当日の2,3日前にミーティングを開く(2019年度は夏休み期間のためなし.けど必ずやっておくべき).
合宿の概要をしおりに沿って説明する.(日程,宿舎の設備,持ち物などを明確にする)
%
\subsection{当日の仕切り}
旅行係がしおりに沿って仕切る.
\color{red}
旅行会社はいないので円滑に進行できるように協力を呼び掛ける.
学年ごとに人数確認をする人を決めると効率がいい.
遅刻した人がいる場合はバスの運転手と相談の上遅刻する人を待つか先に出発するかを決める.
遅刻した人を待つ場合は警備員と相談する(駐車場の時間があるため).
先に出発する場合,遅刻した人を途中で拾えるならばバスの運転手と相談の上待ち合わせ場所を決定.
途中で拾わない場合は,遅刻した人と時間を合わせて電車で来てもらう.
お昼など予約している場合はお店に事前に遅れることを連絡すること.
どこかへ寄る場合はその都度明確に口頭とLINEなどで集合時間を知らせる.

あと観光は思った以上に時間がなく全然観光できないかもしれないので,その辺も含めて時間調整を図る.(2019年度は平均45分程観光時間をとっていたが時間が足りなく全部回るのが困難だった)
\color{black}
%
\subsection{費用の清算}
費用の清算を行いミーティングで報告する.費用の詳細がわかったら即時みんなに伝える.
余った場合は後日返金,もしくは宴会費に回す.(2019年度は約5000円余った)
%
\subsection{当日までに}
アメニティや体育館(女子がいる場合は更衣室確認必須)を使う場合は用具の確認.\\
宴会の買い出し.数日前にパーティ係に手伝ってもらって買っておく.(2019年度の場合,室長,副室長などに手伝ってもらった.酒はパーティー係にやり方を学んだ.これらは世田谷で保管.)\\
ビンゴをやる場合はビンゴカードを人数分よりも多めに用意する.\\
支払いが現地の場合もあるので,金額を確認してぼったくられないように気をつけましょう.\\
学生支援センターに合宿申請の書類を提出する(その際,野中先生及び学部主任、大学院の主任の先生のハンコを貰いに行く必要があるので早めに執り行う).
駐車場にバスを駐車する際は警備員へ事前に連絡する必要がある.
また,前日くらいに最終確認オリをやる.
%
\subsection{外出が制限された場合}
2020年度は,新型コロナウイルスの影響により合宿を行えなかった.今後もコロナの影響が続く場合には,変化に合わせた取り組みとしてバーチャル観光やバーチャル旅行といったオンラインツアーが行えると考えられる.
%
\color{black}
\subsection{補足}
\subsubsection{ホテル}
ホテルは駅から近い場所が理想.遠いと遅れてくる人がタクシーやバスを乗り継がなくてはならず大変.
宴会場が遅くまで使えるかどうかの確認.
お酒を持って行く場合に冷蔵庫が利用できるかの確認.
ホテルの近くにコンビニやスーパーがあるとやりやすい.
\subsubsection{アクティビティ}
例年バレーボール大会をやってきているためみんな飽きてきている.2年連続でやることになってしまうため.よってそれ以外の種目を2020年度以降は考えるべき.
\subsubsection{ビンゴ}
2017年度はスマホアプリでビンゴをした.2018年度はビンゴカードを用いた.2019年度はビンゴカードを用いたのと旅館の人にマシンを用意してもらった.
大体一人平均の景品500円程度が目安.2017年では上が3000円,下は50円など幅あり.2018年度も同様.2019年度も同様で,全員に1~2個景品が渡るようにした(あくまで予算の範囲内で.)
\color{black}
\end{document}

\documentclass[12pt]{jsbook}

\usepackage{input}
\usepackage{msethesis}
\usepackage{amssymb,amsmath}
\usepackage{bm}
\usepackage{url}
\usepackage{here}
\usepackage{math}
\usepackage{subcaption}
\usepackage{comment}
\usepackage{float}
\usepackage{ulem}
\usepackage{color}

\begin{document}

\section{発表会係}

発表会係の大まかな仕事内容を列挙します.

\begin{itemize}
  \item 報告会または発表会の日程調整及び告知
  \item 報告会または発表会のタイムプログラムの作成
  \item 発表資料等の管理
  \item 報告会または発表会の準備・司会進行
  \item 報告会または発表会の画面録画
\end{itemize}

上記の通り,発表会係の仕事の多くは報告会または発表会(以下,発表会)に関するものとなっています.
全てを含めると年間で十数回開催されるため,年間を通して小まめに仕事を続ける係と言えます.

\subsection{報告会と発表会の区別について}
報告会は研究の進展を端的に報告する会であるため,
背景など面倒な前置きを省いて構わないことになっています.
この時,前置きを省く際には,発表の論理的な流れに注意して下さい.
対して,発表会は背景やこれまでの研究内容など,自身の研究内容を一通りさらいながら発表します.
このため,前置きで発表時間の半分以上を使うことがざらです.
また,時間管理が発表会毎に微妙に異なるため,注意して下さい.

\subsection{日程調整及び告知}
発表会の大まかな日程は,新年度の研究室オリエンテーションで配布される資料に記載されているので,そちらを参照下さい.
資料に記載された時期となった場合,slack(発表会日程調整チャンネル)で先生と相談し,詳細な日程を詰めて下さい.
基本的には,4週間前をめどに日程調整をお願いします.
初めは4月中に2ヶ月ぐらい先まで計画しておいた方が良いです.
特に4~6月は就活などで予定が入る人が多いため,早めに日程を決めることをお勧めします.
早めに決めると,教室予約や先生との日程調整が楽になります(特に12月は先生方が忙しくなり,日程調整することが難しいので早めが良いです).
日程調整の際,googleカレンダーに不在や就活などの予定書かれているので,人がいない日は避け,開催日の候補を何個か決めてから,先生に予定を伺うようにしてください.
予定を伺うときは〇日の△時限~△時限と日程だけでなく時間まで決めてください.
また,発表会の時間は夜遅くならないようにしましょう(遅くても19時までのほうが良いです).
発表会当日に全員が参加することが望ましいですが,先生との日程の兼ね合いより,全員の参加が難しいことがあります.
その場合は,発表者が発表時間に必ず予定が空いているように調整してください.
発表会は,野中先生と関口先生がいらっしゃることが前提です.
候補日が決まったら,教学課(月~金は17:00まで)に行き,教室を予約しましょう(複数の候補日で先生とのご都合がついた場合は,良い教室が取れる日を選ぶといいです)
教室はなるべく各机に電源がある教室が好ましいです(絶対ではありません).
各机に電源がない教室の場合は,教学課から延長コードを借りましょう.
2025年度の各教室の設備については,NASのws2025/Lab2025/発表会に入れておきますので参考にしてください.
また,ネット接続が安定してできる教室を選ぶようにしてください. \\
詳細が決まったら,全体ミーティングで発表会について全員に告知しましょう.
基本的には,開催日・開始時間と終了時間・発表資料の提出締め切りについて告知すれば良いです.
開催日が複数ある場合は,その日に発表を行う人の学年も告知しておくといいと思います(基本的には,学部生が1日目となります).
また,googleカレンダーにも発表会の予定を入れておきましょう.

\subsection{タイムプログラムの作成}
タイムプログラムの作成はだいたい2週~3週前に行うと良いです.
作成したタイムプログラムは,先生方に確認してもらい,必要に応じて修正を行って下さい.
問題なければ,slack(発表会チャンネル)で共有してください.

\subsubsection{デザイン}
タイムプログラムについては,デザインなどの指定はありません.
2025年度のタイムプログラムをNASのws2025/Lab2025/発表会/タイムプログラムに入れておくので,参考に自作して下さい.
作成する際に,Word・Excel・LaTeXなどのいくつか選択肢がありますが,こちらについても指定はありません.
2025年度のタイムプログラムはExcelで作成したものをWordに貼り付けてpdf化しました.

\subsubsection{研究テーマの記入}
タイムプログラムに作成する際に,発表者の研究テーマを記入します.
このタイトルは2週~3週前までに把握して発表者に確認をしておいてください.
2025年度のタイムプログラムには,一部の発表者の研究テーマが記入されていませんが,基本的には記入しましょう.

\subsubsection{発表順}
タイムプログラムを作成する際に,発表順などは発表会係が決めます.
基本的には,学部生>修士生>博士生の順に行うようにしましょう.
学年内での順番はランダムに決めるといいと思います.
2025年度ではルーレットで順番を決めました.
なお,卒業論文公聴会などの場合は学籍番号順にしましょう.

\subsubsection{時間設定}
発表会の時間設定は,タイムプログラムを作成する際に最も慎重に決定しなければいけない事項です.
発表会での発表者の持ち時間は,発表時間と質疑応答時間の2要素で構成されます.
時間構成は,発表会,報告会ごとに必要とされる時間が異なります.
余裕のある時間設定が望まれるため,少し大目の時間設定を心がけて下さい.
また,休憩時間は1時間から1時間30分毎に10分間取るようにしてください(学年が変わる際には休憩を挟むことが推奨されています).
タイムプログラム通りにいかなかった場合は,発表会係から何時まで休憩かを告知するようにしてください.
昼休憩については,1時間を目処に設定して下さい.

\subsection{準備・司会進行}
発表会当日の大まかな流れと仕事内容を以下に示します.

\begin{enumerate}
  \item 発表会準備
  \item 発表会開始の宣言
  \item 先生方へ開会のあいさつを求める
  \item 発表会の構成についての簡単な説明
\\
\\
-----以下,発表が一通り終わるまでループ-----
  \item 発表者のタイトルおよび氏名の読み上げ
  \item 発表中に,発表時間を知らせるベルを鳴らす
  \item 発表が終わり次第,発表時間を告げる
  \item 質問を受け付け,挙手した人を指名
  \item 学生の質疑を終了し,発表者へ一言挨拶して,拍手 \\
-----以上,発表が一通り終わるまでループ-----
\\
\\
  \item 休憩を挟む際には,休憩時間と再開時間の告知を行う
  \item 先生方へ発表会全体に対する講評を求める
  \item 発表会終了の挨拶
  \item 発表会の片付け
\end{enumerate}

報告会または発表会に応じて必要な作業が増減しますので,臨機応変に対応して下さい.
挨拶の内容などにテンプレートは無いので,事前に自分で考えて準備して下さい.
発表時間には交代等の時間も含まれているのでそれを踏まえて司会進行を行って下さい. \\

\subsubsection{発表資料の提出}
発表会までに,発表者から発表資料を提出してもらう必要があります.
提出方法については,提出場所を事前に作成しておいてください(2025年度はTeamsのLab2025/発表会に書く発表会ごとのフォルダを作成して入れてもらいました).
提出期限は発表会前日(遅くても発表会当日の開始時間の1時間前)までにしておいたほうが良いと思います.
提出された発表資料は,発表会当日に発表用PCにダウンロードしておいてください.

\subsubsection{発表会の準備}
発表会当日,発表会係は1時間前をめどに予約していた教室に行き,発表会の準備をしましょう.
教室のカギは教学課で借りることができます.
発表会の準備としては,プロジェクターとマイクの準備・チャイムの消音・発表資料のダウンロード・ZOOMを用いた画面録画(詳細は「ZOOMを用いた画面録画について」を参照してください)があります.
発表会開始予定時間までには準備を終わらせるようにしましょう. \\

\subsubsection{司会進行}
司会進行の流れと内容について説明します. \\
まずは発表会開始の宣言です.
発表会開始の宣言としては,発表会開始時間になったら,「時間になりましたので,〇月発表会を始めます.」と宣言しましょう.
続いて,「開会のあいさつを〇〇先生,お願いします.」のように,先生方へ開会のあいさつを求めます(職歴順で1人:関口先生).
挨拶をいただいたら,発表会の構成(発表時間・質疑応答時間)について,簡単に説明しましょう.
以上が発表会開始の流れになります. \\
次に発表に移ります.
発表者に前に出てもらい,発表の準備が完了したら,「(題名)と題して,〇〇さん,発表をお願いします.」のように,発表者のタイトルおよび氏名の読み上げを行いましょう.
発表が終了したら,「〇〇さん発表ありがとうございました.ただ今の発表時間は@分@秒でした.それでは,質問のある方は挙手をお願いします.」のように,質疑応答の時間に移る.
質疑応答の時間が終了したら,「時間ですので,質疑応答を終了します.〇〇さん,発表ありがとうございました.」のように,発表者へ一言挨拶して,拍手しましょう.
これを発表者の数だけ繰り返します. \\
最後に,全ての発表が終了したら,「すべての発表が終了しました.それでは,先生方から講評をいただきます.〇〇先生,お願いします.」のように,先生方に発表会全体に対する講評を求めましょう(職歴順で全員:関口先生→野中先生).
講評をいただいたら,「〇〇先生,ありがとうございました.以上をもちまして,〇月発表会を終了します.ありがとうございました.」のように,発表会終了の挨拶をしましょう. \\
以上が,発表会の司会進行の流れと内容になります.

\subsection{発表会の片付け}
発表会が終了したら,速やかに片付けを行いましょう.
片付けの内容としては,発表会で使用した道具の片付け・忘れ物の確認・教室のカギの返却があります.
鍵の返却については,発表会終了時間が17時を過ぎた場合は,教学課ではなく警備室(1号館入ってすぐ左)に教室のカギを返却してください.

\subsection{発表時間}
2025年度の各発表時間の表を示します.
発表時間と質疑応答時間については,発表会の種類や発表者の学年によって異なります.
また,中間発表会については,ポスターセッション形式で発表を行いました.
なお,忘れがちですが最終発表会では発表者は全員スーツ着用なので注意してください.
さらに,修士生以上で学会発表などが近い場合は,学会発表練習として,発表時間と質疑応答時間が他の人とは異なる場合があります.
希望者がいる場合は,相談して時間を決めてください.

\begin{table}[H]
  \centering
  \caption{発表時間一覧}
  \label{table:Method comparison}
  \begin{tabular}{|l|l|c|c|} \hline
    & 学年 & 発表時間 & 質疑応答時間 \\ \hline
    卒論構想発表会 & B4 & 3分 & 2分 \\ \hline
    5月報告会 & B4 & 5分 & 2分 \\ \hline
    5月報告会 & M1~ & 10分 & 5分 \\ \hline
    6月報告会 & B4 & 5分 & 2分 \\ \hline
    6月報告会 & M1~ & 10分 & 5分 \\ \hline
    8月中間発表会 & B4 & 30秒 & 9分30秒 \\ \hline
    8月中間発表会 & M1~ & 8分 & 2分 \\ \hline
    10月報告会 & B4 & 7分 & 3分 \\ \hline
    10月報告会 & M1~ & 10分 & 4分 \\ \hline
    11月中間発表会 & B4 & 30秒 & 4分30秒 \\ \hline
    11月中間発表会 & M1~ & 30秒 & 4分30秒 \\ \hline
    12月報告会 & B4, M1 & 30秒 & 9分30秒 \\ \hline
    12月報告会 & M2~ & 10分 & 4分 \\ \hline
    事例研究中間発表会 & B3 & 3分 & 2分 \\ \hline
    事例研究最終発表会 & B3 & 5分 & 3分 \\ \hline
    卒業論文公聴会 & B4 & 10分 & 4分 \\ \hline
  \end{tabular}
\end{table}

\subsection{合同発表会}
2025年度では,8月中間発表会と11月中間発表会において,ロボティクス・メカトロニクス研究室と合同で発表会を行いました.
発表会が行われる1ヶ月程前にから,ロボティクス・メカトロニクス研究室の代表者とslackで連絡を取り合いながら,発表会の日程調整を進めるようお願いします.
また,提出資料と研究テーマについては,slack(発表会チャンネル)で共有してもらい,タイムプログラムを作成してください.
なお,合同発表会の際には,開会のあいさつは職歴順で1人:藪井先生,講評は職歴順で全員:藪井先生→関口先生→佐藤先生→野中先生となります.


\subsection{道具の管理}
発表用PC・ベル・ポインターの管理が任されます.
引継ぎの際,新しく発表会係となった人が管理をして下さい.
発表用PCについては,充電しながら発表会を行いましょう.
ベルについては,現在使用していません.
ポインターについては,あまり充電しなくても使えますが,充電しておいたほうが安心です.

\subsection{ZOOMを用いた画面録画}
発表会では,ZOOMを用いた画面録画を行います.
ZOOMのURLは先生(基本的には関口先生)に作成してもらうようにしてください.
作成してもらったZOOMのURLは,googleカレンダーに記載されます.
発表用PCでZOOMに入室したら,ホスト権限をもらい,画面共有(画面を拡張して画面2を共有しましょう)と録画(クラウド)を開始してください.
また,発表用PCのミュートを解除しておきましょう.
休憩に入ったら,録画を一時停止してください.
発表会が終了したら,録画を停止してZOOMから退出してください.
発表用PCの他に,確認用として個人のPCでZOOMに入室しておくと安心です.

\subsection{Time Keeperを用いた時間管理}
発表会の時間管理には,Time Keeperというサイトを用いています.
このサイトでは,開始時間・ベルが1回鳴る時間・ベルが2回鳴る時間・ベルが3回鳴る時間を設定できます.
開始時間は0:00に設定し,ベルが1回鳴る時間は発表時間の半分の時間に設定してください.
ベルが2回鳴る時間は発表時間の終了時間に設定してください.
ベルが3回鳴る時間は質疑応答時間の終了時間に設定してください. \\
発表会当日,発表会係は個人用PC(通知は切っておきましょう)でこのサイトを開き,時間設定をしてください.
発表者が発表を開始したら,Time Keeperの開始ボタンを押してください.

\subsection{最後に}
最後に,全体を通した注意点として,発表会について先生と相談する際には,先生に具体的な意見を仰ぐのは極力控えて下さい.
日程調整の相談などを行う際にも,先生に日時を決めていただくのではなく,自身で具体的な日時を提示するようにしてください.
また,発表会を行う教室が縦長の場合は,聴講者たちを前の席に着席させるようにしてください.
発表会で用いた資料などはNASのws2025/Lab2025/発表会に入っているので参考にしてください.
引継ぎ内容は以上となります.

\end{document}

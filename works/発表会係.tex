
%\usepackage{color}
%\usepackage{ulem}
%//\newcommand{\test}{waya}
\documentclass[12pt]{jsbook}
%
\usepackage{input}
\usepackage{msethesis}
\usepackage{amssymb,amsmath}
\usepackage{bm}
\usepackage{url}
\usepackage{here}
\usepackage{math}
\usepackage{subcaption}
\usepackage{comment}
\usepackage{float}
\usepackage{ulem}
\usepackage{color}
\begin{document}
%\maketitle
\section{発表会係}
\color{black}

発表会係の大まかな仕事内容を列挙します.


\begin{itemize}
  \item 発表(報告)会日程の調整及び告知
  \item 発表(報告)会プログラムの作成
  \item 発表(報告)会の運営・進行
  \item (学会)発表練習の司会
  \item 発表会用の道具の管理
\end{itemize}

上記の通り,発表会係の仕事の多くは発表会および報告会に関するものとなっています.
仕事内容は基本的にシンプルであり,箇条書きした以上の内容は有りません.
その代わりに,発表会,報告会,発表練習など,全てを含めると年間で十数回開催されるため,
年間を通して小まめに仕事を続ける係と言えます.発表の様子は撮影しますがこれははアルバム係が行います.\color{black}

{\bf [報告会と発表会の区別について]}
報告会は研究の進展を端的に報告する会であるため,
背景など面倒な前置きを省いて構わないことになっています.
この時,前置きを省く際には,発表の論理的な流れに注意して下さい.
対して,発表会は背景やこれまでの研究内容など,自身の研究内容を一通りさらいながら発表します.
このため,前置きで発表時間の半分以上を使うことがざらです.
発表会係は,この点に注意しながら次回の発表(報告)会を告知して下さい.
また,タイムスケジュール管理が発表(報告)会毎に微妙に異なるため,その点にも注意して下さい.

続いて各仕事内容の詳細を記載します.

\subsection{発表(報告)会日程の調整及び告知}
発表会および報告会の大まかな日程は,新年度の研究室オリエンテーションで配布される資料に記載されているので,そちらを参照下さい.
資料に記載された時期となった場合,ミーティングもしくはメールなどで先生と相談をし,詳細な日程を詰めて下さい.
以前の発表会プログラムがサブバージョンにありますので,そちらも参考にしながら,
4週間前をめどに日程調整および準備をお願いします.
初めは4月中に2ヶ月ぐらい先まで計画しておいた方が良いです.特に4~6,7月は就活などで予定が入る人が多いと思うので,早めに決めておくと発表会係だけでなく,他の人としても発表日以外で面接などの予定が入れやすくなると思います.
計画する際、4年生だけ発表か、院生も発表かを過去の発表会プログラムから調べ(PC発表練習などでM1生が別日になることもある為)、その月の発表時間よりおおよそ発表会全体で掛かる時間を把握してから,先生と予定を詰めていってください.不在表等より発表可能な時間の候補を何個か決めてから,先生に予定を伺うようにしてください.(予定を伺うときは〇日の△時限~△時限と日程だけでなく時間まで決めてください.)
発表時間の候補を決めるときは,不在表や就活の人の予定がgogleカレンダーに書かれているので,人がいない時間はさけるようにしてください.(発表会時に不在の人がいないことが望ましいですが,先生との日程の兼ね合いより,全員参加が難しいことがあります.その場合は,発表者が発表時間に必ず予定が空いているように計画は立ててください.)
発表会は基本的には,野中先生と関口先生がいらっしゃることが前提としてありますが,鈴木先生も来られるほうが望ましいと思うので,野中先生,関口先生の他に鈴木先生にも予定を伺ってください.その際にメールで3人の先生方に一斉送信して予定を伺うと効率的に行えます.
また複数の候補日で先生とのご都合がついた場合は,予約できる教室を支援センターで調べ,いい教室が取れる日を選ぶといいと思います.
日程の調整は,前期は一か月前に,後期は9月中(夏休み後直後)に12月までのものを全て決定,2月のものは1月に決定しました.早めに決めると,教室予約,先生との日程調整が楽になります(特に12月は先生方が忙しくなりあとになって日程調整することが難しいので早めが良いです.).
詳細が決まりましたら,ミーティング,メールリスト(googleメール),googleカレンダーなどで早め早めの告知を心がけて下さい.
また,
%鈴木先生(ksuzuki@tcu.ac.jp)への連絡を忘れないようにして下さい.メーリスには鈴木先生は含まれていません.
サブバージョンの「lab>発表会プログラム」に過去の発表会プログラムがありますので,そこに毎回発表会プログラムを上げてください.
%subversion,maillist,workspace(lab_data)にあげる?,pdf推奨
%居る人は基本発表

また,日程調整の際に,会場予約などを行う必要が出てきます.
先生や聴講者はスライドをonedriveなどから落とすため,ネット接続は必ずできる環境の教室を予約してください.
また,会場の選定は,なるべく電源が各机にある教室(1号館2階ただしPCがある教室は除く)が好ましいです.(絶対ではありません)
会場の選定予約は基本毎回発表会係が会場予約を行ってください.
学習支援センターでの手続き自体は簡単なので出来るようにしておいた方がいいと思います.
8月や12月,2月は他の研究室も発表会を行うため教室の予約が取りにくくなります.なるべく早く予約しましょう.
学習支援センターは月~金は17:00まで,土曜は午前中,日曜休みなので手続きの際は注意してください.
新年度のオリエンテーションの日程に従って,日にちを決めてください.学部生,院生,博士全ての人が原則発表します.また,院生以上で学会発表などが近い場合は発表練習に替えることが出来ます.発表会プログラム作成時に発表者に聞いてください.

また,新年度のオリエンテーションで変更がありましたらそちらを優先してください(2017年度では9月に報告会が増えました).

2021年度では,対面での発表のほかにZOOMを用いたリモートでの発表を行いました.
発表会のZOOMのルームは関口先生が作成してもらうことになっていると思うので事前に連絡し研究室全体に周知してください.
また,対面での発表では,発表場所に世田谷キャンパスの教室やコンファレンスホール(総研の2階)を利用していました.
コンファレンスホールの予約は先生しか行えないため日程が決まり次第先生に予約のお願いをするようにしてください.(2021年度は関口先生にお願いしていました.)

対面で報告会や発表会をする場合には,会場のWifiのアクセスポイントとパスワードを簡単にまとめたスライドを用意し,発表会前に周知しておくことが望ましいです.
また,コンファレンスホールを利用する際には以下の点に注意してください.
\begin{itemize}
  \item ネット環境がないため,実験室からLANを伸ばしてネット環境を整える必要があります.席替えなどで長いLANケーブルが利用されないようにしてください.
  \item カンファレンスホールには備品として延長コードが3本あります(2021年度現在).誤って研究室に持って行かないように気を付けてください. 
\end{itemize}

%通常の報告会,発表会の開催頻度については学部生は1ヶ月に1回の発表があります.
%後期は卒論の内容について詳しく詰めていく必要があるので,B4を半分に分けて交代で月二回発表,報告書提出をローテーションする形になります.
%修士の先輩方の発表については,学会等での発表を行う人を除いて2ヶ月毎に報告会または発表会を行います.

\subsection{発表(報告)会プログラムの作成}
{\bf [デザインについて]}サブバージョンの発表会プログラムを参考に自作して下さい.
この際,レイアウトの指定などは特にありません.
制作を行う際にWord,Excel,Texなど,いくつか選択肢がありますが,
こちらについても特に指定はないため,先輩などと相談しながら,自身のセンスに従って制作をお願いします.
今後の扱いやすさを考慮した際に,Texに軍配が上がる可能性があります.
作成はだいたい2週~3週前に行うといいです.WordのテンプレをWork2017/niwa/発表係引継ぎ用に入れておきます.


{\bf [発表タイトルについて]}発表プログラムに作成するに当たり発表者の研究タイトルを記入します.このタイトルは2週~3週前までに把握して発表者に確認をしておいてください.
2020年度はonedrive上のWork2020/ShareにExcelファイルを入れておき,そこに記入してもらうことで把握していました.


{\bf [発表順について]}発表会プログラムを作成する際に,発表順などは発表会係が決めます.
順番の決め方に指定はないため,先輩や同輩と相談しながら決定して下さい.
また,中間発表会など,複数の学年が発表を行う場合は,学年ごとに,学年が低い順に発表を行います.
学年が変わる際には,休憩を挟むことが推奨されています.
特に順番に決まりはないです.前後の予定によって変化します.基本は学年が低いほうがいいです.
発表順は適当にExcelでランダム関数をだして並び替えが楽です.このExcelのテンプレもWork2017/niwa/発表係引継ぎ用に入れておきます.また,このテンプレの中に上記で述べたタイトルを書く欄を設けています.

{\bf [複数の学年が同日に発表する場合]}
中間発表会などでは複数の学年が同じ日に発表することがあります.その際に開会の挨拶や講評はその発表会で一回行うことに成ります.つまり,同じ日の場合,最初の学年の前に開会の挨拶をし,最後の学年の最終発表者が終わった後に講評を纏めて行う形に成ります.
発表順は上の記述を参考にしてください.


{\bf [時間設定について]}発表会のタイムスケジュールの設定は,
発表会プログラムを作成する際に最も慎重に決定しなければいけない事項です.
発表会での発表者の持ち時間は,発表者の発表時間,学生の質疑応答時間(学部生からの質問と院生からの質問で分けていました.)
先生の質疑応答時間(2020年度:先生の質疑時間は,発表者が多くなったためあまり行いませんでした.)の3要素で構成されます.
これらの時間構成は,発表会,報告会ごとに必要とされる時間が異なるため,
その都度先生や先輩と相談をする必要があります.
プログラム作成では,余裕のある時間設定が望まれるため,少し大目の時間設定を心がけて下さい.(まず時間通りに終わることはありません)

過去のプログラムを参考にしてください.

上記はあくまで目安であり,その都度先生と相談しながら決定して下さい.
%また休憩時間の設定についてですが,90分を目安に10分程度の休憩をはさんで下さい.
4人~6人で一回休憩をはさんでください.
この時,注意するのは,休憩時間には「参加者の休憩」と「先輩からの発表フィードバックなどを受ける時間」の2つの意味合いがある点です.なので,休憩時間はなるべく10分くらいは取るようにしてください.(発表会時は時間が長引く為,プログラム通りにいかなかった場合は発表係から何時まで休憩かを告知する.)
%特に中間発表会など長丁場になりそうな場合は長めに設定するようにして下さい.
お昼休憩については,1時間を目処に設定して下さい.

プログラムが完成しましたら,実際に印刷して先生方に目を通してもらって下さい.
(2020年度では,Slackやgoogleメールの高機能研グループメールにて発表会日程の告知とともに添付しました.)
また,稀にですが発表会直前に先生の都合がつかなくなることがあります(会議とか).その為,発表会の2週間くらい前にもう一度発表会当日の時間帯で予定がないかを確認することが望ましいです.もし変更する場合はその都度プログラムを更新し,メールやSlackで告知してください.
%

\subsection{発表(報告)会の運営・進行}
発表会係は発表会を指揮,運営する係です.
必要があれば同輩に手伝いを頼んで下さい.(事例研究最終発表で全員に当日行う準備は教えたからできるはずです.)
野中先生,関口先生はonedriveからPCで見るので,コピーできるように早めにパワポを提出させてください.(2018年度は発表1時間前までに提出してもらってました.)鈴木先生は印刷してパワポを渡してください.(発表(報告)会当日に鈴木先生が来るのか油圧班の人に聞くか自分でメール等で事前に確認してください.いらっしゃるようなら印刷してください.
司会進行は発表会係がメインで行いますが,
会場設営や発表時間を知らせるベルなどは積極的に手伝いをお願いして下さい.
パワーポイントの提出場所を提出フォルダを事前に作っておいてください.(2021年度はonedrive上のlWorkspace2021/lab2021/発表会 というフォルダーに入れてもらいました.)
また,12月中間発表会よりB3に手伝いをお願いしてください.

発表(報告)会当日の大まかな流れと仕事内容を以下に示します.

\begin{enumerate}
  \item 発表会準備(会場の鍵の貸し出しやプロジェクター,Wifiのアクセスポイントとパスワードの周知,マイクの準備,会場のチャイムの消音,先生用の発表資料準備)
  \item 発表会開始の宣言
  \item 発表会の構成についての簡単な説明
\\
\\
-----以下,発表が一通り終わるまでループ-----
  \item 発表者のタイトルおよび氏名の読み上げ
  \item 発表中に,発表時間を知らせるベルを鳴らす
  \item 発表が終わり次第,発表時間を発表者に告げる
  \item 質問を受け付け,挙手した人を指名
  \item 適当なタイミングで学生の質疑を終了し,先生の質疑へ
  \item 先生の質疑が終了したら,発表者へ一言挨拶して,拍手  \\(2018年度は先生の質疑は作りませんでしたが作っても構いません.)
\\
-----以上,発表が一通り終わるまでループ-----
\\
  \item 休憩を挟む際には,休憩時間と再開時間の告知を行う
\\
  \item 先生方へ発表(報告)会全体に対するコメントを求める
  \item 発表会終了の挨拶
\end{enumerate}

発表(報告)会に応じて必要な作業が増減しますので,臨機応変に対応して下さい.
挨拶の内容などにテンプレートは無いので,事前に自分で考えて準備して下さい.
発表時間には交代等の時間も含まれているのでそれを踏まえて質疑等の進行を行って下さい.
発表会では原則1人1回以上質問します.
定期的に伝えないと質問しない人が出てくるので適度に刺激するようにして下さい.\\

例)\\
最初:これより〇月中間発表会を始めます.発表時間@分,質疑時間@分で行います.\\
発表時:(題名)と題しまして,@@さん,発表をお願いします.\\
質問時:**さんどうぞ.\\
時間になったら:@@さん発表ありがとうございました.\\

\subsection{発表時間}
2021年度の発表時間です.
\begin{table}[h]
  \centering
  \caption{発表時間一覧}
  \label{table:Method comparison}
  \begin{tabular}{|l|l|c|c|}\hline
&学年&発表時間&質疑時間\\ \hline
卒論構想発表会 &B4&3分&2分 \\ \hline
5月報告会 &B4&5分&5分 \\ \hline
5月報告会 &M1~&10分&5分 \\ \hline
6月報告会 &B4&5分&5分 \\ \hline
6月報告会 &M1~&10分&5分 \\ \hline
8月中間発表会 &B4&7分&5分 \\ \hline
8月中間発表会 &M1~&10分&5分 \\ \hline
9月中間発表会 &B4&7分&5分 \\ \hline
9月中間発表会 &M1~&10分&5分 \\ \hline
10月報告会 &B4&7分&5分 \\ \hline
10月報告会 &M1~&10分&5分 \\ \hline
11月報告会 &B4&7分&5分 \\ \hline
11月報告会 &M1~&10分&5分 \\ \hline
12月中間発表会 &B3&3分&2分 \\ \hline
12月中間発表会 &B4~&10分&5分 \\ \hline
事例研究最終発表会&B3&4分&3分 \\ \hline
卒業論文公聴会&B4&10分&10分 \\ \hline
  \end{tabular}
\end{table}

また,忘れがちですが最終発表会では発表者は全員スーツ着用なので注意してください.

\subsection{学会発表練習の司会}
発表会係の発表会以外の仕事内容ですが,発表練習の司会があります.
院生などが学会などで発表を行う際,事前に研究室で練習発表を行います.
この際も,司会を発表会係が行います.
基本的には前節の発表の流れを参考に進行すれば問題ありませんが,事前に発表時間の構成等は先輩に確認する必要があります.場合によってはプロジェクターなどを学園支援センターであらかじめ予約して借りる必要があるのでなるべく早めに確認してください.
発表練習では発表時間を記録し,質疑応答に入る前に発表者に伝えて下さい.

2017年度では研究室で行うときと教室で行う時がありました.どちらにするかはあらかじめ発表練習する先輩と相談して決めてください.また,学会発表の場合もプログラムを作成してgoogleメールとミーティングにて告知をしてください.


\subsection{PC発表練習の運営について}
M1の先輩は前期に自分の研究する内容を学科全体で集まりポスター形式で発表する機会があります.本研究室では毎年PC(ポスターセッション)に向けて発表練習を行います.発表係はPC発表練習の司会進行とその準備を行います.

まず会場ですが,研究室5階で行う形になります.会場の形式は,フィールドに2か所,学生室に1か所の計3か所同時に進行をします.
従って,プロジェクターとスクリーンも3つづつ必要となります.研究室5階にはプロジェクターとスクリーンが一つしかない為,学習支援センターで事前に予約し,借りる必要があります.学習支援センターに借りに行く際は人数が必要ですので,声をかけて手伝ってもらいましょう.

次に運営進行ですが3か所同時に行うため,発表係の行うことは会場全体のタイムキーパーに成ります.タイムキーパーとして全体に経過時間を伝えるために,ベルにて1鈴2鈴3鈴を鳴らしていきます.この際に,2鈴でベルを鳴らした後にそれぞれの発表場所にまわり,「次で最後の質問にしてください」と声をかけるとスムーズに終わることが出来ます.ポスター形式の発表なのでなかなか質問が切れないことにより経過時間が大幅に伸びることが考えられます.(2018年では質問がなかなか終わらずに予定していた時間の1.5倍以上掛かることがありました.)発表終了後はそれぞえのグループに他の発表場所に向かうように指示をしてください.3回同じ発表をして聴講者が全ての発表を聞き終えたら,発表者が入れ替わりそれを繰り返します.

B4生は発表を聞く際にそれぞれ発表時間と質疑のメモを取る仕事があります.発表場所は3か所なのでB4もそれに応じれ別れます.そこで発表係が事前にB4だけでもグループ分けをしてそれぞれのグループ内で役割をPC発表前に決めるように声掛けをするようにしてください.


B4生も学校説明会に対するポスターによる発表と研究室紹介をする為の発表練習がありますが,PCの発表練習と同じ要領で行えばいいと思います.



\subsection{道具の管理}
発表会係はベル,ポインター,ストップウォッチの管理が任されます.
引継ぎの際,新しく発表会係となった人が管理をして下さい.
道具を管理する際に注意して欲しいのが,ポインターの電池残量です.
現在ポインターには単4エネループを使用しています.だいたい一回の発表(6時間程度)で電池切れるっぽいので注意してください.発表(報告)会前日には発表用PCと電池を充電しておくのがオススメです.楕円形のポインタが発表会用,細長い棒状のものが関口先生から預かっているポインタです.両方とも使って大丈夫です.
特に,院生の先輩方がポインターを学会へ持っていく場合などは細心の注意をお願いします.

ストップウオッチは2016年度に購入しました.が万が一壊れてしまった場合は早急に備品係に言って購入してもらってください.\color{black}
また,この際,ベル付きタイマーの購入や,PCなどでベル付きタイマーアプリの導入するなどを検討してみてください.
2019年度には,充電池を新たに3本購入していただいたので,充電池は合計5本あります.

\subsection{他の係の人との連携}
室長と副室長は先生方の予定をいち早く把握していると思うので基本はどちらかと相談することになると思います.

発表会の日程を先生方に相談後,ある程度行う日程が分かったらパーティー係の人に行うであろう日程を伝えておくとパーティー係の人が打ち上げの会場予約などがしやすいと思います.

発表会の日程調整の際にゼミを行う週には基本発表会は行いません.従って,ゼミの日程を調整する副室長とも先生の空いている日程を共有するといいと思います.また事前に発表会を行う日を伝えておくと日程調節もしやすいと思います.

発表会の日時の予定を立てる際に不在表を元に決定します.そこで,前々期と前後期の変わり目である6月あたりに行う発表会や9月に12月の予定を決める際は,次のタームの不在表が必要となる場合があります.そこで早めに備品係の人に不在表を作成して貰うように頼むといいと思います.

発表会をZOOMで行う場合,発表の録画をアルバム係が行うことになっています.係で分担して発表会を進行していきましょう.


\subsection{ATACSについて}
ATACS(毎年参加している学会のこと)では,研究室で持ち回りでチェアマン(発表の時間管理)をします.微妙に形式が違うので注意.動画がworkspaceに残っているみたいなので参考にしてください(ATACS係に聞いてみると,どこにあるかわかると思います).2017年度はチェアマン3人(司会,タイムキーパー,マイク係)だったので発表係のほかに2人チェアマンをB4の中からあらかじめ決めてください.チェアマン以外のB4は基本的にサクラ(質問者がいない時に代わりに質問をする人)になります.ベルとストップウォッチは会場にあると思うが念のため持っていきましょう.発表用PCは絶対に持っていきます.

\subsection{合同報告会・発表会について}
2020年度からロボティックライフサポート研究室と合同で発表会を行います.発表会や報告会が行われる1ヶ月前にロボティックライフサポート研究室の代表者とメール等で連絡を取り合いながら,発表会の日程を決めるようお願いいたします.

\subsection{ZOOMでの開催について}
2020年度から新型コロナウイルスの影響により,教室での発表会の実施ができなくなってしまうことがあります.その際,発表会はZOOMでの開催になります.日程が決まった際には,先生にZOOMのURLを作成してもらうようお願いいたします.

\subsection{最後に}
最後に,全体を通した注意点として,発表会について先生と相談する際には,
先生に具体的な意見を仰ぐのは極力控えて下さい.
日程調整の相談などを行う際にも,先生に日時を決めていただくのではなく,
自身で具体的な日時を提示するようにしてください.

発表会で用いた資料(Wordテンプレや題名記述エクセル)や当日の流れなどを示したファイルを研究室サーバーのWorkspace/work2018/nishioka/発表係資料に残しておきますので参考にしてください.
引継ぎ内容は以上となります.



\end{document}

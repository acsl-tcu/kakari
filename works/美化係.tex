\usepackage{input}
\newcommand{\Red}{\color{red}}
  \newcommand{\test}{waya}
\begin{document}
\maketitle
\section{美化係}
\subsection{チェックシート}
研究室を最後に退出する人には,戸締りや電気などをチェックしてもらっています.
そのためのチェックシートがありますのでそれを月初めに印刷して,研究室のロッカー上,425総研に置いてください.
チェックシートは2種類あります(「works/美化係/美化チェック表」フォルダに入っています).学校に提出するものなので注意してください.
設置時には,ボールペンも一緒に置いておきましょう.昨年度は秘書さんに貸してもらっていました.
研究室で余っているものがあればそれを置いてもいいと思います.
最初の内に,12ヶ月分のチェックシートを作ってしまうと楽です.

\subsection{日々の掃除}
研究室全体のルールで,ごみ出しは毎日行います(係の仕事ではないです).ピカチュウが席の島ごとに毎日動くので,ピカチュウの位置を確認してごみ出しを行ってください.
ごみを出したらピカチュウを次の島に移動させることも忘れないようにしてください.

週1回,学生室,実験室,フリースペース,野中先生・関口先生の部屋の掃除を行ってください.
例年,全体ミーティング後に行っていますが,それ以外の日に行う場合は掃除前までにslackのリマインダー等で連絡しましょう.
主な掃除としては,床の掃除機がけ,机をふく,モニターの埃取り,空気清浄機のフィルター掃除等です.
また,2~3週間に1回は実験室のマットの張り替えを行うようにしてください.
空気清浄機フィルタのゴミを掃除機で吸うのも忘れずに.
{\bf{※長期間研究室を空けるときは,必ず空気清浄機の水抜きを忘れずに.分解して乾燥させること.これをしないとカビが生えます.}}

\subsection{大掃除}
長期休暇などの前にある大掃除では,普段の掃除に加え,棚・本棚の整理整頓,エアコンのフィルター掃除,ブラインド・窓の掃除,フィールドの掃除など普段行わない場所を掃除も掃除するようにしてください.
掃除の担当や時間割などを決めておくとスムーズに進みます.
日時は発表会後の長期休暇に入る直前に入れると良いです.日時を決定したら早めに告知を行いましょう.
2025年度の大掃除は8月の中間発表の翌日と2月末に行いました.大掃除では備品管理・廃棄などがあるので,引継ぎを兼ねてM0を中心に行うと良いと思います.
年度末大掃除の役割分担用の表を「works/美化係/大掃除2025.xlsx」に置いていますので参考にしてください.

大掃除でしか使わない掃除用具などもあるので,「works/美化係/掃除用具」フォルダに掃除用具のリストと画像をまとめていますので参考にしてください.

\subsection{ごみの分別}
ごみの分別としては,燃えるごみ,燃えないごみ,ペットボトル,缶・ビン,金属,段ボールに分けて捨ててください.
本やカタログなどを捨てる際にはビニールひもで縛り,階段とトイレの間のごみ箱の横に持って行ってください.
総研の場合も同様です.

\subsection{掃除用具の購入}
掃除用具等の購入は基本的に秘書さんに相談してください.
物によっては先生や備品係とも相談する必要があります.
2025年度は掃除機を購入しましたが,この時は関口先生に相談して購入しました.


\end{document}

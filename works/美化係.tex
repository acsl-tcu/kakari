\usepackage{input}
\newcommand{\Red}{\color{red}}
  \newcommand{\test}{waya}
\begin{document}
\maketitle
\section{美化係}
\subsection{チェックシート}
研究室を最後に退出する人には,戸締りや電気などをチェックしてもらっています.
そのためのチェックシートがありますのでそれを月初めに印刷して,世田谷キャンパス4階,5階,総合研究所学生室に貼ってください.
また,閉鎖障害のチェックシート,火気関係のチェックシートも同様に貼ってください.こちらは学校に提出するものなので注意してください.
これらのテンプレは美化表テンプレ.xlsxとして置いておきます.
最初の内に,12ヶ月分のチェックシートを作ってしまうと楽です.

\subsection{日々の掃除}
週1回,総合研究所学生室と実験室,本館1階,2階の野中先生が所有する部屋の掃除を行ってください.掃除前までにslackのgeneralに連絡するのも忘れずにしてください.
主な掃除としては,床の掃除機がけ,廊下の箒がけ,机をふく,空気清浄機のフィルター掃除等です.
また,2~3週間に1回は実験室のマットの張り替えを行うようにしてください.
空気清浄機フィルタのゴミを掃除機で吸うのも忘れずに.
2019年度では総研に完全移設をしたため掃除の際に人員が余ってしまいました.そのため,2020年度では2~3班に分割して週ごとに掃除を行う班を変えていくことをお勧めします.
掃除終了後に,ごみは下にあるごみ回収の場所に持っていってください.
※忙しい時期だと,いつもの日程で掃除を行なえない場合があります.そのときは,別日に掃除をするようにしてください.研究室メンバーへの連絡も忘れずに.
※燃えるごみは特に3~4日で満タンとなってしまうため,掃除の日でなくても捨てるようにする.
{\bf{※長期間研究室を空けるときは,必ず空気清浄機の水抜きを忘れずに.分解して乾燥させること.これをしないとカビが生えます.}}

\subsection{大掃除}
長期休暇などの前にある大掃除では,普段の掃除に加え,棚・本棚の整理整頓,エアコンのフィルター掃除,ブラインド・窓の掃除,フィールドの掃除など普段行わない場所を掃除も掃除するようにしてください.美化表テンプレ.xlsxに掃除箇所のチェック表がありますのでそれを参考に掃除を行ってください.
掃除の担当や時間割などを決めておくとスムーズに進みます.(大掃除_~月.xlsx 参照)
日時は発表会後の長期休暇に入る直前に入れると良いです.日時を決定したら早めに告知を行いましょう.
昨年度の大掃除は8月の中間発表の翌日と1月の修士生の中間発表の翌日に行いました.1月の大掃除では当日にM0が授業があった関係上,M0に世田谷の大掃除を任せて,それ以外の学生で総研の大掃除を行いました.

\subsection{ごみの分別}
ごみの分別としては,燃えるごみ,燃えないごみ,ペットボトル,缶・ビン,金属,段ボールに分けて捨ててください.
本やカタログなどを捨てる際にはビニールひもで縛り,世田谷の場合は下にあるごみ回収の場所,または階段の踊り場にあるゴミ捨て場に持って行ってください.総研の場合は,学生室の裏側にある倉庫に持って行ってください.
また,備品を捨てることになった際は,3月中に粗大ごみの回収を学校が行いますので,その時に捨ててください.

2019年度に引っ越しを行い,総研では新しく机やいす等を購入しました.その際梱包で使用されていた段ボールが大量に出ました.総研では段ボールの回収が一ヶ月に一度しかなく,回収される日にちもいつも同じではありません.また,出す場所は学生室他の研究室が出せる分を残したうえでの裏にある倉庫でスペースが決まっている上,他の研究室との共用の場所となっています.そのため,まだ本館2階に段ボールの山が残っています.申し訳ないですが,事務室に回収日を聞いたうえで回収日の1.5週間くらい前までに,他の研究室が出せる分を残して捨ててください.なお,回収日の1週間前に回収されることもあるので動向に注意しましょう.回収される時間は午前8次頃と聞いております.

\subsection{掃除用具の購入}
掃除用具は管理係と相談して購入して下さい.
消耗品としてはごみ袋,食器用洗剤,ハンドソープ,掃除シートがあります.
掃除シートは尾山台のセイジョーで販売,その他のものは学校の購買部で購入できます.

3月中に商品を購入する場合は,3月中に研究室の予算の決済などがありますのでその日程に注意して購入するようにしてください.



\end{document}

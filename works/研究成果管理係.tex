\usepackage{input}
  \newcommand{\test}{waya}



\begin{document}
\maketitle
\section{研究成果管理係}
研究成果管理係の仕事は,大きく次の3つです.
\begin{enumerate}
  \item 誰かが学会に参加する度に論文やエビデンスの保管,投稿情報の記載などを行っているかの確認
  \item 昨年度行われた学会発表の内容を英語でまとめたスライドの作成
  \item 卒論$\cdot$修論の保管が行われているかの確認
\end{enumerate}
研究成果管理係は例年M0が担当するので,分からないことがあれば元係だった先輩に聞いてください.
なんなら学会に参加した方であれば大体の流れは把握していると思うので,学会に参加した先輩に聞いてもいいと思います.

\subsection{誰かが学会に参加する度にやること} \label{subsec:conference}
\textcolor{blue}{teams} > Lab20oo > Conference > 学会に関する流れ.xlsx に学会発表に際しての流れが記載されています.
研究成果管理係としては,学会に参加した人が「学会に関する流れ.xlsx」下部の事務作業の部分をきちんと行っているかの確認を行います.
確認事項は以下の通りです.
\begin{enumerate}
  \item 予稿集に掲載された自身の論文を \textcolor{blue}{teams} > Published > 20oo04-20oo03 へ格納しているか
  \item \textcolor{blue}{teams} > Published > domestic.xlsx, international.xlsx, journal.xlsx に投稿情報を記載しているか
  \item エビデンスとして,自身の論文のPDF,tex, 生データ, program,turnitin結果$^{\ast}$を \textcolor{red}{NAS} > ws20oo > Lab20oo > Papers > 学会名 > ローマ字個人名 に保存しているか
  \item 学会の予稿集を \textcolor{red}{NAS} > ws20oo > Lab20oo > Conference に格納しているか
\end{enumerate}
特に,3つ目の\textbf{エビデンスの保存は最重要}です.
年度末になると全体ミーティング等で先生が度々仰ると思いますが,世の中には自身の論文$\cdot$研究の不正を疑い,エビデンスを要求してくる人がいます.
こうした場合に自分自身を守るため,また論文の内容を再現するために必要なものなので,きちんと保存されているか確認してください.
1~4の作業を年度末にまとめてやると大変なので,誰かが学会に参加する度に確認することが大事です.
特に各学会の予稿集$^{\ast\ast}$は,保存期限が設けられているので早めにやらないとダウンロードできなくなるので注意が必要です.

{\footnotesize $\ast$ turnitinとは,論文の剽窃チェックをするためのサービスです.研究室では学会発表の際に提出する論文をturnitinにかけて,その結果をエビデンスとして保存することになっています.}

{\footnotesize $\ast\ast$ 予稿集とは,学会発表の内容が掲載された冊子(データ)のことです.その学会の全ての論文が記載されており,容量が大きいのでNASに保存します.}

\subsection{昨年度対外発表のまとめスライド作成}
昨年度行われた学会発表の内容を\textbf{英語で}まとめたスライドを作成します.
研究室には海外からのお客さんがちょくちょく来られるので,その方々に研究室の研究内容を知ってもらうために,毎年作成しています.
(ただ2025年度時点では,ここ数年スライドを作成したものの実際に見せる場面は無いようです...)
後期が始まるまでには完成させてください.
1人で作成するのは大変なのでB4生に割り振るのをおススメします.
2025年度はM0に割り振って作成してもらいました.
\textcolor{blue}{teams} > 研究室共通 > 研究成果管理係 > 英語スライドテンプレート.pptx にスライドのテンプレートが保存されているので,それを使用して作成してください.
スライドの内容は以下の通りです.
\begin{description}
  \item[Title:] 発表タイトル,発表者名,学会名,記載ページ,開催地(1ページ)
  \item[Introduction:] 導入,研究背景,問題設定など(1~2ページ)
  \item[Method:] 手法(何ページでも)
  \item[Result:] シミュレーション/実験結果(何ページでも)
\end{description}
\textcolor{blue}{teams} > 研究室共通 > 研究成果管理係 > 20oo に過去年度のスライドが保存されているので,それを参考にしてください.


\subsection{卒論$\cdot$修論の保管確認}
B4とM2は年度末に概要集と学位論文を提出することになっています.
概要集とは2コラムで書かれた論文の要約のことで,B4は2ページ,M2は6ページで書くことになっています.
LaTexテンプレートは学科側から配布されますが,例年配布されるのが遅いので先輩からもらったものを使っても良いです.
学位論文とはB4は卒業論文,M2は修士論文のことです.
どちらも数十ページからなる論文で,B4の卒論は研究室での成果及び引継用として残すため,M2の修論は学位授与のために必要なものになります.

\ref{subsec:conference}節の3と同様に,卒論$\cdot$修論のエビデンスも保存する必要があります.
特にB4生は学会発表を経験していない人がほとんどなので,全体ミーティングやSlackで以下の保存を催促しましょう.
\begin{description}
    \item[保存場所:] \textcolor{red}{NAS} > ws20oo > Lab20oo > Papers > 卒論概要集,卒業論文,修論概要集,修論論文 > 個人名
    \item[保存するもの:] 論文PDF,texファイル,生データ,コアとなるプログラム
\end{description}
参考までに卒論$\cdot$修論に関する2025年度の大まかなスケジュールを以下に示します.
\begin{table}[h]
\centering 
  \begin{tabular}{|c|c|c|}
    \hline
    \textbf{日付} & \textbf{スケジュール} & \textbf{提出場所} \\ \hline
    12月末 & 卒論$\cdot$修論の第一稿提出(12月報告書の代わり)& \textcolor{blue}{teams} \\ \hline
    1月頭 & 第一稿に対して先生からのフィードバック & - \\ \hline
    1月末 & B4は概要集提出,M2は概要集と修論の提出 & 大学,\textcolor{red}{NAS} \\ \hline
    2月末 & 公聴会後,卒論$\cdot$修論の最終稿提出 & \textcolor{red}{NAS} \\ \hline
  \end{tabular}
\end{table}

\end{document}

\usepackage{input}
  \newcommand{\test}{waya}
\begin{document}
\maketitle
\section{パーティー係}
パーティー係は研究室行事後の打ち上げや懇親会を行う際に,会場を決めたり飲食物を購入するなどが主な仕事内容です.
本研究室では,これらの飲み会を学校内(食堂,ラウンジオーク,研究室)か店で行います.
特に決まりはないのですが,基本的には中間報告会の打ち上げは学校内で行い,中間発表や卒論発表などでは店で行います.
理由としては研究室でやった方が安いからです.流れとしては

\begin{itemize}
  \item 室長,発表係に発表会の日程を確認
  \item 野中先生,関口先生に必ず確認を取る
  \item 予算を決める
  \item 鈴木先生に口頭にて,飲み会に参加するかどうかを確認(鈴木先生には研究室のグループメールが届いてないので必ず口頭で確認しましょう!)
  \item 古田さんに飲み会に参加するかどうかを確認
  \item 参加人数の確認
  \item 場所の確保,または予約(アレルギーの人がいれば報告)
  \item 会計と相談後,ミーティング及びメールにて全員に時間,場所,料金を連絡
  \item お金を回収(これはパーティー係の仕事ではなく,会計係の仕事なので丸投げしましょう!)
  \item 前日(または$2,3$日前)に全員が確認できるよう,研究室のグループメールにて集合場所,時間,会場場所などをメールする
  \item 当日の発表後に,全員を誘導する
\end{itemize}

となります.また,各飲み会のだいたいの予算は以下のとおりです.

学校内でお酒とお菓子で行う場合500~1000円

学校内で出前(ドミノピザやたまき)をとる場合1500~2000円

飲み屋で飲む場合3000~4000円

上の例はあくまで参考ですので,それ以外でも大丈夫です.ただあまり高いと,上級生とお酒が飲めない人に文句を言われるので気を付けて下さい.
また,以下に今までのパーティ係主催の飲み会をまとめたものを示します.
\begin{table}[h]
  \centering
  \caption{2019度の飲み会一覧}
  \label{table:partytable2019}
  \begin{tabular}{|c|c|c|c|}\hline
	月日&名目&場所&料金(円) \\ \hline
	4月	&本配属歓迎会	&5Fフリースペース		&500	\\ \hline
	5月	&中間報告会		&ラウンジオーク			&2500	\\ \hline
	6月	&中間報告会		&5Fフリースペース		&500 	\\ \hline
	8月	&中間発表会		&とと炉 溝の口			&3000	\\ \hline
	9月	&仮配属歓迎会	&ラウンジオーク			&3000	\\ \hline
%  10月 &報告会			&5Fフリースペース		&700 \\ \hline
	11月&報告会				&総研多目的室		&300	\\ \hline
	12月&忘年会 				&ハッケン酒場 自由が丘 &3500 \\ \hline
	2月	&卒論・修論公聴会・事例研最終発表会		&◎NIJYU-MARU 溝の口店		&3600 \\ \hline
	3月 & 卒業式 & \multicolumn{2}{c|}{新型コロナウイルスにより中止} \\ \hline
%  3月(予定)	&卒業式			&5Fフリースペース			&0 \\
  \end{tabular}
\end{table}

\begin{table}[h]
  \centering
  \caption{2018度の飲み会一覧}
  \label{table:partytable2018}
  \begin{tabular}{|c|c|c|c|}\hline
    月日&名目&場所&料金(円) \\ \hline
	5月		&中間報告会								&ラウンジオーク				&2000	\\ \hline
	6月		&中間報告会								&5Fフリースペース			&500 	\\ \hline
	8月		&中間発表会(台風で中止)		&居心伝 自由が丘	&3500	\\ \hline
	9月		&仮配属歓迎会							&ラウンジオーク				&3000	\\ \hline
	10月	&中間報告会(おでんパ)				&5Fフリースペース			&700	\\ \hline
	11月	&中間報告会								&カフェSORA				&700	\\ \hline
  	12月	&忘年会 										& わん 溝の口店 		&3500	\\ \hline
	2月		&卒論・修論公聴会・事例研最終発表会		&KANATA 渋谷店		&3700 \\ \hline
	3月		&卒業式										&5Fフリースペース			&0 		\\ \hline
%   & &(未定)							& \\ \hline
  \end{tabular}
\end{table}

以下に店で行う場合と,研究室で行う場合の$2$パターンでの詳細な説明を示します.

\subsection{飲み屋で行う場合}
飲み屋で行う場合は,店の予約をしてお金を払うだけなので簡単ですが,料金が高く,また人数も多いので早めの店の予約が必要になります.流れは,

\begin{itemize}
  \item 室長,発表係に発表会の日程および,終了時間の確認
基本的に発表は伸びるので,発表係が設定した終了時間プラス$1$時間くらいに終わることを考えて,店の予約時間を決める.必ずメンバーの終電を確認して,一番早い人の1時間前には終わらせるように決める.

  \item 鈴木先生に口頭にて,飲み会に参加するかどうかを確認

 \item 古田さんに飲み会に参加するかどうかを確認

  \item 参加人数の確認
用事があってこれない人もいるので,ちゃんと確認しましょう.また,行きたくないという人がいる場合は頑張って誘いましょう.

  \item 店の検索,予約,料金の確認
店は$2\sim3$週間前には予約しましょう.店を決めるのはパーティー係の特権です.自分の好きなところを選びましょう
人数の最終確定期日を聞いておきましょう.キャンセル料についても聞いておきましょう.

  \item 会計に料金の連絡後,ミーティング及びメールにて全員に時間,場所,料金を連絡

  \item 会計係にお金を回収してもらう
丸投げです

  \item 前日(または$2,3$日前)に全員が確認できるよう,研究室のグループメールにて集合場所,時間,飲み屋の場所などをメールする
なにが起こるかわからないので,基本的には$1\sim2$日前に連絡するのがおすすめです

  \item 当日の発表後に,全員を誘導する

先ほども書いたよう,発表はほぼ予定どおりに終わらないので,臨機応変に対応してください.
集合場所は店にした方がパーティ係は楽ですが,お店前などは道が狭かったりして点呼がとりづらいために集合場所としてはオススメできません.
なので,近くの駅の広いところに集合してパーティ係先導の下お店まで移動するのが良いと思います.
ただし,先生方はお店に直接いらっしゃる場合が多いので,事前連絡にて使う店の場所や食べログ等のURLは周知しておきましょう.
カレンダーに場所がわかるようにしとくのがベストです.
\end{itemize}


店の選び方ですが,やはり大手チェーン店を選ぶのがおすすめです.
下手にお洒落な場所を選ぶと,店員が少なくて店の回転が遅かったり,食事の量自体が少なかったりといったことが起こります.
ただ混んでいる所も料理や飲み物が来なくて文句を言われます.
またチェーン店を選ぶと「えーこの店高いじゃーん」とか「あっちの店の方が安いじゃーん」とか「そこの店の酒はおいしくないじゃーん」とか言われますが,そこは大人の対応をしましょう.
ただし,先生がいらっしゃる飲み会はあんまり酷いお店(○休とか)にしてしまうとなんか申し訳なくなるので,値段とパフォーマンスとの最適化を図ってベストなお店をチョイスできるとパーティ係として素晴らしいと思います.
参考としてTable~\ref{table:shopdata}に過去に利用した店名とだいたいの料金を示します.

\begin{table}[h]
  \centering
  \caption{いままで利用した店}
  \label{table:shopdata}
  \begin{tabular}{|c|c|}\hline
    店名&料金 \\ \hline
    牛角自由が丘店						& 4000 \\ \hline
    くいもの屋わん自由が丘店			& 3000 \\ \hline
    off cafe										& 3800 \\ \hline
    はじめの一歩								& 3500 \\ \hline 
  鳥良 自由が丘駅前店				& 3500 \\ \hline
  地鶏黒木屋 溝の口総本店		& 4000 \\ \hline
  茶き茶き 溝の口店						& 3500 \\ \hline
  千年の宴 自由が丘駅前店		& 3000 \\ \hline
  囲 溝の口店								& 3500 \\ \hline
  29BEERFEST 溝の口店			& 3800 \\ \hline
  ちばチャン渋谷店						& 3000 \\ \hline
    せかいち 自由が丘店					& 3500 \\ \hline
    土間土間 自由が丘店				& 3500 \\ \hline
    鳥良 自由が丘店						& 3000 \\ \hline
    黒木屋 溝の口総本家				& 3000 \\ \hline
    わん 溝の口店 						& 3500 \\ \hline
    KANATA 渋谷店 						& 3700 \\ \hline
	 とと炉 溝ノ口店							& 3500 \\ \hline
	 ハッケン酒場 自由が丘店			& 3500 \\ \hline
	◎NIJYU-MARU 溝ノ口店			& 3600 \\ \hline
  \end{tabular}
\end{table}

以下は過去に先輩が利用されたお店のコメントです.
\begin{itemize}
  \item せかいち 自由が丘店 \\
  お店の雰囲気とスタッフの対応はすごくよかったです.
  お酒はすぐに出てきますが,ちょっと食ベものが少ないかもです.
  お店はそんなに広いわけではないので,50人を超える規模は厳しいと思います.

 	\item 土間土間 自由が丘駅前店 \\
  大手チェーンだけあって安定です.
  ただ,大人数も可能とありましたが,複数の個室を使っての部屋だったので少し大変でした.
  大人数で行う場合ではお店の1つと考えていいと思います.

 	\item 鳥良 自由が丘店 \\
  食事,お酒ともにおいしくいいお店でした.
  50人近くでしたが,大広間を使えたのでよかったです.
  味が良かった分ちょっと量が少なめかもしれないです.

 	\item 黒木屋 溝の口総本家 \\
  M2の先輩達のリクエストで選びました.
  小さめのお店でしたが,2階を貸し切りにしてくださったので自由にできました.
  スタッフの対応も早くよかったと思います.
  食事の味はよかったですが,お酒の種類が少ない感じでした.
  3000円で2時間半の飲み放題だったので長居でき,最後の飲み会としてはよかったと思います.

 	\item わん 溝の口店 \\
  安定のわんです.混んでいたのか飲み物が来るペースが少し遅かったですが特に大きな問題はなかったように思います.
  大手チェーンだけあって全般的な対応はよかったです.
  困ったらわんか土間土間辺りが外れないと思います.

 	\item KANATA 渋谷店 \\
  おすすめしません.

	\item とと炉 溝ノ口店 \\
	お酒の種類が非常に多いのは良いですが,料理が少ないです.
	少し場所がわかりづらい場所だったので,注意が必要です.

	\item ハッケン酒場 自由が丘店 \\
	料理はおいしいですが,やはり量が少ないです.
	

	\item ◎NIJYU-MARU 溝ノ口店 \\
	お座敷を貸し切り状態で使用できました.料理も飲み物もあまり待たされずに来ました.
	50人以上でも可能です.
\end{itemize}

近年,研究室の母体がかなり大きくなっているので,小さいお店だとまず入りませんし,宴会ができるとしても早めの連絡が必須になるので,ここだけは本当に気を付けてください.早め早めに動きましょう.
特に仮配属が決まった後は50人を超えてくるので気をつけてください.50人を超えるとお店もだいぶ限られてしまいます.早めに目星をつけとくといいです.

退店をする際には必ず,忘れ物がないかチェックしましょう.貴重品を無くしたりすると大変なのでチェックをしましょう.
お店の迷惑にならないよう退店予定時間にはお店を出られるように声をかけましょう.
できる限り食べ残しや飲み残しがないように,終了10分前くらいには声を掛けましょう.

\subsection{学校内で行う場合}
学校内で行う場合は,スタート時間の決まりがないので気は楽ですが,会場の予約や消費する飲食物の購入などをしなければいけなせん.流れは,

\begin{itemize}
  \item 室長,発表係に発表会の日程および,終了時間の確認

  \item 野中先生,関口先生に必ず確認を取る

  \item 会場の予約
研究室が去年は多かったですが,人数が多いときは,基本的にはラウンジオークか食堂を予約してください.

  \item 鈴木先生に口頭にて,飲み会に参加するかどうかを確認

 \item 古田さんに飲み会に参加するかどうかを確認

  \item 参加人数の確認
腕の見せ所です.行きたくないとかいう人を頑張って誘いましょう.

  \item 何を食べるのかを決め,だいたいの予算を算出.その後,その予算から$1$人あたり料金の設定

  \item 会計に料金の連絡後,ミーティングにて全員に時間,場所,料金を連絡

  \item 会計係にお金を回収してもらう
丸投げです

\item 買い出し
好きなものを買いましょう!買ったあとは研究室の冷蔵庫に詰め込みます.
買い物に行く前日には研究室の人に周知し,冷蔵庫をできるだけ空けるよう協力を呼びかけましょう.
結構,満杯になります.

  \item 前日(または$2,3$日前)に全員が確認できるよう,研究室のグループメールにて時間,場所などをメールする.

  \item 当日の発表後に全員を誘導.また同時進行で飲食物の準備
臨機応変に対応しましょう.飲食物の準備は周りの人に協力してもらいましょう.

\end{itemize}

\subsection{ラウンジオークで行う場合}
ラウンジオークで行う場合は,会場の予約や終わりの時間が決まっています.また消費する飲食物の購入などをしなければいけなせん.流れは,

\begin{itemize}
  \item 室長,発表係に発表会の日程および,終了時間の確認

  \item ラウンジオークのサイトで空いているか確認

	\item 野中先生,関口先生に必ず確認を取る

	\item 鈴木先生に口頭にて,飲み会に参加するかどうかを確認

	\item 古田さんに飲み会に参加するかどうかを確認

	\item 書類の記入.書類はポータルサイトから.

	\item 書類に不備が無いようにして,野中先生にハンコをもらう

	\item ラウンジオークの人にハンコをもらう.

	\item 学支の4階の総務課に書類を提出
\end{itemize}

\subsection{総研多目的室で行う場合}
行う日が決まったら,関口先生に予約をお願いしてください.流れは,

\begin{itemize}
	\item 室長,発表係に発表会の日程および,終了時間の確認

	\item ラウンジオークのサイトで空いているか確認

	\item 野中先生,関口先生に必ず確認を取る

 	\item 鈴木先生に口頭にて,飲み会に参加するかどうかを確認

	\item 古田さんに飲み会に参加するかどうかを確認

	\item 関口先生に総研多目的室の予約をお願いする
\end{itemize}

となっています.先ほども書きましたが,飲み会の場所は世田谷の研究室,食堂,ラウンジオーク,総研多目的室のどこかです.
世田谷の研究室だと時間の制約がないうえ,ガスなども使えるので料理もできます.
ただ人数が多いのでどうしても狭くなってしまいます.
食堂,ラウンジオークですと,夜8時までしか使えなく,またカセットコンロなどの火器が使えないので,
基本的に食料は出来上がったものを持ち込む形になりますが,スペースが広いのでいろいろできます.
ちなみに先生のお気に入りはラウンジオークです.
予約手順は,ラウンジオークの場合,ラウンジオークにいって,その日程が空いているかを確認後,1号館4階の事務室に行き,
施設利用申請書をもらって必要事項を記入し,先生の印鑑(先生の名前と印鑑が必要ですが,先生の名前は自分で書いてください)と,
ラウンジオークの人の印鑑をもらい,1号館4階の事務室に提出して終了です.
食堂も同じような手順ですが,学生支援センターと食堂の人に確認をとってから,両方の印鑑をもらいます.施設予約は以上です.

買い出しについては,学内での飲み会の場合まいばすかカクヤスを使用していました.
去年はほとんどまいばすでした.
カクヤスの場合は前日までにネット注文すればキャンパス中央の道路まで運んでくれます.

以下に過去行った学校内での飲みを示します.
\begin{table}[h]
  \centering
  \caption{2019年度に学校内で行った飲み会}
  \label{table:shopdata}
  \begin{tabular}{|c|c|c|c|}\hline
	番号&場所&内容&料金 \\ \hline
	1	&ラウンジオーク		&NECライベックスに注文									&2500	\\ \hline
	2	&フリースペース		&お菓子															&500	\\ \hline
	3	&ラウンジオーク		&仮配属歓迎会,NECライベックスに注文		&3000	\\ \hline
	4	&総研多目的室	&留学生歓迎会,ソフトドリンクのみ					&0		\\ \hline
	5	&総研多目的室	&お菓子															&300	\\ \hline
  \end{tabular}
\end{table}
\begin{table}[h]
  \centering
  \caption{2018年度に学校内で行った飲み会}
  \label{table:shopdata}
  \begin{tabular}{|c|c|c|c|}\hline
	番号&場所&内容&料金 \\ \hline
	1	&ラウンジオーク	&NECライベックスに注文									&2000	\\ \hline
	2	&フリースペース	&お菓子															&500	\\ \hline
	3	&フリースペース	&留学生歓迎会,ソフトドリンクのみ					&0		\\ \hline
	4	&ラウンジオーク	&仮配属歓迎会,NECライベックスに注文		&3000	\\ \hline
	5	&フリースペース	&学生歓迎会,ソフトドリンク							&0		\\ \hline
	6	&フリースペース	&おでんパ,セブン												&700	\\ \hline
	7	&カフェSORA	&菓子パ															&700	\\ \hline
  \end{tabular}
\end{table}

\begin{enumerate}
  \item NECライベックスに料理を注文して行いました.お酒はカクヤスです.料理は量を重視して用意した方がいいです.麺やご飯もの(寿司以外)は量があります.
  \item まいばすでお菓子を買いました.お酒はカクヤスです.お酒の余りがあると安く済みます.
  \item ソフトドリンクのみ用意して,M2の先輩に研究紹介をしていただきました.一応研究室メンバーの自己紹介の時間も用意しました.
  \item 同じくNECライベックスで,お酒はカクヤスです.NECは色々言えば対応してくれます.自己紹介と名前ビンゴ,M2の先輩にお願いして研究紹介の時間を設けました.終了時刻は気をつけてください.
  \item ソフトドリンクのみ用意して,M2の先輩に研究紹介をしていただきました.一応研究室メンバーの自己紹介の時間も用意しました.
  \item セブンでおでんパックを大量に買いました.鍋と火の用意が大変ですが,一番好評でした.量は多めに用意したほうがいいです.
  \item まいばすでお菓子を買いました.お酒はカクヤスです.
\end{enumerate}


\subsection{研究室メンバーのお酒の好み}
次に1年間で把握した研究室メンバーのお酒の好みを記します.去年までは,ビールはあまり売れませんでした.また,酒に強い人もいますがそこそこの人が多いので,5\%のチューハイは多めに用意しておいた方が良かったです.新4年生の好みは随時把握しておいてください.
%新4年生の好みは把握しきれなかったので,新パーティ係は随時把握してください.

先生
\begin{itemize}
\item 野中先生~ビール日本酒ワインが好きです.ビールだと黒がお気に入りのようです.
\item 関口先生~自転車通勤なので飲みません.飲む場合は甘いカクテルなどが好きです.
\item 鈴木先生~あまり飲みませんが,ビールを飲みます.
\end{itemize}

% 修士2年
% \begin{itemize}
% \item 柴田~まあなんでも飲んでるイメージ,基本ビール?
% \item 津野~飲ませ過ぎに注意.とりあえずなんでも飲んでる.
% \item 成勢~飲まない.必ずソフドリがある所で.
% \item 日浦~日本酒,梅酒が好きみたい.
% \item 森廣~基本的になんでも!ただジンジャーハイボールがあれば幸せ.
% \end{itemize}

2019年度は,お酒の管理をエクセルで行いました.その他資料があるので,必要があれば2019年度のパーティ係まで.

\subsection{その他}
\begin{itemize}
  \item 留学生が来る時があります.その時は来た日の空いた時間をつかって歓迎会をしましょう.去年はソフトドリンクだけ出して,先輩方に自分たちの研究紹介をしてもらいました.
  \item B3生の歓迎会の時には名札を作ります.B3生も入れて全員分になります.資料を見ればわかりますが,各自で凝っているものと,普通のものを用意しています.
\end{itemize}

\subsection{注意事項}
最後にパーティ係を行う上での注意事項を示します.

\begin{itemize}
  \item 先輩の意見は参考程度に聞きましょう.全員の意見を聞いたら一生飲み会の内容は決まりません.
  \item ビールの他にもカクテル,ソフトドリンク,ワイン,日本酒,梅酒は買いましょう.ちなみにドリンクが余ったら研究室に保管できるので,余ることの心配はしなくて大丈夫です.
  \item アレルギーには気をつけてください.お店には言えば対応してくれます.
  \item 野中先生はそばアレルギーなのでそばはやめましょう.
 \item 古田さんは甲殻類アレルギーです.外部で行うときや食堂に注文するときは,必ずこの事を伝えてください.
  \item 先生の席は初めに決めておきましょう.みんなが気をつかって席が決まりません.
  \item 全員の財政状況に気を使いましょう.合宿や学園祭などのイベントと同じ月にはお金がかかる飲み会はやめましょう.
 \item 年度末の飲み会等のお金のかかる飲み会の時は,学生には出来るだけ早く予算金額だけ周知してあげましょう.具体的に何も決まってない段階でもいいので,お金を用意する時間を与えるのが参加率アップにつながります.
\end{itemize}

以上です.大人数を連れて飲み会を企画するのは大変ですが,研究室の雰囲気づくりにとても重要な仕事なので頑張ってください.


\subsection{リモートでの催しを行う場合}
パワーポイントや作成した動画など,使用したものは全てlabやワークスペースに入れてあります.
今年度からリモートという形になり初めての試みだったので,参考程度に見てください.これを模倣する必要はありません.
\subsubsection{毎月の報告会の打ち上げ}
《流れ》\par
開始の挨拶(事前に野中先生に乾杯の音頭を依頼)\\
⇒ブレイクアウトに別れる⇒各セッション会話⇒全体に戻る\\
⇒雑談してもらってるうちにブレイクアウト作成⇒移動\\
上記を時間内で繰り返す\\
⇒終了の挨拶(全体)\\
※最初の飲み会,最終発表など節目の打ち上げは室長,副室長の挨拶も入れる\\
《日取り》
\begin{itemize}
\item{研究後の場合}
 コアタイムから全員の帰宅時間を把握して設定
\item{研究がない日の場合}
事前にアンケートを取るなどして全員の予定を把握,後に先生と相談(あくまで先生方が最優先)
\end{itemize}
《ズームの割り振り》
\begin{itemize}
\item{基本は毎回自動生成後バランスを考えて適宜移動}
\item{あらかじめ用意したcsvファイルを読み込む方法もある}
\end{itemize}
《事前準備》
\begin{itemize}
\item{先生などに乾杯の音頭の依頼}
\item{日取りの確認や流れなどを全体に流す(前日にリマインド)}
\end{itemize}
\subsubsection{新入生歓迎会}
《流れ》\\
M0生が5限があったため,表\ref{table:新入生歓迎会}のようなスケジュールになりました.\par
\begin{table}[h]
  \centering
  \caption{新入生歓迎会のスケジュール(2020)}
  \label{table:新入生歓迎会}
  \begin{tabular}{|c|c|}\hline
    時刻&内容 \\ \hline
    17:55~&集合 \\ \hline
    18:00~&開会の言葉(野中先生) \\ \hline
    18:05~&自己紹介(1人30秒×54) M0自己紹介は最後\\ \hline
	18:35~&ブレイクアウト(5分×3) \\ \hline
	18:50~&紹介動画(3分×5=15分) \\ \hline
	19:05~&学会紹介(5分×3=15分) \\ \hline
	19:20~&ブレイクアウト(5分×3) \\ \hline
	19:35 &閉会 \\ \hline
  \end{tabular}
\end{table}
\vspace{3cm}
当日はzoomではない新しい物を使用しようとした結果1\,時間ほど予定が遅れました.新しい事をする時は事前の練習を必ず行いましょう.\\
《自己紹介》
\begin{itemize}
\item{自作したパワーポイントをワンドライブに載せて各自に作成してもらった(パワーポイントはワンドライブに載せてあります)}
\item{留学生もいたので英語で書いてもらうように促す}
\end{itemize}
《紹介動画》\\
B3が配属する班を決めるのに大きく関わるのと,研究室紹介の際に伝えられない面白い部分を伝えたくて作りました.
\begin{itemize}
\item{B4に手伝ってもらった}
\item{研究室全体の紹介(総研,世田谷)⇒各班の説明⇒〆の言葉}
\end{itemize}
《学会紹介》
\begin{itemize}
\item{必ず海外,国内はマスト(居なかったら先生とかに相談してください)}
\item{去年は海外に行った人が1人と国内の人2人に依頼}
\item{事前に3名を指名,流れを説明}
\end{itemize}
《ブレイクアウト》\\
毎月の飲み会同様行う.
\begin{itemize}
\item{割り当てはB3固定でその他が初めのセッションのの次番号のセッションに移動}
\item{初めの割り当てはcsvで事前に作成したものを仕様(班をバランスよく配置) }
\end{itemize}





\end{document}

\usepackage{input}
\usepackage[dvipdfmx]{hyperref}
\usepackage{wrapfig}
\usepackage[dvipdfmx]{graphicx}
\usepackage[dvips]{graphicx}
\usepackage{url}
\begin{document}
\maketitle
\section{動画企画係}
\subsection{主な仕事内容}
動画企画係は,高機能機械制御研究室で行っている研究内容について外部の人間に知ってもらうための動画の作成を行う係です.
主な仕事内容として,
\begin{enumerate}
  \item 動画化する研究の選定やアイデア出し
  \item 実験の協力依頼
  \item 実験動画の撮影と編集
\end{enumerate}
である.詳しい内容は以降で説明する.

\subsection{動画化する研究の選定}
ある程度専門知識を持っている人を想定したデモや発表会とは異なり,一般の方にも視聴していただく動画を作成するので,直感的に見てこの技術を使うと何が達成できるかというビジョンを見せられる研究を選定することをおすすめします.
動画化する研究は一般的な手法のデモ,または学会などで発表済みのものを選定します.合同研究内容や,未公開情報を含むものはアップロードできないので,動画化する研究については必ず一度先生の確認をとってください.
社会的な意義と研究室の技術を包括した紹介動画が好ましいです.


\subsection{実験の協力依頼}
動画化する研究が決定したら,研究担当者にアポをとり,実験日や実験内容,動画としてどのように見せるべきかについて話し合います.
実験に人手が必要な場合は,ほかの学生に協力をお願いし,実験当日は動画係も一緒に実験を手伝います.

\subsection{実験動画の撮影と編集}
視聴者に内容が直感的に伝わるように動画を編集し,解説音声の吹込みや解説字幕を追加します.
(英語はM1の野崎君に依頼することをお勧めします.by潘瓊)
(例:電動車椅子の実験で,定点カメラから全体の動きを撮影するに加え,車椅子にもカメラを付けることで乗車中の人の目線からの動画も撮影できる.)
高機能研ではAdobe Premiere Proという動画編集ソフトのライセンスを所有しています.
関口先生に伺えば,使用方法を教えていただけるとのことなので,是非早い段階で聞いてみてください.
また,動画撮影は研究室所有のGoProを使うこともできます.詳細については,2021年度動画企画係である,M1の野崎と潘に聞いてください.

\subsection{youtubeチャンネル}
こちらが高機能機械制御研究室のyoutubeリンクです.過去の研究動画があがってるので,是非参考にしましょう.
\url{https://www.youtube.com/channel/UCJbvNjUHcmQqbrsOUF8ZkEw}

\end{document}


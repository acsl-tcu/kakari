\usepackage{input}
\usepackage{color}
\def\red{\color{red}}
\def\black{\color{black}}
  \newcommand{\test}{waya}

\begin{document}
\maketitle
\section{室長}
室長の大まかな仕事内容を以下に示します.
\begin{itemize}
  \item オープンキャンパスの仕切り
  \item 3年生研究室見学会の仕切り
  \item ミーティングの時間決めおよび司会進行
  \item 発表会・報告会の出欠確認
  \item 年度特有の活動の仕切り
\end{itemize}

\subsection{オープンキャンパス・研究室見学会\\
ポスター作成および運営の進め方}
オープンキャンパスに関係する資料はサーバーのWorkSpaceの
\par
/workspace/Work2019/Work2025/YamatoOrii/引継ぎ資料/室長関係"
\par
にあります.
\par
\subsection{全体方針(2025年度の進め方)}
2025年度は,オープンキャンパス用と3年生研究室見学会用で
同一のポスターを使用できるよう,
オープンキャンパスの時点でポスターを作成した.
ポスターは研究内容ごとに5つの研究班に分けて作製した.

オープンキャンパス当日は,
総研担当と学科担当に分かれて運営を行った.

\subsection{ポスター作成}
\subsubsection{作成対象}
ポスターは以下の5つの研究班ごとに作製した.
ドローン,
バタフライ,
ローバー,
脚車,
車椅子.

\subsubsection{作成と印刷}
オープンキャンパスおよび研究室見学会で共通使用することを前提に,
オープンキャンパス前にポスターを完成させた.
完成後は図書館の大型印刷機を用いて印刷した.

\subsection{オープンキャンパスでの担当区分}
\subsubsection{総研担当}
総研担当ではポスター展示は行わず,
プロジェクターを用いて壁面にスライドを投影した.
以下のデモを実施した.
東急ドローンの飛行デモ,
脚車による Go2 のデモ,
電動車いすの歩行者回避デモ.

会場設営(21A/21B 教室の壁移動,機材配置など)や
全体の段取りは室長が担当した.
デモ内容については事前に教授と相談して決定する必要がある.
2025年度はデモ準備とオープンキャンパス全体の仕切りを同時に行ったため,
作業負担が大きかった.
次年度以降は副室長を含めて役割分担することが望ましい.

\subsubsection{学科担当}
学科担当はポスターセッション形式で研究紹介を行った.
運営は永野先生が担当しており,その指示に従って対応した.

\subsection{総研展示の概要(2025年度)}
\subsubsection{展示内容}
インテリジェントロボティクスセンターの紹介,
およびドローン,電動車いす,4足歩行ロボットのデモを実施した.

\subsubsection{スケジュール}
8/1(金)午後に会場設営および動作確認を実施した.
8/2(土)は会場締切のため作業は行っていない.
8/3(日),8/4(月)の両日とも,
10:00--16:00 に展示およびデモを実施した.
デモ時間は前年実績に基づいて設定しており,
変更の可能性がある.
8/4(月)の展示終了後に片付けおよび壁復旧を行った.

\subsubsection{担当学生}
担当学生は,
ドローン2名,
電動車いす3名,
4足歩行(Go2)1名とした.
4足歩行については,
佐藤先生の研究室の学生1名と合わせて2名体制で実施した.
8/3,8/4 の 10:00--16:00 の6時間はアルバイト対象時間である.

\subsection{注意事項}
オープンキャンパス当日はアルバイト代が発生するが,
総研担当と学科担当は別団体扱いとなるため,
可能な限り担当は固定する必要がある.
1日目は総研,2日目は学科といった分担は避けることが望ましい.

また,例年1名は入試課対応として
オープンキャンパス運営側に回る.
室長が担当することが多いが,正直誰でもいい.



\subsection{3年生研究室見学会の仕切り}
日程等は先生と相談して決めてください.
例年だと制御システム設計の最終回の授業後とテスト期間後に行っています.

内容は,オープンキャンパスと似ていて,ポスターを使ってそれぞれの研究の説明を行います.
だたオープンキャンパスよりも詳細な内容を説明するようにしましょう.
研究室でどんな研究をやっているのかを説明して,
3年生に自分たちの研究室に興味を持ってもらえるようにできると良いです.

オープンキャンパスと同様にシフト表やタイムスケジュールを作る事も忘れずに.
場所は10号館4階(総研)で行いました.まず,全体説明をプロジェクターで映し,その後は各班好きなところを回ってもらう形で行いました.
開催日の連絡は,webクラスで野中先生に周知してもらいました.


\subsection{ミーティングの時間決めおよび司会進行}
週に1回行っていました.
先生方の都合により時間が前後する場合がありますが,基本的にはミーティングを行う曜日と時間は固定していました.
ミーティングを行う日時が決定したらgoogleカレンダーに書き込みをしてください.
基本どちらかの先生がいればミーティングは行っていました.不在の先生の連絡事項があれば代わりにアナウンスしていました.

何らかの理由でがzoomで開催する場合があるので,必要に応じてzoomを立ち上げてください.基本対面です.
室長はミーティング開始時刻になったら先生方を呼び,前に立って司会進行を務めます.
連絡事項は学生→関口先生→野中先生の順で伝えてもらっていました.
議事録は副室長などにとってもらい,まとめたファイルをslackに流してください.


\subsection{発表会・報告会の出欠確認}
発表会と報告会の出席は卒業要件に含まれているため,室長は当日の出欠状況を確認し,Excelなどにまとめてください.
また,出欠状況をまとめたファイルは,発表資料と同じ階層にアップロードしてください.(場合によっては関口先生に直接送るよう指示されるかもしれません.)
→今年はExcel作成してましたがときに提出とかはしてません.欠席が目立つ人等は先生に連絡していました.たまに先生に見せてって言われます.

\subsection{1年間のスケジュール}
スケージュールとして2025年度のスケジュールも示しておきます.
\begin{table}[h]
\centering 
\begin{tabular}{|c|c|}
\hline
\textbf{日付} & \textbf{スケジュール} 		\\ \hline
4/2         & 研究室ガイダンス        		\\ \hline
7/25       & 研究室紹介(制御システム設計 授業後)						\\ \hline
7/30        & 研究室紹介 期末試験・配属説明会後)         			\\ \hline
8/3,4       & オープンキャンパス       			\\ \hline
9/19        & 研究室配属会          		\\ \hline
9/23        & 初回事例研究          		\\ \hline
\end{tabular}
\end{table}
\subsection{年度特有の活動の仕切り}
引継ぎ資料のフォーマット整理,総研の引っ越し作業の段取り,留学生等の新しく来る人の受け入れ準備,来客への研究室見学の段取り等,様々な仕事がありましたが,いろんな人に任せましょう.

\subsection{最後に}
様々な仕事を任されてると思いますが,決して一人で抱え込まずにみんなで協力して取り組みましょう.
今年は,いろんな事やりすぎた気がしてます.暇な人もいるのでどんどん仕事を投げちゃいましょう!
ストレスがかかる場面も多いとは思うけど,ぜひポジティブな気持ちで取り組んでほしいなと思います.
きっといい経験になると思います.頑張ってください.

\subsection{過去の仕事}
今年はやってないですが,卒論・修論の製本カバーを買う,夜間申請と休日申請を先生に提出(コロナの時は入校申請),副室長と院ゼミ・学部ゼミの日程決めなどの活動もあったみたいです.
詳細は 2024old-version ブランチの main.pdf を確認してください.
\end{document}

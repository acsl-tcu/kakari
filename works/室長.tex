\usepackage{input}
\usepackage{color}
\def\red{\color{red}}
\def\black{\color{black}}
  \newcommand{\test}{waya}

\begin{document}
\maketitle
\section{室長}
室長の大まかな仕事内容を以下に示します.
\begin{itemize}
  \item 夜間申請と休日申請を先生に提出(コロナの時は入校申請)
  \item 副室長と院ゼミ・学部ゼミの日程決め
  \item オープンキャンパスの仕切り
  \item 3年生研究室見学会の仕切り
  \item ミーティングの時間決めおよび司会進行
  \item 発表会・報告会の出欠確認
  \item \sout{卒論の製本カバーを買う}
\end{itemize}

\subsection{夜間申請と休日申請を先生に提出}
\red2023年度の途中から,これらの申請は教員が一括で行うことになりました.\black
ここでは,夜間申請と休日申請について説明します.必要な資料は,
\par
/workspace/Work2020/makino/引き継ぎ資料/室長関係/申請書類2020
\par
にあります.
\par
毎週月曜日に夜間申請を関口先生に提出します.2019,2020年度は月曜日でしたが,曜日は年度によって変わる可能性があるため,最初に先生に渡す日程を確認してください.
2020年度からコロナの関係で総研と世田谷で分かれて活動しています.
夜間申請は総研と世田谷でそれぞれ提出が必要です「【総研】研究室・実験室等(夜間)使用願.docx」のWordファイルを使用してください.
これからこのWordファイルの夜間申請の記入箇所について記します.最初に夜間申請を記入する際には以下のことを行ってください.

\begin{itemize}
	\item 4年生の指名と学籍番号を「使用者名簿」に新しく記入してください.
  	\item M1に関しては学籍番号が変更するため,確認して「使用者名簿」に記入してください.
  	\item 「使用学生代表者」に自分の名前,電話番号,学籍番号,学年を記入してください.
\end{itemize}

毎週提出する箇所として以下の2つです.基本的に夜間申請書は同じファイルで使いまわします.

\begin{itemize}
  	\item 1ページ目右上の「届出日」を変更して下さい.今年度は月曜日に先生に渡していて,提出日は渡した翌日とかにしてました.
  	\item 2ページ目の最後の「利用週」を提出する次の週の月曜日から土曜日までの日付を書いてください.
  	\item 世田谷の申請書は,利用時間を記入する欄があります.平日は夜間利用時間に当たる20:00~23:00とし,休日は8:00~20:00としてください.土曜は休日に当たります.
\end{itemize}

過去には(2016年度)関口先生からの提案で”月の初めに,その月の夜間申請をまとめて渡す”形になったそうです.もしかしたらそのような感じになるかもしれないので,夜間申請を週ごとに提出するのか,月ごとにまとめて渡すのか、を先生に確認してみてください.祝日や,入試などで総研に入れない日の場合は,夜間申請書の1pの使用時間におけるチェック欄と2pの使用日の表における使用時間,終了時間を変更してください.
\par
次に休日申請について説明します.総合研究所においては,学年歴に従い休館日になっている場合があります.休館日に利用する場合は,夜間申請とは別に休日使用申請を出す必要があります.休日申請書は,先ほどのファイル内の「夜間・休業日使用願.docx」を使用します.基本的な記入内容は夜間申請と同じでうが,申請する日付けを記入することを忘れないようにして下さい.
総研はキャンパスとは別の機関なので,休日等も学年歴と違うことがあります.長期休暇の前などは事務室に確認することをお勧めします.
\par
ちなみに夜間申請の提出を忘れると,研究室に20時以降入れなくなるので注意してください.

次に2020年度でのコロナの状況での入校申請について記します.
2020年度の主に緊急事態宣言下の状況では,入校すること自体に申請書が必要になっていました.なので,申請書を週ごとに作成して,ワンドライブ上に置き,各自で記入してもらいましょう.室長としてはファイルの準備と,締め切りの勧告.また関口先生への提出を行ってください.
提出するタイミングは先生と話て決めてください.2020年度は前期は研究の進捗として急遽実験等が入ることが少ないので入る週の前の週の月曜日としてました.1月~は研究も佳境に差し掛かっていたので,入る週の前の週の金曜の午前中にしていました.

%---2019まで-----%
%\subsection{院ゼミ・学部ゼミの日程およびページ決め}

%2019年度は院ゼミと学部ゼミが週に1回ありました.行う曜日は月ごとに先生やゼミを行う人と相談して決めてください.院ゼミが何曜日,学部ゼミが何曜日と決めといた方がいいのです.学会の日程次第で2回連続で院ゼミを入れてもいいです.でも翌週は学部ゼミを2回連続で入れる様にバランスを取ってください.去年に引き続き今年度も,ゼミ関係は副室長に全面的に任せていました.

%院ゼミについて記します.

%院ゼミは論文紹介を行います.論文紹介は担当者を決め,その人が読んだ論文を紹介するといった形になります.その担当者を決めてください.学会の発表練習がある週に関しては,その学会に何人参加するかを確認し,先生の都合の付く日に発表練習してもらいます.人数が多い際には2日に分ける事もあります.先輩がメインで決めてくださる場合もあるので,その場合は先輩に日程を聞き,ゼミの日程を調整してください.

%学部ゼミについてです.

%学部ゼミでは解説記事の輪講とテキストの輪講や自分の研究に関する論文や解説記事の紹介を行います.2017年度は,年度前半に自分の研究における制御手法の紹介の輪講,年度後半ではテキストの輪講でした.2018年度は,自分の関係する手法の解説でした.2019年度は,2018年度と同様に自分の研究に関する論文の解説を行いました.今年度の内容に関しては先生と相談するようにしてください.また,何のテキストを使うのかは関口先生に確認してください.
%論文の内容を説明する必要があるため,およそ1人1時間は最低でも必要だと考えてください.

\subsection{勉強会の日程,内容決め}
2019年度までは院ゼミや学部ゼミが開催されていました.2020年度は活動が一部ロボ研と合同になったこともあり,勉強会という形で月に2回程度,1回2名の修士生が研究の要素技術の紹介,解説をするものとなっていました.事前に高機能研の発表の内容,順序を先に決めてその後ロボ研含めて日程調整という形だったようです.
この仕事は副室長に全面的にやってもらっていたので,今後もそれでいいかと思います.何かあったら副室長経験者に聞いてみてください.

\subsection{オープンキャンパスの仕切り}
2020年度はオープンキャンパスはありませんでした.そのためオープンキャンパスについては更新しません.

オープンキャンパスに関係する資料はサーバーのWorkSpaceの
\par
/workspace/Work2019/ktoya/引き継ぎ資料/室長関係/オープンキャンパス
\par
にあります.
\par
オープンキャンパスが6月初旬と8月初旬に,3年生研究室見学会が7月下旬~8月上旬にあります.2019年度に関しては,6月のオープンキャンパスでは去年度の先輩のポスターを使って説明をしました.7月下旬~8月上旬では,自分たちでポスターを作成して説明を行いました.恐らく2020年度も同様の方針となると思います.オープンキャンパスでポスターを使って研究紹介をするに当たり,誰がポスターを作り,紹介するのかを事前に決めます.ポスターを作り終えたら,図書館の印刷機の所に行き,印刷します.図書館に行く理由としてA0もしくはA1の用紙を用いるからです.大きさはM1の先輩にどうだったか確認してみてください.オープンキャンパスでは6月は去年のものを,8月は3年生研究室見学会で作成したものを使用します.また,オープンキャンパスではポスターを使って説明をする人のシフト表を作ると便利です.資料内に2019年度のシフト表のExcelファイルもアップしてあるので,参考にしてください.

次にポスター紹介です.3年生研究室見学会は時間制限がありません.しかし,オープンキャンパスには「ツアー」と「一般」に分かれており,「ツアー」は時間制限があります.そのため,最低でも3つのポスター担当者に時間制限を考慮した台本を作ってもらいましょう.ここは忙しくなると思いますので,ある程度の流れを先輩から確認してください.

\subsection{3年生研究室見学会の仕切り}
日程等は先生と相談して決めてください.
例年だと制御システム設計の最終回の授業後とテスト期間後に行っています.
\subsubsection{例年の対面での場合}
3年生研究室見学会に関係する資料はサーバーのWorkSpaceの
\par
/workspace/Work2019/ktoya/引き継ぎ資料/室長関係/研究室見学会
\par
にあります.
\par
ここでは,2019年度の見学会について記述します.2019年度は,車椅子とドローンは総研で,油圧と車両と脚車は世田谷で行いました.内容は,オープンキャンパスと似ていて,ポスターを使ってそれぞれの研究の説明を行います.だたオープンキャンパスよりも詳細な内容を説明するようにしましょう.研究室でどんな研究をやっているのかを説明して,3年生に自分たちの研究室に興味を持ってもらえるようにできると良いです.また,ポスター以外にも実物の展示やデモンストレーションも行います.2019年度は,車椅子とドローンのデモを行いました.デモで何を見せるのか,そのためのプログラムの準備等も発生するので,早めに準備に取り掛かりましょう.詳細はM1に必ず確認を取りましょう.特に2020年度は世田谷は空っぽですので,総研で全ての見学会を行う可能性がありるため,これについても先生や先輩に確認を取りましょう.当然,オープンキャンパスと同様にシフト表やタイムスケジュールを作る事も忘れずに.
\subsubsection{2020年度のコロナでの場合}
昨年度はコロナの都合上研究室紹介をzoom上で2回行うこととなりました.おおよその回し方は対面の場合と変わりません.
その場合の資料を
/workspace/Work2020/makino/引き継ぎ資料/室長関係/研究室見学会

にあります
大きな変更点としては,例年ポスターでやっているところをショートスライドに変更して紹介を行った点です.
回し方ですがブレイクアウトルームを利用しローテーションを行いました.細かくローテーションを行うのは開催的にも参加者的にも難しいことから,2つのセッションに分けて前後半で入れ替えを行いました.
後半のフリータイムではより少人数のほうが多く質問ができるため,ブレイクアウトセッションを再編成して行いました.
時間の縛りがある行事なので,事前にどんなふうに回すかを話し合い,4年生には細かく共有,理解してもらい,協力してくださる修士の先輩方にも共有しましょう.ここは対面でも変わりませんね.


\subsection{ミーティングの時間決めおよび司会進行}
\subsubsection{~2018年度までのミーティング}
2018年度までのミーティングは以下のように行われていたようです.
\par
ミーティングは週一で基本的に先生方両方の都合のつく日に行われます.ミーティングは研究室全員が連絡事項を把握するために行うものです.手順としてまず野中先生と関口先生が空いている時間を自分のわかる範囲で確認して,日程の候補を3パターンほど考えておきます.

その後,先生方に「この曜日の何時から行いたいと考えているのですが、ご都合の方はどうでしょうか?」などと失礼のないようにい聞きにいきます.(先生方に聞きに行くときは,まず,「ミーティングのことに関してなのですが,お時間はありますでしょうか?」と,今聞いて大丈夫なことを確認してから本題に入りましょう.)それで,2人の都合のよい時間に行います.またゼミや発表会がある週はその終了後に行います.(その場合も先生方への確認は行いましょう.)2017年度では野中先生からの提案で関口先生のみに確認を取っていました.年度によって変わると思うので先生方から提案されない限りはどちらの先生にも確認を取るようにしてください.さらになるべく多くの人数が参加できる時間帯をグーグルカレンダーで把握して,ミーティング時間を決定します.

決定したら,先生方に「○時からミーティングを行います.」と伝え,研究室生徒にはカレンダー,メーリングリストやラインなどで促します.その際にラインを見ない人やガラケーの人には直接言ってください.そして,ミーティング時間になったら5階のレグザ前に集まります.学生全員が集まり次第,先生をお呼びして開始となります.ミーティングはあらかじめ係の人に連絡があるのかを聞いておいてください.そうすると円滑に会が進みます.

まずあらかじめ聞いていた係の人たちを順番に指名し,連絡事項を言ってもらってください.
あらかじめ聞いていた係の人が終わったら,確認で「係で連絡ある方いますか?」などと促し,手を挙げた人を指します.学生における連絡がなくなったら,関口先生,野中先生の順で連絡があるかをお聞きします.野中先生の連絡事項が終り次第「ミーティングを終わりにします.お疲れ様でした.」などといって終了となります.

その後,メーリングリストに内容をまとめたものを記し,全員が周知するようにしましょう.

\subsubsection{2019年度のミーティング}
2019年度のミーティングは昨年までとは異なり,週1回行っていました.発表会後やゼミ後など,研究室の全員が集まっているときに行いたかったからです.2019年度は,だいたい毎週何かしらに発表やゼミがあったため,その時間で行うことができました.特に発表等がない週に関しては,ミーティングはやっていませんでした.自分は先生や学生に事前にミーティングの確認などは行っておらず,いきなり始めていました.
\par
ミーティングの流れとして,まず初めに学生で連絡を持っている人に挙手をさせて,順番に指名.学生が終わったら,関口先生,野中先生の順番で行っていました.ミーティング終了後は,Gmailで本日の連絡事項を纏めて送信していました.Gmailを送信するのが面倒な時にはSlackで流していました.

\subsubsection{2020年度のミーティング}
2020年度はコロナの都合上zoomでミーティングを行っていました.情報共有がしずらい環境だったこともあって1週間に1回のペースで行っていました.
日程の決め方については,すべてスラックで行っていました.先生方と自分のダイレクトメッセージを用いて,行事と先生の予定のカレンダーを共有されると思うのでそこから候補の日程を3,4個挙げてその中から都合のいい日程はどこか確認しましょう.
決まったら,スラックのgeneral等で連絡して研究室のGoogleカレンダーに予定として入れておきましょう.

報告会や勉強会がある週については特に連絡や日程の確認はせずに報告会後や勉強会後にやっていました.

進行の流れとしては,「学生→関口先生→野中先生→連絡事項に対しての質疑応答・連絡し忘れたこと」の順番で行っていました.
また,メモを自分でもいいし副室長などにとってもらって,ミーティングの内容をGmailのメーリングリスト等を利用して周知してください.参加できなかった人,参加し忘れた人がいるので面倒化もしれませんが行ってください.

\subsubsection{2023年度のミーティング}
2023年度はコロナが収まったため対面で週に1回行っていました.
先生方の都合により時間が前後する場合がありますが,基本的にはミーティングを行う曜日と時間は固定していました.
ミーティングを行う日時が決定したらgoogleカレンダーに書き込み,zoomリンクも一緒に設定してください,
何らかの理由でミーティングを欠席する人がzoomで参加する場合があるので,必要に応じてzoomを立ち上げてください.

室長はミーティング開始時刻になったら先生方を呼び,前に立って司会進行を務めます.
連絡事項は学生→関口先生→野中先生の順で伝えてもらっていました.
議事録は副室長などにとってもらい,まとめたファイルをslackに流してください.
また,全体ミーティングについてはslackbotを用いてリマインダーを設定することをお勧めします.

\subsection{発表会・報告会の出欠確認}
発表会と報告会の出席は卒業要件に含まれているため,室長は当日の出欠状況を確認し,Excelなどにまとめてください.
また,出欠状況をまとめたファイルは,発表資料と同じ階層にアップロードしてください.(場合によっては関口先生に直接送るよう指示されるかもしれません.)


%\subsection{ワークスペースのlabDataに必要なファイルを作る}
%院ゼミや報告会PPT・報告書など,必要に応じてファイルを作成します.手順を以下に示します.
%\begin{enumerate}
% \item labdataを開く
% \item 新しいフォルダを作る(名前は他人が見てすぐ分かる様に)
% \item フォルダを右クリックして,プロパティを押す
% \item 項目に「セキュリティ」があるので押す
% \item 「編集」を押す
% \item アクセス許可のフルコントロールの項目に「許可」・「拒否」があるから「許可」をクリック
% \item 「OK」を押す
%\end{enumerate}

\subsection{卒論・修論の製本カバーを買う}
論文の製本作業に関係する資料はサーバーのWorkSpaceの
\par
/workspace/Work2020/makino/引き継ぎ資料/室長関係/製本作業
\par
にあります.
\par
論文が仕上がる前に論文を包む製本カバーを買わなければなりません.購入場所は14号館地下1階です.ちなみに,2019年度からPDF提出のため製本は希望者のみになっています.2020年度は希望者が少なかったため研究室にある在庫のみで対処したため購入はしていないです.

手順として,2月下旬に卒論の仮提出があります.そこから追加されるページ数を各学生は検討しているので,論文の最終的なページ数に応じたカバーの厚さをB4・M2の印刷希望者に聞いてください.製本カバーの厚さは論文のページ数によって決まります.例えば論文10枚で製本カバーが1mmを購入といった形です.しかし論文は両面印刷するので,本来のページ数を半分にして計算してください.背幅サイズは1,2,3,4,5,6,9,12,15,18,21,24,27,(30)mmがあるとのことです.

購入する際,ミスがあったときのために予備分を足して発注してもいいと思います.また,去年以前の余りも総研学生室にあるので,先輩に聞いてみてください.発注なら1週間,もしくは2週間あれば確実に渡せると一昨年の売店の方は言っていました.製本カバーの色は白,オリーブ,ミントグレー,ベージュ,ダークグレー,オーシャンブルー,サックスブルー,パステルブルーの8種類で,2017年度の色はダークグレーでした.全員印刷で提出だった時は色はそろえていたようですが,今は記念用なので厚みを調査するときに同時に色の希望もとると良いと思います.

製本は学科事務室で圧着機を借りるので,カバーだけ買います.現在は主に総研で活動しているため等々力キャンパスでも借りれるかもしれないので,来年度は確認してみてください.
購入時のお金についてですが去年は''研究室に配分された予算(名義は関口先生)''としました.変更がある可能性もあるので事前に購入時のお金が2017年度と同じなのかを関口先生に確認を取ってください.圧着機のレンタル期間ですが事前に先生方にレンタル期間について相談してください.例年だと卒論最終稿の提出における翌日からの5日程度ですが,2017年度はMSCSで印刷物が多くなる関係でレンタル期間が後ろにずれました.2019年度は3月の1週目にレンタルしました.2020年度はコロナの緊急事態宣言が終わる予定だった3/8から4日間としてました.

製本作業における論文の印刷ですが,2018年度以前は情報基盤センターの各自の印刷ポイントを全て使用し,足りなかった分を研究室で印刷していたようです.2019年度は希望者のみの製本作業であったため,研究室のプリンタで印刷を行いました.レンタル後テンプレートに基づいて印刷したタイトルを背表紙に透明テープで貼り付けます.背表紙についてですが,上記のサーバーのフォルダ内のWordファイルにあります.

背表紙のフォントは設定されているMS明朝の12ptで,縦書きです.数字・英字がある場合には縦中横を使って向きを縦にあわせてください.タイトルが長く,1行では収まらない場合は,まず名前の前後のTabスペースを全角1つ分にして収まるかどうか確認してください.それでも収まらない場合はスペースを半角へ,それでもダメだった場合はフォントを1段階ずつ小さくしていってください.名前の間にスペースは入れません.また,行間は1.5に設定しています.

2019年度の予定を示します.2020年度は少し遅れたので,こちらを参考にしてください.
\begin{table}[h]
\centering 
\begin{tabular}{|c|c|}
\hline
\textbf{日付}  & \textbf{内容} \\ \hline
12/31        & 第1稿提出       \\ \hline
1/31         & 第2稿提出       \\ \hline
2月2週目        & 製本カバー余り確認   \\ \hline
2/28         & 論文提出(PDF)   \\ \hline
3/2          & 製本機レンタル     \\ \hline
3/2$\sim$3/6 & 製本作業(希望者)   \\ \hline
\end{tabular}
\end{table}

\newpage%改ページ
\subsection{1年間のスケジュール}
室長が関わる研究室の1年間の予定を表で示しておきます.なお,下記の予定は2020年度のスケジュールです.コロナの影響で全体的に遅れています.
\begin{table}[h]
\centering 
\begin{tabular}{|c|c|}
\hline
\textbf{日付} & \textbf{スケジュール} 							\\ \hline
4/2       	& 研究室ガイダンス        								\\ \hline
6/16      	& オープンキャンパス(中止)       						\\ \hline
8/11      	& 研究室紹介(オンライン)      						\\ \hline
8/14      	& 研究室紹介(オンライン)       						\\ \hline
8/23,24  	& オープンキャンパス(オンラインのため不参加)       \\ \hline
9/24      	& 研究室配属会          								\\ \hline
9/29      	& 初回事例研究          								\\ \hline
2月中    	& 論文製本作業準備        							\\ \hline
3月初週 	& 製本作業            									\\ \hline
\end{tabular}
\end{table}

例年通りのスケージュールとして2019年度のスケジュールも示しておきます.
\begin{table}[h]
\centering 
\begin{tabular}{|c|c|}
\hline
\textbf{日付} & \textbf{スケジュール} 		\\ \hline
4/2         & 研究室ガイダンス        		\\ \hline
6/16        & オープンキャンパス       		\\ \hline
7/19        & 研究室紹介						\\ \hline
7/23        & 研究室紹介          			\\ \hline
8/2,3       & オープンキャンパス       			\\ \hline
9/19        & 研究室配属会          		\\ \hline
9/24        & 初回事例研究          		\\ \hline
2月中         & 論文製本作業準備        	\\ \hline
3月初週        & 製本作業            		\\ \hline
\end{tabular}
\end{table}
\subsection{最後に}
様々な仕事を任されてると思いますが,決して一人で抱え込まずにみんなで協力して取り組みましょう.
自分は室長の仕事をネガティブな気持ちで取り組むことが多く,せっかくの機会だからもっとまじめに取り組めばよかったと後悔しています.
ストレスがかかる場面も多いとは思うけど,ぜひポジティブな気持ちで取り組んでほしいなと思います.
きっといい経験になると思います.頑張ってください.


\end{document}
